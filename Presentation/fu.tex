\documentclass{beamer}
\usepackage{graphics}
\usepackage{beamerthemesplit}
\setbeamercovered{transparent}
\usepackage{tikz}
\usepackage{verbatim}
\usetikzlibrary{arrows,shapes,backgrounds}
\usepackage{multicol}

\title[Voting lines in Florida]{\small Voting lines, equal treatment,\\and early voting check-in
  times in Florida} \author{David Cottrell\inst{1} \and Michael C.\
  Herron\inst{1} \newline \and Daniel A.\ Smith\inst{2}}
\institute{\inst{1} Dartmouth College \and %
  \inst{2} University of Florida} \date{\today}

\begin{document}
	\frame{\titlepage}
	\section{Background}
	
	\begin{frame}
		\begin{columns}[t]
			\column{.5\textwidth}			
			\centering
			\includegraphics[height=  .7\textwidth]{../Plots/columbus.jpg} \\		
			\includegraphics[height=  .7\textwidth]{../Plots/miami_line.jpg}\\
			\column{.5\textwidth}
			\includegraphics[height=  .5\textheight]{../Plots/scottsdale_line.jpg}\\
			\footnotesize{
				(Top Left) Columbus, OH during early
                                voting in 2012 General Election \\ \vspace{3mm}
				(Bottom Left) Miami, FL on 2012 Election Day \\ \vspace{3mm}
				(Top Right)  Scottsdale, AZ during primary on March 22, 2016
				}
		\end{columns}
	\end{frame}
	
	\begin{frame}
		Most voters vote in person (75\% in 2016).  \\~\\
	
		The existence of voting lines is a major concern \ldots
			\begin{itemize}
				\item[1.]<1-> Opportunity costs - essentially a time tax {\tiny (Mukherjee 2009)}
				\item[2.]<1-> Downstream consequences  {\tiny (Pettigrew 2017)}\\~\\
			\end{itemize} 
		We want to know \dots
			\begin{itemize}
				\item[1.]<1-> Who bears the burden of this tax? Is it fairly distributed?
				\item[2.]<1-> Does waiting in line
                                  discourage future participation?  Or
                                  possible encourage future participation?
			\end{itemize}
	\end{frame}

		
			
	\begin{frame}
          Challenge: data availability 
          \begin{itemize}
          \item<2-> Need to identify those who waited in line, their characteristics, AND link them to future behavior data availability.\\~\\
          \end{itemize}
          Scholars of lines have relied on two sources:
          \begin{itemize}
				\item[1.]<3-> Election Day exit polls or post-election surveys (SPAE, CCES) {\tiny (Stewart 2013 and 2015)}
					\begin{itemize}
						\item<3-> {\small 10\% of voters waited longer than 30 minutes in 2012}
						\item<3-> {\small Limitation: self-reporting and social desirability bias}
					\end{itemize}
				\item[2.]<4> Closing times of polling location and aggregate demographic data {\tiny (Herron and Smith 2014; Pettigrew 2016)}
					\begin{itemize}
						\item<4-> {\small Waiting in line reduced voter turnout by 1\%}
						\item<4-> {\small Limitation: aggregate data and ecological fallacy}
					\end{itemize}
			\end{itemize}
	\end{frame}
		
	
	\section{Data}
	\subsection{About the data}
		\begin{frame}
                  We turn to a unique data source produced by Electronic Voter iDentification (EViD) machines during the early voting period for Florida's general election. 
		 	\begin{itemize}
				\item<2-> Voters can cast their ballots at any polling location within a county during early voting in Florida.
				\item<2-> EViD machines allow pollworkers to check-in and verify registration statuses electronically, then synchronize across locations.
				\item<2-> Used to thwart double
                                  voting.
				\item<3-> Early voting is distinct in
                                  some ways from Election Day voting.
			\end{itemize}
		\end{frame}
		
		\begin{frame}
		EViD data:
			\begin{itemize}
				\item<2->Timestamps of individual votes during early voting period in General Elections in 2012 and 2016.
				\item<2->6 counties in Florida: Alachua, Broward, Hillsborough,  Miami-Dade, Orange, Palm Beach.
				\item<2->8 days, 78 locations, and 942,194 early voters in 2012.
				\item<2->14 days, 104 locations, and 1,687,304 early voters in 2016.
			\end{itemize}
			\begin{itemize}
				\item<3->[] Polls open at 7am and close at 7pm.  Anyone in line before 7pm can vote.
				\begin{itemize}
					\item This is key!! We know
                                          that anyone who voted past
                                          7pm waited to voted.
				\end{itemize} 
			\end{itemize} 
		\end{frame}
			
		\begin{frame}
		Pros of EVID data
			\begin{itemize}
				\item[1.]  We can leverage closing times to identify individuals who waited in line to vote during early voting in 2012. 
				\item[2.]  Individual voters can be linked to the Florida voter file, including
                                  \begin{itemize}
                                  \item  demographic characteristics (voter extract). 
                                  \item  prior and subsequent voting behavior (voter history). 
                                  \end{itemize}
                                \end{itemize}
		Cons
			\begin{itemize}
				\item[1.] Does not identify arrival time for those who voted.
				\item[2.] Does not identify
                                  individuals who left before voting.
				\item[3.] And does not identify
                                  individuals who were deterred by an
                                  existing line.
			\end{itemize}
		\end{frame}
		
		\begin{frame}
		I will proceed as follows \ldots
			\begin{itemize}
				\item[1.] The 2012 General Election.
				\item[2.] How things changed in 2016.
				\item[3.] Estimating the effect of waiting in line on future turnout.
			\end{itemize}
		\end{frame}
		
		\subsection{2012}
		
		\begin{frame}				
		\centering 2012
		\end{frame}
		
		\begin{frame}
		 Number of locations where early votes were cast			
			\centering
			\includegraphics[height=  .8\textheight]{../Plots/number_of_locations.pdf}
		\end{frame}
	
		\begin{frame}	
		Distribution of early voting check-in times by hour			
			\centering
			\includegraphics[height=  .8\textheight]{../Plots/histogram_by_hour.pdf}
		\end{frame}
		

		\begin{frame}				
		\centering Lines tend to affect minority voters \ldots
		\end{frame}

		\begin{frame}			
			\centering 
			Racial composition of early voters by hour	\\
			\includegraphics[height=  .8\textheight]{../Plots/racial_composition.pdf}
		\end{frame}

%		\begin{frame}				
%		Lines tend to affect Democrats \ldots
%		\end{frame}
%
%		\begin{frame}			
%			\centering 
%			Partisan composition of early voters by hour and by race	\\
%			\includegraphics[height=  .8\textheight]{../Plots/partisan_composition_by_race.pdf}
%		\end{frame}
%		
%		
		
		\subsection{2016}
		
		
		\begin{frame}				
			\centering 2016
		\end{frame}
		
		\begin{frame}				
		\centering 
		Number of locations where early votes were cast, 2016
		\includegraphics[height=  .8\textheight]{../Plots/number_of_locations_2016.pdf}
		\end{frame}
		
		\begin{frame}				
		\centering 
		Number of locations where early votes were cast, 2012
		\includegraphics[height=  .8\textheight]{../Plots/number_of_locations.pdf}
		\end{frame}
		
		\begin{frame}				
			\centering 
			Distribution of voter check-ins
		\includegraphics[height=  .7\textheight]{../Plots/histogram_by_hour_by_race_2012_2016}
		\end{frame}


                \section{Results}
	
		\begin{frame}
		Does waiting in line to vote in the current election reduce one's propensity to vote in the future?
		\end{frame}

		\begin{frame}
                  How do we identify who waited in line?
		\end{frame}
 
		\begin{frame}				
			\centering 
			Fred B.\ Karl County Center, Hillsborough County\\
			\includegraphics[height=  .35\textheight]{../Plots/example00.pdf} \\
			West Kendall Regional Library, Miami Dade County\\
			\includegraphics[height=  .35\textheight]{../Plots/example01.pdf} \\ 
		\end{frame}

		\begin{frame}
                  We use a 7:30pm rule, which is conservative but not perfect.
		\end{frame}

		\begin{frame}
                  Voters who waited may be different than voters who
                  did not wait.  We condition on covariates from the
                  Florida \emph{voter file}.
		\end{frame}

		\begin{frame}
                  Florida voter file
                  \begin{itemize}
                    \item <2-> Lists all registered Florida voters.
                    \item <3-> Contains demographics (``extract'') and
                      voting history (``history'').
                    \item <4-> Does not contain any income/wealth
                      variables.
                    \item <5-> The party in a voter file is a party of
                      registration.
                    \item <6-> Identification numbers allow linking
                      across years, in our case, from 2008 to 2016.
                    \end{itemize}
                  \end{frame}


	
		\begin{frame}
		\centering 
		\includegraphics[height=  .2\textheight]{../Plots/logit}			
		\end{frame}

		\begin{frame}				
			\centering
			Probability of voting in 2016, given 2012 check-in time \\
			(Black, male, 50-60 yo, Democrat, did not vote in '08)
			\includegraphics[height=  .9\textheight]{../Plots/probability_of_voting_in_2016_over_under.pdf}
		\end{frame}
		
		
		\begin{frame}				
                  \centering
			Probability of voting in 2016, given 2012 check-in time \\
			(White, male, 50-60 yo, Democrat, did not vote in '08)
			\includegraphics[height=  .9\textheight]{../Plots/probability_of_voting_in_2016_over_under_white.pdf}
                      \end{frame}
                
                      
                      \begin{frame}	
                        Average effects in the complete sample
                        \footnotesize\input{../Paper/table_out_ate.tex}
                      \end{frame}

                      \begin{frame}	
                        Young voters excluded\ldots this probably
                        makes our results conservative.
                      \end{frame}
                      



		\begin{frame}				
                  \centering
			Probability of voting early in 2016, given 2012 check-in time \\
			(White, male, 50-60 yo, Democrat, did not vote in '08)
			\includegraphics[height=  .9\textheight]{../Plots/probability_of_voting_in_2016_over_under_early.pdf}
                      \end{frame}




              \section{Conclusion}
	
		\begin{frame}
			Conclusion
			\begin{itemize}
				\item[1.] Voting lines were in some location in Florida were severe in 2012.  Line lengths improved in 2016.
				\item[2.] Certain voters---specifically minorities and Democrats---are more likely to be affected by long wait times. 
				\item[3.] Holding all else constant,
                                  waiting in line results in about a
                                  1\% decrease in turnout.
				\item[4.] Waiting in an early voting line
                                  results in less early voting.
			\end{itemize}
		\end{frame}
		
		\begin{frame}
		\centering
			\includegraphics[height=  \textheight]{../Plots/result}
		\end{frame}
		
			
\end{document}


