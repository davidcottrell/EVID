\documentclass[12pt]{article}
\sloppy
\usepackage{times}
\usepackage{xr}
\usepackage{url}

\externaldocument{evid}

\begin{document}
\thispagestyle{empty}

\noindent {\Huge Memo}\\[0.5in]

\noindent
\begin{tabular}{ll}
  \textbf{To:} &  Editors, \emph{State Politics \& Policy Quarterly}\\
  %% \textbf{From:} & Authors here\\
  \textbf{Date:} & \today \\
  \textbf{Re:} & Revision of Manuscript SPPQ-18-0089,\\& ``Voting lines, equal treatment, and early voting check-in times in Florida''
\end{tabular}


\vspace{.1in}
\hrule
\vspace{.2in}

\noindent This memorandum describes changes we have made to our
manuscript in light of a referee report. We very much appreciate the
thoughtful feedback provided to us and the opportunity to submit a
revision of our manuscript.

\begin{itemize}

\item \emph{I will commend the authors: it is a rare joy in reviewing
    manuscripts to find an ms that when I ask a question, the answer
    is provided in the next paragraph just about every time. It was
    that way with this one. The literature review does a fair and good
    job placing the study amongst the voter survey literature and then
    discussing the contributions, assumptions, and limitations of this
    work. This work is not oversold and/or held up as a panacea as so
    many ms's are these days. Thank you for that.}  We very much
  appreciate the referee's positive reaction to our manuscript and
  hope that our revision is even better.

\item \emph{I might have done a couple of things differently on county
    selection--namely including a few more to get a more diverse set
    (e.g., on the urban vs.\ rural and perhaps more v.\ less educated
    front instead of maximizing for population).}  We appreciate being
  pushed in this direction and have again reached out to counties to
  try to get more data.  Based on this new data collection effort, we
  received EViD timestamps from six more county Supervisors of
  Elections: Calhoun, Hernando, Levy, Manatee, Osceola, and Putnam.
  Calhoun, Levy, and Putnam counties are all in north Florida. All are
  rural, which we point out in the revised manuscript. Calhoun County,
  for example, is in the Florida Panhandle and has fewer than 9,000
  registered voters.  Moreover, these three counties have populations
  with lower than average levels of educational attainment.  Also in
  terms of education, Alachua County, which was included in the
  manuscript's initial submission, is exurban, home to the University
  of Florida and a well-educated population; this county nonetheless
  has one of the highest GINI coefficients in Florida.

\item \emph{There's a *lot* of descriptive evidence/results in this
    paper. I find that valuable, and just about each piece of evidence
    is interesting and needed to get to an understanding of what's
    going on. I only mention this because obviously it adds length,
    and I'm sure the editors might think about encouraging the authors
    to put at least some of this into an appendix. I think the flow is
    pretty good as it is, but space is always a concern.}  We agree.
  We have cut two pages from the initial submission; the revision is
  now 38 total pages and the body of the manuscript, 29 pages.  We
  moved some material to an appendix, including a new table, and are
  hopeful that this has helped with the length of the manuscript.

\item \emph{With regard to the data and choices made about research
    design and measurement, I really didn't have many concerns and I
    thought the authors' choices made sense.  I do have to wonder if
    different measurement choices would have affected the results, but
    I can't really throw cold water on the ms for the decisions the
    authors made.} With our additional counties, our findings remain
  significant with no substantive changes.

\item \emph{More importantly, the inferences drawn from these results
    prior to the regressions are valid, which leads up to the logits
    which are on-point and well-specified, and the inferences and
    conclusions drawn from them make sense.} We thank the reviewer for
  his/her support of our modelling choices.


\item \emph{It's a neat paper that uses interesting data to augment
    our understanding of a political phenomenon. I am going to suggest
    an R\&R to the editors; best of luck to you.}  Again, we thank the
  reviewer for supporting our research.

\end{itemize}
\end{document}
