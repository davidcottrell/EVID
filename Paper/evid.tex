\documentclass[12pt,titlepage]{article}

% \usepackages21{fancyhdr}

\usepackage[printwatermark]{xwatermark}

\usepackage{grffile}
\usepackage{xcolor}
\usepackage{lipsum}
\usepackage{times}
\usepackage{soul}
\usepackage{epsfig}
\usepackage{rotating}
%\usepackage[hyphens]{url}
%\usepackage[breaklinks=true]{hyperref}
\usepackage{hyperref}
\usepackage{etoolbox}
\appto\UrlBreaks{\do\-}

\usepackage{latexsym}
\usepackage{graphicx}
\usepackage{amsfonts}
\usepackage{amsmath, amsthm, amssymb}
\usepackage{setspace}
\usepackage{natbib}
\usepackage{longtable}
\usepackage{keyval}
\usepackage{caption,subcaption}
\usepackage{arydshln}


\providecommand{\keywords}[1]{\textbf{\normalsize{Keywords: }} \normalsize{#1}}

%% set 1-inch margins
\usepackage{fullpage}

%% APSR submission: no commas in citations between name and year
%% See http://merkel.zoneo.net/Latex/natbib.php
\bibpunct{(}{)}{;}{author-year}{}{;}

% the opening bracket symbol, default = (
% the closing bracket symbol, default = )
% the punctuation between multiple citations, default = ;
% the letter `n' for numerical style, or `s' for numerical superscript style, any other letter for author-year, default = author-year;
% the punctuation that comes between the author names and the year
% the punctuation that comes between years or numbers when common author lists are suppressed (default = ,);

\usepackage{footmisc}
\renewcommand{\footnotelayout}{\doublespacing} % set spacing in footnotes
\newlength{\myfootnotesep}
\setlength{\myfootnotesep}{\baselineskip}
\addtolength{\myfootnotesep}{-\footnotesep}
\setlength{\footnotesep}{\myfootnotesep} % set spacing between footnotes

% make footnote font size same as regular font size in text
\renewcommand{\footnotesize}{\normalsize} 

%% make possessivecite macro since this does not exist in natbib
\newcommand{\possessivecite}[1]{\citeauthor{#1}'s (\citeyear{#1})}

%% Use this for a "DRAFT" watermark
% \newwatermark[allpages,color=pink!30,angle=45,scale=5,xpos=-25,ypos=40]{DRAFT}

%% List all locations for graphics here
\graphicspath{ {../Plots/} }
\begin{document}
\sloppy
\thispagestyle{empty}

%% APSR submission requires double-spaced footnotes
%%\newcommand{\footnote}[1]{\footnote{\doublespacing #1}} %% <-- note \doublespacing here.

\renewcommand{\topfraction}{.85}
\renewcommand{\bottomfraction}{.7}
\renewcommand{\textfraction}{.15}
\renewcommand{\floatpagefraction}{.66}
\renewcommand{\dbltopfraction}{.66}
\renewcommand{\dblfloatpagefraction}{.66}

% \urldef\myurlncsl1\url{foo%.com}
% \begin{document}
% text\footnote{WWW: \myurl}


\title{Voting lines, equal treatment,\\and early voting check-in times
  in Florida\thanks{Earlier versions of this manuscript were presented
    at the 2017 Annual Meeting of the Midwest Political Science
    Association, Northwestern University, and the Free University
    of Berlin.  The authors thank Evan Morgan for research assistance,
    Michael Hanmer and Hollye Swinehart for comments on earlier
    drafts, and staff members from various Florida Supervisor of
    Elections offices for data on early voting check-in
    times.}\author{David Cottrell\thanks{Lecturer, Department of
      Government, Dartmouth College.  6108 Silsby Hall, Hanover, NH
      03755 (\texttt{david.cottrell@dartmouth.edu})} \and Michael C.\
    Herron\thanks{Professor of Government, Dartmouth College.  6108
      Silsby Hall, Hanover, NH 03755-3547
      (\texttt{michael.c.herron@dartmouth.edu}).} \and Daniel A.\
    Smith\thanks{Professor of Political Science, University of
      Florida, 234 Anderson Hall, Gainesville, FL 32611-7325
      (\texttt{dasmith@ufl.edu}).}}\vspace{1cm}\\\keywords{voting
    lines, equal treatment, election administration, turnout, early
    voting}}

%\title{Voting lines, equal treatment,\\and early voting check-in times
 %in Florida}


\maketitle \doublespacing 



% \topskip0pt
% \vspace*{\fill}
% \begin{center}
%   \Large{\textbf{Word count: 8,790}}
% \end{center}
%  \vspace*{\fill}

%Moreover, long lines may \mbox{additionally} influence the willingness
%of eligible voters to participate in future elections.


\begin{abstract}
  \noindent 
  Expansive lines at the polls raise the cost of voting and can
  precipitate unequal treatment of voters.  Despite the importance of
  lines, research on them is hampered by a fundamental measurement
  problem: little is known about the distribution of wait times across
  voters.  We argue that early, in-person voter check-in times from
  Florida---942,166 check-ins from the 2012 General Election and
  1,687,217 from 2016---allow us identify individuals who waited in
  line to vote.  We highlight disproportionately long wait times
  incurred by minority voters in 2012 and show that Florida early
  in-person voters who waited excessively in 2012 had a slightly lower
  probability---approximately one percent---of turning out to vote in
  the 2016 General Election, \emph{ceteris paribus}.  These
  individuals also had slightly lower turnout probabilities in the
  2014 Midterm Election, \emph{ceteris paribus}.  Our results draw
  attention to the ongoing importance of the administrative features
  of elections that influence the cost of voting and ultimately the
  extent to which voters are treated equally.
\end{abstract}


% > evid %>% group_by(year) %>% count
% # A tibble: 2 x 2 # Groups: year [2] year n <int> <int> 1 2012
% 942166 2 2016 1687217

\newpage
\section*{Introduction}

Modern democracies are characterized by regular and free elections,
the legitimacy of which can suffer in the face of perceptions of
unfairness and unequal treatment of voters \citep{norris2014electoral}.
An election can fail to treat all voters equally if, for example,
there are meaningful differences in vote tabulating technologies
across an electorate \citep[e.g.,][]{kimballkropf:tech}; if some
groups of voters must contend with registration hurdles that others do
not \cite[e.g.,][]{ansolhersh:registration}; if some voters face
strict photo identification requirements and others no such
requirements \citep[e.g.,][]{benteleetal:newjimcrow}; and, if some
voters have to travel much longer distances than others to cast their
ballots \citep[e.g.,][]{dyckgimpel:distance}.  More generally, when in
a given election the \emph{cost of voting} in the sense of
\cite{downs:econtheory} varies significantly by voter group---where
group membership could be based on race/ethnicity or party
affiliation, for example---the principle of equal treatment of voters
is at risk.

%   Elections affected by fraud certainly fall in this
% category, as \citet{cottrelletal:fraud2016} argue, and elections
% conducted in the midst of acrimonious debates over the nature of the
% franchise risk being thought of as unfair \citet{hasen:votingwars}.
% Fairness can also be in the eye of the beholder:
% \citet{sancesstewart:confidence} show how confidence in a given
% election outcome can be a function of whether an individual's
% preferred candidate won.

Here we explore how voting lines, whose lengths reflect interactions
between voter turnout and administrative decisions about resource
allocation \citep{herronsmith:hanoverstudy}, are related to equal
treatment.  The connection between lines and equal treatment is
straightforward: if lines adversely affect some types of voters more
than others, then the cost of voting will vary across voter groups.
This would threaten the principle of equal treatment of voters.

% Thus, any assessment of
% the extent to which a given election featured equal treatment of
% voters must include an analysis of voting lines and whether certain
% groups of individuals waited in disproportionately long  lines.

%  In-person voting experiences have a number of
% features that are distinct from the experiences of those who vote by
% mail, and our interest here is the act of waiting in line.

% Casting an in-person ballot requires traveling to
% a voting precinct; possibly spending time in a line prior to voting;
% authenticating oneself in some fashion to election officials;
% physically registering candidate and ballot measure preferences; and,
% submitting a ballot for tabulation.  

%% 2016 EAC is from p. 10
%% really think we need to stick with polling station terminology, not precincts, as early voting is not precinct-based.

% as opposed to lines in
% which voters, already in the act of casting ballots, are forced to
% stand. 

Our analysis considers two aspects of voting lines.  First, waiting in
line to vote constitutes a time tax \citep{mukherjee:timetax}. This
tax can be negligible---for example, a voter waits a scant number of
seconds prior to initiating her voting process---or imposing---some of
the Florida voters we describe shortly waited several hours to vote in
2012. Associated with a voting line time tax are the distributional
questions associated with all forms of taxation: is the burden of this
tax spread uniformly across voters or concentrated on selected groups?

Second, waiting in line to vote may have ``downstream'' electoral
effects in the sense of
\citet{pettigrew:longlinesminorityprecincts}. When an individual is
compelled to pay a relatively high tax in order to participate in a
social or political activity, a natural response might be to avoid the
activity, or substitute for it, in the future.  If voting lines
operate in this fashion, then they will depress voter turnout.  And if
lines affect some voter groups more than others, then their associated
depressive effect on turnout will not burden all voters equally.
Unequal burdens notwithstanding, given the relatively low rate of
voter participation in American presidential elections---approximately
58 percent of the voter eligible population turned out in 2012, and 60
percent in 2016---any aspect of the voting experience that might have
a depressive effect on turnout should be considered a matter worthy of
study.\footnote{The national turnout rate for a general election
  depends on the way that the number of eligible voters in the country
  is calculated.  The figures cited here are from the United States
  Election Project, available at
  \url{http://www.electproject.org/home} (last accessed December 11,
  2017).\label{fn:uselectionproject}}

% To make matters
% concrete, if a line in front of a particular restaurant is long, some
% individuals may substitute an alternative eatery for the restaurant in
% question, one without an imposing line, or even forego
% eating.\footnote{Alternatively, a line in front of a restaurant might
%   be a quality signal, in which case the presence of a line might be
%   an incentive to wait in it.  The same could be said for voting, and
%   here we ignore the possibility that lines form due to potential
%   voters using the presence of lines as a means to gather information
%   on whether voting is valuable.}

%  While it is not
% \emph{a priori} clear that voting can be substituted for in the way
% that selecting a restaurant might be, we should not dismiss outright
% the possibility that the effect of waiting in line to vote, which as
% we have already argued raises the cost of voting, might decrease the
% likelihood of voting in the future.  

The time tax associated with voting lines and their potential
downstream consequences are conceptually distinct. The former could be
uniformly spread across voters, which might be normatively pleasing in
the sense of not violating equal treatment of voters, and yet there
could be a significant effect of waiting in line on future electoral
participation. Or, the time tax could be concentrated on certain voter
groups, violating equal treatment regardless of whether there are
downstream consequences of waiting to vote.

Given the relationship between voting lines and potential violations
of equal treatment, one would expect a comprehensive literature to
have developed on this subject.  This has not happened, and the reason
is as follows. Notwithstanding post-election surveys of voters aimed
at understanding their experiences at the polls, little is known about
the distribution of wait times that affect voters. This measurement
problem, we argue, is the most meaningful impediment to voting lines
research.

To address what is ultimately a measurement issue, we turn to a
relatively untapped source of electoral information: official early
voter check-in times from Florida counties in the 2012 and 2016
General Elections.  We provide details shortly on check-in times, but
for the moment it suffices to note that a voter's check-in time is
when he or she began to vote.  This said, the hundreds of thousands of
check-in times that form the basis of our research design have several
valuable features: they are generated by machines installed in voting
locations, are not subject to voter self-reporting biases, and are
linkable to public records on race/ethnicity, party registration,
gender, age, neighborhood income, and previous voting history.

Still, in the interests of full disclosure we want to emphasize at the
outset that a voter's check-in time by itself cannot specify the
length of the line at the polls in which said voter waited before
checking-in to vote.  Rather, in what follows we offer a set of
assumptions under which check-ins can be used to assess the prevalence
of voting lines.  Our assumptions are not costless, of course, but
they differ from key assumptions in the extant voting lines
literature, which assumes that voters can accurately remember, and
when so queried, recount their wait times.  All research designs have
limitations, ours included.  Invoking triangulation, we argue here
that the literature on voting lines and election administration more
generally will be strengthened to the extent that varied research
designs invoke alternative sets of assumptions.

% With this in mind, our empirical
% analysis of voting lines during early in-person voting in Florida is
% divided into two sections. First, we assess the types of voters who in
% 2012 experienced voting lines prior to casting their ballots. Second,
% we consider whether individuals who waited excessively to vote in 2012
% were less likely to vote in subsequent elections, namely, in 2014 and
% 2016.

% One of the difficulties in studying voting lines in the United States
% is data availability: it is not easy to determine precisely how long
% voters waited in line at the polls.  It is accordingly difficult to
% study if there are consequences for subsequent voting behavior of
% waiting in line.  Some researchers have dealt with this hurdle by
% using surveys---either Election Day exit polls or post-election
% questionnaires---which integrate responses to waiting time questions
% with other survey items, namely socioeconomic and partisan queries
% \citep{stewart:waitingtovote2012}.  Post-election surveys on voting
% experiences can foster access to wide swaths of voters, and this is
% valuable.  A limitation of surveys regarding electoral experiences,
% though, is that they rely on self-reports of wait times, which may not
% be accurate, and in addition it is known that self-reporting of
% turnout in general can suffer from over-reporting
% \citep{karpbrockington:overreport,bellietal:overreport}.

%In light of potential concern with survey-based evidence on voting
%lines,

% Still, in the interests
% of being transparent about limitations in our research design, a
% voter's check-in time is not directly connected to the amount of time
% she spent waiting in line.  We describe below how we link voter
% check-in times with corresponding waiting times, but the connection is
% not perfect.

Regarding the time tax associated with waiting to vote, we find a
disproportionate concentration of this tax on minority voters.  The
situation was worse in Florida in 2012 than in 2016, and in general
our collection of check-in times highlights fewer troubling issues in
the more recent presidential election.  This is consistent with
national, survey-based evidence on voting lines in the 2012 and 2016
General Elections.\footnote{For a national 2012 versus 2016
  comparison, see
  \url{http://electionupdates.caltech.edu/2016/12/14/this-just-in-lines-at-the-polls-shorter-in-2016-than-in-2012}
  (last accessed March 30, 2017), as well as
  \url{http://www.richmond.com/opinion/their-opinion/guest-columnists/fortier-and-palmer-column-who-waits-the-longest-to-vote/article_81efabad-aa23-577b-83d1-0e524cf844ef.html}
  (last accessed June 30, 2017).}  Whether the recent decrease in
waiting times reflects progress in Florida election administration,
substitution effects and voter sorting, or idiosyncrasies from either
the 2012 or 2016 General Election is not clear.  Two election-years of
data are unlikely to be sufficient to gauge broad progress in a state
as large and heterogeneous as Florida.

% conditional on our strategy for identifying those
% Florida early voters who suffered long waits in 2012, 

Regarding the effect of waiting to vote on future electoral
participation, we find small yet negative consequences for turnout in
the 2016 General Election.  As a consistency check on our 2016
results, we find similar effects of waiting in line in 2012 on turnout
in the 2014 Midterm Election.  The consequences of excessive line
waiting on turnout are not negligible insofar as long lines do appear
to be associated with lower participation levels in the future.  These
consequences are nonetheless small, in the neighborhood of one to
three percentage points, consistent with the findings of
\citet{pettigrew:longlinesminorityprecincts}.  We also find that early
voters in 2012 who waited in line to vote were less likely to vote
early in 2016, \emph{ceteris paribus}.

The next section situates our study in the literature on voting lines
and election administration.  We then turn to early voting check-in
times in Florida and explain what these times mean and how they can be
interpreted.  Our empirical results on lines are divided into two
sections: first we describe the distribution of the voting time tax
across our set of Florida voters, and second we analyze the downstream
effects of the tax.  We conclude with observations about congestion at
the polls and equal treatment of voters.

\section*{Voting lines and equal treatment in American elections}

Voting is a fundamental right in the United States.  Exercising this
right entails costs, as discussed by \cite{downs:econtheory} in his
formulation of the calculus of voting, and an excessively high cost of
voting can diminish the meaning of the right to vote.  Per the United
States Supreme Court in \emph{Reynolds v.\ Sims} (1964), ``[T]he right
of suffrage can be denied by a debasement or dilution of the weight of
a citizen's vote just as effectively as by wholly prohibiting the free
exercise of the franchise.''\footnote{See
  \url{https://supreme.justia.com/cases/federal/us/377/533/case.html}
  (last accessed December 12, 2017) for the text of the decision.}

In assessing the costs of voting that may debase or dilute the right
to cast a ballot, the high court has considered the fairness of an
assortment of election-related rules adopted by states.  These include
rules relating to redistricting (\emph{Shaw v.\ Reno} (1993)),
literacy tests (\emph{Oregon v.\ Mitchell} (1970)), proof of
citizenship (\emph{Arizona v.\ Inter Tribal Council of Arizona}
(2012)), voter identification (\emph{Crawford v.\ Marion County
  Election Board} (2008)), and the purging of registered voters
(\emph{Husted v.\ A.\ Philip Randolph Institute} (2018)). In various
cases, the Court has been asked to balance state interests with the
equal treatment of voters.  Although there is ample evidence that long
voting lines in Florida in the 2012 General Election
disproportionately affected racial and ethnic minorities
\citep{herron_smith2014}, as of yet the Supreme Court has not been
asked to rule on whether Section 2 of the Voting Rights Act applies to
excessively long wait times at the polls, which in principle could
constitute ``a denial or abridgement of the right of any citizen of
the United States to vote on account of race or color'' (52 U.S.C.\ \S
10301(a)).

A lack of Court attention notwithstanding, long voting lines have
received considerable attention in the last several election cycles in
the United States.  ``To me,'' Cynthia Perez complained in March 2016
after seeing a long line wending its way around an early voting center
in Maricopa County, Arizona, with those in line reporting standing in
line for more than three hours before checking in to vote, ``this is
not what democracy is about.''\footnote{Quote drawn from ``Angry
  Arizona Voters Demand: Why Such Long Lines at Polling Sites?''
  \emph{The New York Times}, March 24, 2016, available at
  \url{https://www.nytimes.com/2016/03/25/us/angry-arizona-voters-demand-why-such-long-lines-at-polling-sites.html}
  (last accessed December 15, 2017)} Ahead of her state's 2016
presidential primary, election administrators in Maricopa County (over
four million residents and one of the largest counties in the United
States) had cut the number of Election Day polling places by roughly
70 percent, increasing the strain on early voting
centers.\footnote{According to the 2012-2016 American Community
  Survey, 5-Year Estimates, the population of Maricopa County is
  4,088,549.}  Voters in a handful of states also experienced long
lines in the 2016 General Election, but their experiences were largely
mild compared to what transpired four years prior in
Florida.\footnote{On various states with line issues in 2016, see
  ``Why Long Voting Lines Could Have Long-Term Consequences,''
  \emph{The New York Times}, November 8, 2016, available at
  \url{https://www.nytimes.com/2016/11/09/upshot/why-long-voting-lines-today-could-have-long-term-consequences.html}
  (last accessed December 9, 2017).  On Georgia in particular, see
  ``Early voting lines are so long, people are fainting. That harms
  democracy,'' \emph{The Guardian}, October 19, 2016,
  \url{https://www.theguardian.com/commentisfree/2016/oct/19/early-voting-lines-georgia}
  (last accessed December 9, 2017).}  In 2011, the Florida state
legislature curtailed the state's early voting period, which resulted
in long lines for many electors who tried to vote in person during the
truncated early voting period in the 2012 General Election
\citep{herron_smith2014}.  Following the election and the seemingly
inevitable anger over long waits at the polls, then-President Barack
Obama formed a Presidential Commission in early 2013 to address
general election administrative issues in the United States.  In doing
so, he highlighted the plight of 102-year old Desiline Victor, a
Haitian-American woman who on October 27, 2012, waited in line for
nearly four hours at Florida's North Miami Public Library early voting
center.  Obama's Presidential Commission devoted some of its January
2014, report to long lines, concluding that, ``[A]s a general rule, no
voter should have to wait more than half an hour in order to have an
opportunity to vote'' \citep[p.\ 13,][]{pcea:2014}.\footnote{The
  executive order creating this commission is available at
  \url{https://obamawhitehouse.archives.gov/the-press-office/2013/03/28/executive-order-establishment-presidential-commission-election-administr}
  (last accessed July 12, 2017).}

Voting lines are potential issues wherever voters cast their ballots
in-person as opposed to voting remotely via mail.  The vast
majority---over 75 percent in 2016 and approximately 66 percent in
2012---of voters in the United States who cast ballots in presidential
elections do so in-person, either on an Election Day itself or during
a corresponding early voting period whose duration depends on relevant
local laws.\footnote{For the composition of the 2012 and 2016
  electorates, see \citet{eac:2012} and \citet{eac:2016},
  respectively.}  As of this writing of this paper, there only three
states---Colorado, Oregon, and Washington---in which almost every
ballot is cast via mail.\footnote{See ``EAVS DEEP DIVE: EARLY,
  ABSENTEE AND MAIL VOTING,'' United States Election Assistance
  Commission, October 17, 2017, available at
  \url{https://www.eac.gov/documents/2017/10/17/eavs-deep-dive-early-absentee-and-mail-voting-data-statutory-overview}
  (last accessed September 13, 2018).}


\subsection*{Survey-based research on voting lines}

% These anecdotes raise questions about voting lines in general---in
% particular, how long they tend to be and which voters have to wait in
% them, how much variance there is across jurisdictions in waiting
% times, and what consequences lines have, if any, beyond the cost of
% standing in line.  

With few exceptions \citep[e.g.,][]{spencermarkovits:renege,
  herronsmith:hanoverstudy, pettigrew:longlinesminorityprecincts},
much of what we know about wait times for voters is derived from
survey research.  Most notably, \citet{stewart:waitingtovote2012} and
his coauthors have conducted post-election, Internet-based surveys in
the United States following the last three general elections.  These
surveys query voters about their experiences at the polls and include
items on wait times.  \citeauthor{stewart:waitingtovote2012}'s
\emph{Survey of the Performance of American Elections} (SPAE), which
surveys 200 individuals in each state and Washington, D.C., and the
Cooperative Congressional Election Study (CCES) both contain questions
about voter wait times.  According to
\citeauthor{stewart:waitingtovote2012}, wait times across the country
vary considerably, but a state's average wait time tends to be
consistent over time. Floridians consistently report having to endure
some of the longest wait times in the country.  In the 2012 General
Election, in-person voters in Florida reported on average waiting 39
minutes to cast a ballot, three times the national average.


% Surveys like the SPAE can also help identify when during the voting process
% wait times are most likely to occur.  In the 2012 General Election,
% \citeauthor{stewart:waitingtovote2012} documented that over
% three-fifths of early in-person voters nationwide who reported waiting
% in line to vote said their wait was primarily during the check-in
% stage.

% Surveys are also helpful when trying to tease out the possibility of
% differential voting line time taxes across sub-populations of voters.

Drawing on 2008 CCES data, \cite{mukherjee:timetax} finds minority
voters were more likely, relative to white voters, to pay such a tax
when queuing to vote. \citet{kimball:voting}, using the 2012 SPAE,
reports that voters in urban areas faced longer lines than rural
voters, and \citet{pettigrew:racegapwaittimes} estimates that
typically non-white voting locations in the United States were
associated with voter wait times approximately twice as long as those
in white locations. In his study of the 2012 General Election, which
builds on the 2008 General Election research design of
\citet{alvarez:survey}, \cite{stewart:waitingtovote2012} finds that a
voter's race/ethnicity is an important ``individual-level
demographic'' that explains disparate wait times.  ``African Americans
waited an average of 23 minutes to vote,''
\citeauthor{stewart:waitingtovote2012} found, ``compared to only 12
minutes for Whites; Hispanics reported waiting 19 minutes, on
average.''  \citeauthor{stewart:waitingtovote2012} concludes that
these differences in wait times could be ``due to factors associated
with where minority voters live, rather than with minority voters as
individuals'' (pp.\ 457-458). Alternatively, \cite{herron:confidence}
draw on an innovative exit poll conducted across Miami-Dade precincts
in 2014 to gauge the experiences of voters as they were leaving their
polling stations, a nd report that those who waited longer in line had
less confidence in the electoral process.

The use of surveys in the study of voter wait times relies on a key
assumption: self-reports of wait times experienced by a voter in a
recent election are accurate.  In light of the literature on memory
distortion, this assumption is far from trivial. Memory confidence and
memory accuracy can be negatively associated
\citep[e.g.,][]{hirstetal:sept11memories}, how an individual thinks
about a given task can be related to the extent to which she thinks
about the passage of time \citep{conti:timeflies}, and one's enjoyment
of a task can depend on feelings about time progression
\citep[e.g.,][]{sackettetal:timeflies}.  In short, it is not obvious
that scholars interested in wait times should be optimistic that
voters in the aftermath of an election can accurately report how long
they waited to vote.  Wait times notwithstanding, surveys on voter
experiences assume that voters honestly report whether they turned out
to vote, and overreporting of turnout is a known problem
\citep{ansolhersh:bigdata}.

The point we are making here is that survey-based designs in the study
of voting lines impose assumptions on the problem they analyze.  The
research design we offer, of course, carries with it its own set of
assumptions, and in this way it is similar to survey-based designs.
However, as will be clear shortly, our design imposes a different set
of assumptions than survey-based designs, and most notably it does not
risk biases associated with memory and self-reporting.

\subsection*{An alternative to voter line surveys}

Our approach to voting lines draws on an alternative source of
information: observed check-in times of voters in Florida who cast
their ballots at early in-person polling sites prior to the 2012 and
2016 General Elections.  We explain below precisely what we mean by a
check-in time, but for the moment we note that in 2016 more Florida
voters---3.88 million---cast early in-person ballots than voters who
mailed in their absentee ballots (2.76 million) or who voted in person
at their local Election Day precincts (2.96 million)
\citep{FDOS:2016vote}.  The prevalence--represented by millions of
individuals--of early in-person voting is not unique to
Florida. Across the United States, early voting is increasingly
popular; for analyses of early voting reforms and their consequences
see \citet{neelyrichardson:earlyvoting}, \citet{gronkebaum:growth},
\citet{gronketoffey:psychological}, \citet{gronke:2012}, and
\citet{burdenetal:unanticipated}.

% \citet{gronke:earlyvotingreforms},


% Although there are debates in the
% literature about the extent to which early voting has changed the
% electorate, if at all, here we take this mode of voting as given.

Florida notwithstanding, there is good reason to focus on wait times
during early in-person voting.  Michael P.\ McDonald, on his Election
Project website, estimates that over 23 million voters cast early
in-person ballots in ``advance'' of the November 8, 2016, General
Election, some 17 percent of the 137 million votes
cast.\footnote{These numbers are drawn from the United States Election
  Project; see fn.\ \ref{fn:uselectionproject}.}  If one were to be
concerned that our focus on early voters in Florida limits our scope,
these figures imply that our results apply to millions of Americans.

In terms of our use of Florida voting data \emph{per se}, the state's
open records laws make it an excellent laboratory for the study of
election administration, as there is a wealth of information not only
about who votes and by what method of voting, but when and how
indivdiuals register to vote or cast ballots.  Drawing on Florida's
statewide voter registration records, \cite{herron_smith2013} trace
the ebb and flow of new voter registrations after the state
legislature placed restrictions on groups engaged in voter
registration drives; \cite{shinosmith:registrationtiming} demonstrate
that individuals who register to vote immediately prior to a
registration deadline are more likely to turn out in a proximate
general election, but not in subsequent elections; and
\cite{amos_etal2017} find differences in individual-level turnout
across political party and racial and ethnic categories among
registered voters whose precincts were eliminated or relocated by
local election administrators. In their study of congestion at the
polls in Florida in the 2012 General Election,
\cite{herronsmith:closingtimes} use precinct-level data detailing when
the last voter in a precinct checked-in to vote in a study showing
significant differences in the closing times of precincts within
counties, contingent on a precinct's racial and ethnic
composition. Similarly, by linking multiple statewide voter files and
individual-level early voting files, \cite{herron_smith2014}
demonstrate how reductions in the number of early voting days
following the 2008 General Election were associated with a drop in
early voting in 2012, especially among minority voters who had cast
ballots on the final Sunday of early voting in 2008.

%, a day of early voting that was eliminated by the state legislature ahead of the 2012 election.

While Florida's open records laws facilitate election administration
scholarship, with respect to direct election monitoring Florida grants
considerable privacy to voters.  Namely, with the exception of those
serving as candidate, political party, or ballot issue
representatives, state law prohibits precinct observers from tracking
voting processes inside a polling place when votes are actively being
cast. This effectively precludes scholars who want to understand why
lines form in front or inside Florida precincts from replicating
observational studies of voter activities in the vein of
\possessivecite{spencermarkovits:renege} study of California and
\possessivecite{herronsmith:hanoverstudy} research on New Hampshire.

\section*{Florida early voting check-in times}

Election administration in Florida---from registering voters, to
determining precinct sizes, to staffing and locating polling places,
and to setting early voting days and hours---is largely controlled at
the county level within a framework established by the Florida state
legislature. Our early voting check-in times are thus gleaned from
Florida counties.\footnote{We made public records requests to
  individual Florida county Supervisors of Elections, and the data we
  use in this paper is based on these requests.  Most counties that we
  contacted were not able to provide us with early voter check-in
  times.}

% , and these times cover check-ins
% across a variety of early voting stations during the 2012 and 2016
% General Elections.  

A Florida voter who wishes to cast his or her ballot prior to Election Day may vote at
any early voting polling location in the county in which he or she is
registered. This is distinct from in-person, Election Day voting,
during which a registered Florida voter may vote only at his or her
assigned precinct.  This flexibility necessitates that a county track
its early voters.  Rather than relying on traditional, paper-based
pollbooks to check in these individuals, as is the case for Election
Day in many Florida counties, the 67 county Supervisors of Elections
in Florida use electronic pollbooks to check in voters during the
state's early voting period.  These pollbooks are for the most part
known as Electronic Voter iDentification machines, or EViDs for short.
Although there is variance across Florida counties in electronic
pollbook implementation, for simplicity we refer to all electronic
pollbooks as EViD machines.\footnote{For example, Sarasota County uses
  a different electronic voting system for its early voters (source
  for this is a phone call with Cathy Fowler, office of the Sarasota
  Supervisor of Elections, on March 31, 2017).}

%% Cathy Fowler:  cfowler@sarasotavotes.com

EViD machines allow county pollworkers to check-in and verify the
registration statuses and identities of Floridians casting early,
in-person ballots. The EViD system is designed to reduce the time it
takes to process early voters, and it fosters synchronization across a
county's early voting centers as well as with the Florida statewide voter database.  For our purposes, EViD
machines record the check-in times for all early voters, and this
provides us with timestamps that specify when a voter began his or her
voting process. EViD timestamps are recorded to the minute, and using
these timestamps we can, for example, identify early voters who had
not checked-in when polls closed at 7:00pm on a given day of early
voting but nonetheless cast ballots.  Per Florida state law, any
elector in line at 7:00pm is allowed to cast a ballot.\footnote{See
  ``The 2017 Florida Statutes,'' Title IX ELECTORS AND ELECTIONS,
  Chapter 100, Section 100.011, available at
  \url{http://www.leg.state.fl.us/Statutes/index.cfm?App\_mode=Display\_Statute\&Search\_String=\&URL=0100-0199/0100/Sections/0100.011.html}
  (last accessed December 15, 2017).}

Through public records requests, we obtained EViD check-in times from
six Florida counties, Alachua, Broward, Hillsborough, Miami-Dade,
Orange, and Palm Beach. Table \ref{tab:evidcounts} lists the numbers
of EViD timestamps per county.  In the 2012 General Election, there
were 2,409,097 total early ballots cast in Florida, and in 2016,
3,876,753 early ballots.\footnote{2012 General Election turnout
  statistics are located at
  \url{http://dos.myflorida.com/media/693340/2012ballotscast.pdf}
  (last accessed December 10, 2017), and comparable 2016 General
  Election statistics at
  \url{http://dos.myflorida.com/media/697842/2016-ge-summaries-ballots-by-type-activity.pdf}
  (last accessed December 10, 2017).}  Our EViD data thus cover
approximately 39 percent and 44 percent of all early in-person ballots
cast in Florida in the 2012 and 2016 General Elections,
respectively.\footnote{The numbers in Table \ref{tab:evidcounts} do
  not include 115 EViD check-in times (28 from 2012 and 87 from 2016)
  that are listed as midnight.  We are suspicious that these times are
  not accurate and hence drop them.  Of the thousands of EViD
  check-ins we analyze here, 115 cases is
  negligible.\label{fn:midnight}}

% > 942166 / 2409097 * 100
% [1] 39.10868
% > 

% > 1687217 / 3876753 * 100
% [1] 43.5214
% > 

\input{county_evid_counts.tex}

EViD timestamps, one per early voter, are generated when a voter
checks-in at a voting location.  These timestamps are not subject to
self-reporting biases and hence are presumably more accurate than
surveys that inquire as to when voters cast their ballots.  Moreover,
EViD timestamps as associated with official Florida voter
identification numbers, which we can link to the nearly 14 million
individual voter registration records in Florida's statewide voter
file.  To this end, we merge our EViD timestamps with statewide voter
files that cover the 2012 and 2016 General Elections.\footnote{Per
  Table \ref{tab:evidcounts}, we recorded 942,166 early voters across
  our six counties in 2012 and 1,687,217 early voters in 2016.  Of
  those 2012 early voters, 40,455 could not be matched to the
  registration records we have from the January 2014 statewide Florida
  voter extract file.  Of the 2016 early voters, 969 could not be
  matched to the registration records we have from the January 2016
  statewide Florida voter extract file.  In any analysis that uses
  demographic or partisan information about the voters, we drop those
  voters who could not be matched to registration records.}

% > evid %>% filter (year == 2012) %>% count
% # A tibble: 1 x 1
%        n
%    <int>
% 1 942166
% > evid %>% filter (year == 2016) %>% count
% # A tibble: 1 x 1
%         n
%     <int>
% 1 1687217
% > 



In 2012, all 67 of Florida's counties offered between six and 12 daily
hours of early voting over an eight-day period (Saturday through
Saturday) that ended immediately prior to the General Election,
November 6, 2012.  In 2016, after long lines in 2012 had apparently
convinced the state legislature to grant counties more flexibility in
offering early voting opportunities, counties were permitted to offer
up to 12 hours of early voting per day, spread over 14 days, ending
the final Sunday before the November 8, 2016, General Election.
Counties were permitted to offer a maximum of 168 total hours of early
voting.\footnote{See Florida Governor Rick Scott's statement, issued
  in the aftermath of the 2012 General Election, available at
  \url{http://www.flgov.com/2013/01/17/governor-rick-scott-statement-on-election-reforms}
  (last accessed March 31, 2017).}

Figure \ref{fig:tenminhist} displays the distribution of EViD check-in
times on the last Saturday (November 3) of the 2012 General Election
early voting period in two different polling locations in Florida, one
at the Fred B.\ Karl County Center in Hillsborough County and the
other at the West Kendall Regional Library in Miami-Dade County.  The
figure's histograms describe in ten minute blocks the number of voters
who checked into these two polling stations.  The counts begin with
the first voter who checked-in just as polls opened at 7:00am, and
they end with the last voter who checked-in.

\begin{figure}[!ht]
  \caption{Early voting check-in times on Saturday, November 3, 2012, in two Florida locations}
  \label{fig:tenminhist}
  \centering
  \begin{subfigure}[b]{\linewidth}
    \centering\includegraphics[scale = 0.6]{example00.pdf}
    \caption{Fred B.\ Karl County Center, Hillsborough County}
    \label{fig:karlexample}
  \end{subfigure}%
  \\
  \begin{subfigure}[b]{\linewidth}
    \centering\includegraphics[scale = 0.6]{example01.pdf}
    \caption{West Kendall Regional Library, Miami-Dade County}
    \label{fig:kendallexample}
  \end{subfigure}
\end{figure}

Several features of Figure \ref{fig:tenminhist} are notable.  First,
both panels in the figure contain vertical black lines at 7:00pm.
This is the time beyond which additional voters were not allowed to
join a (possibly existing) voting line.  In the Hillsborough County
location (Figure \ref{fig:karlexample}), no check-ins occurred after
7:00pm, and from this it follows that, at 7:00pm, where was no voting
line.  In contrast, the Miami-Dade location (Figure
\ref{fig:kendallexample}) had many post-7:00pm check-ins, the last one
of which occurred around 1:00am on Sunday, November 4.  We thus know
that the last-voting voter at West Kendall Regional Library waited at
least almost six hours to vote.

Figure \ref{fig:tenminhist} incorporates via bar shading information
on voter self-reported race/ethnicity.  Early voters at the Fred B.\
Karl County Center were primarily white, although there were periods
on November 3 when the fraction of non-white voters was
disproportionately high, e.g., toward the end of the day.  At the West
Kendall Regional Library, though, the vast majority of early voters
were non-white.  The fraction of non-white voters appears plausibly
consistent across November 3.

Lastly, and very roughly speaking, we observe a flatter, more-uniform
distribution of check-in times at Karl County Center than in West
Kendall Library.  We draw on this fact later when we try to determine
which early voters in our six counties waited in line, and here we
provide some intuition.  As illustrated in Figure
\ref{fig:tenminhist}, it appears that early voting locations in
Florida that shut down very late had flatter distributions of check-in
times.  We suspect that this is evidence of persistent lines.  In
contrast, the non-uniformity of check-in times that we observe in
Figure \ref{fig:karlexample} is consistent with more of an ebb and
flow of early voters.  We know that there was not a line to vote at
Karl County Center at 7:00pm on Saturday, November 3, this despite the
fact that, per check-in times, the Center was regularly processing
voters.

Figure \ref{fig:tenminhist} highlights the value of EViD check-in
times as well as a notable limitation: check-in times are not linked
to arrival times.  As we have already noted, all data sources and
research designs have advantages and disadvantages, and in the results
that follow we draw on the former and attempt to work with the latter.

\section*{Results}

We present results in two sections.  First, we describe patterns in
check-in times across the 2012 and 2016 General Elections with
particular attention to race/ethnicity and partisanship.  Second we
consider the downstream consequences of extensive early voting wait
times.

% based
% on a identifying assumption we use to characterize individuals who
% waited in line before voting.

\subsection*{Who waits?}

Our analysis of who waits to vote turns in large part on check-in
times that occurred \emph{after} 7:00pm on a day of early voting.  Any
voter with such a late check-in must have waited to vote.

\subsubsection*{Early voting in 2012}

Across our six Florida counties of interest, there were 78 early
voting stations in 2012.  These stations serviced a total of 942,166
voters in the eight day long, 2012 early voting period.\footnote{This
  ignores the small number of EViD check-in times that look suspicious
  to us; see fn.\ \ref{fn:midnight}.  In addition, early voting
  locations in 2012 where two or fewer individuals recorded votes were
  dropped from our analysis.  The EViD data from Orange County and
  Palm Beach County in 2012 and 2016, from Miami-Dade County in 2016,
  and from Broward County in 2016 were not accompanied with voting
  locations for each early voter.  To determine the location at which
  these voters case their ballots, we match their voterids to a
  masterfile in Florida that records the location of every
  vote.}  % XXX fix masterfile

%> unique(evid12$county)
%[1] "ALA" "BRO" "HIL" "DAD" "ORA" "PAL"

%> length(unique(evid12$location))
%[1] 78

\begin{figure}[!ht]
  \caption{Number of locations where early votes were cast, 2012 General Election}
  \label{fig:nrlocs2012}
  \centering
    \centering\includegraphics[scale = 0.8]{number_of_Locations.pdf}
\end{figure}

Figure \ref{fig:nrlocs2012} describes by day of early voting, and by
hourly window, the number of locations across our six counties that
actively served voters.  All locations served early voters virtually
the entire day, and this is evident in the flat line, for the most
part pegged a bit under 80 prior to 7:00pm, at the top of the figure.
The sole exception to this rule occurred at the earliest time of the
day, during which a few early voting stations did not have any active
voters.

After 7:00pm (note the vertical black line at this time), Figure
\ref{fig:nrlocs2012} shows that check-in uniformity across early
voting stations quickly changed.  On initial days within Florida's
eight-day long early voting period in 2012, there was a steep drop in
active stations starting at 7:00pm.  Even with this drop, however,
there were still at least five open locations at 9:00pm on every day
of early voting.  Then, on the last two days of early voting, November
2 and 3, many early voting stations remained open well beyond 7:00pm.
On the final Saturday of early voting, over 30 stations were still
open at 9:00pm, and a few processed voters through midnight.

\begin{figure}[!ht]
\caption{Distribution of check-in times among early voters by hour, 2012 General Election}
  \label{fig:hist2012}
  \centering
    \centering\includegraphics[scale = 0.8]{histogram_by_hour.pdf}
\end{figure}

Figure \ref{fig:hist2012} displays the total number of voters who
voted early by the hour of the day at which they checked-in. Prior to
7:00pm, EViD check-ins were distributed fairly uniformly across the
day with a small peak before noon.  From 7:00am to 7:00pm, our
aggregated six counties consistently served approximately 70,000
Florida early voters per hour. Then, after 7:00pm, at which point new
voters could not join existing voting queues, the number of voters
served per hour dropped drastically---but not entirely.  A total of
83,250 (8.8\%) early voters in 2012 across our six counties cast
ballots after 7:00pm. These represent individuals who were in line at
the time the polls closed and were allowed to continue to wait to
vote.  We know that all of these voters had to wait in this way
although we do not know precisely how long each individual waited.  Of
this group, 36,120 voted after 8:00pm.  These voters waited at least
one hour to vote, and 13,567 voters, who checked-in after 9:00pm, must
have waited at least two hours to vote.

% > evid12 %>% filter(time2 > 19) %>% count()
% # A tibble: 1 × 1
%       n
%   <int>
%   83250

% > evid12 %>% filter(time2 > 20) %>% count()
% # A tibble: 1 × 1
%       n
%   <int>
%   36120

% > evid12 %>% filter(time2 > 21) %>% count()
% # A tibble: 1 × 1
%       n
%   <int>
%   13567

\begin{figure}[!ht]
\caption{Racial/Ethnic composition of early voters by hour, 2012 General Election}
  \label{fig:race2012}
  \centering
    \centering\includegraphics[scale = 0.8]{racial_composition.pdf}
\end{figure}

Aggregating across locations, Figure \ref{fig:race2012} describes the
composition of the 2012 early voting pool by race/ethnicity and by
hour of check-in.  For most of the day, whites were the majority
racial group, followed by blacks, Hispanics, and Asians.  This ranking
is similar to Florida's registered voter pool.  Among the 12,580,602
registered voters in the December 2012 Florida voter file,
approximately 66.4 percent are white, 13.9 percent Hispanic, 13.6
percent black, and 1.63 percent Asian.  As in years prior, blacks in
Florida in 2012 were disproportionate users of the state's early
voting period \citep{herronsmith:souls}.

% mysql> select count(*) from fLvoterfile_dec_2012_extract;
% +----------+
% | count(*) |
% +----------+
% | 12580602 |
% +----------+
% 1 row in set (4.20 sec)

% mysql> select race, count(*) AS num, round(100 * count(*) / (select count(*) from fLvoterfile_dec_2012_extract),2) AS percent from fLvoterfile_dec_2012_extract group by race;
% +------+---------+---------+
% | race | num     | percent |
% +------+---------+---------+
% |    0 |     220 |    0.00 |
% |    1 |   42165 |    0.34 |
% |    2 |  204542 |    1.63 |
% |    3 | 1711107 |   13.60 |
% |    4 | 1748251 |   13.90 |
% |    5 | 8358730 |   66.44 |
% |    6 |  203878 |    1.62 |
% |    7 |   29860 |    0.24 |
% |    8 |       4 |    0.00 |
% |    9 |  281839 |    2.24 |
% |   10 |       5 |    0.00 |
% |   11 |       1 |    0.00 |
% +------+---------+---------+
% 12 rows in set (5.72 sec)

% mysql>

% RACE CODES
% Race Code	Race Description
% 1	American Indian or Alaskan Native
% 2	Asian Or Pacific Islander
% 3	Black, Not Hispanic
% 4	Hispanic
% 5	White, Not Hispanic
% 6	Other
% 7	Multi-racial
% 9	Unknown

What is striking in Figure \ref{fig:race2012}, however, is the
racial/ethnic composition of the early voting pool immediately before
and then after 7:00pm.  Simply put, the pool becomes rapidly non-white
starting around 7:00pm; by the 8:00pm-9:00pm window, the pool is less
than 25 percent white.

The party registration of the early voting pool varies slightly with
time although not nearly as starkly as its racial/ethnic composition.
This is illustrated in Figure \ref{fig:party2012}, which describes by
hour the partisan breakdown of all early voters in our six counties as
well as the party breakdown of the four aforementioned racial/ethnic
groups.  While Democrats made up 54\% of all those who voted early in
our six Florida counties, we can see from the black line in the figure
that this percentage varied by hour of the day.  Notably, Democrats
composed a greater share of the voters in both the early hours of
voting and the hours after 7:00pm.  Hence, Democrats were
disproportionately affected by voting lines that forced voters to cast
their ballots after 7:00pm.

\begin{figure}[!ht]
\caption{Partisan composition of early voters by hour and by race/ethnicity, 2012
  General Election}
  \label{fig:party2012}
  \centering
    \centering\includegraphics[scale = 0.8]{partisan_composition_by_race.pdf}
\end{figure}

%  XXX Need to check 54% figure above.  
%table(evid12$party)/nrow(evid12)
%
%        DEM         IDP         NPA         OTH         REP 
%0.541646589 0.002073945 0.163840555 0.057074868 0.235364044 

Figure \ref{fig:party2012} describes the breakdown of party
registration by race/ethnicity, which reflects the aforementioned
racial/ethnic patterns of voting.  For example, black early voters in
Florida were almost entirely Democratic, regardless of when they
checked-in to vote.  Similarly, the party registration of Hispanic
voters---the least Democratic group in our six counties---remained
relatively constant throughout the day.  However, there is a slight
trend for white and Asian early voters, who became increasingly likely
to be registered Democratic as time progressed.

% > evid12 %>% summarize(pct = mean(party == "DEM"))
% # A tibble: 1 x 1
%        pct
%       <dbl>
% 1 0.5416466

\subsubsection*{Early voting in 2016}

By the time the 2016 General Election had occurred, Florida had
increased its number of early voting days from eight to 14, ostensibly
in an effort to reduce voting location congestion. Was this change
effective?  Figure \ref{fig:nrlocs2016} is analogous to the earlier
figure that described active early voting locations. Our six counties
had 104 early voting stations in place for the 2016 General Election.

% > length(unique(evid16$location))
% [1] 102

\begin{figure}[!ht]
  \caption{Number of locations where early votes were cast, 2016 General
    Election}
  \label{fig:nrlocs2016}
  \centering
    \centering\includegraphics[scale = 0.8]{number_of_locations_2016.pdf}
\end{figure}

The most important aspect of Figure \ref{fig:nrlocs2016} is the
pictured dropoff in voter check-ins that occurred after 7:00pm.  There
were indeed check-ins that took place after this time but not nearly
as many as in 2012.  On the busiest day in 2012, more than half of the
polling locations were open past 9:00pm and more than five were open
past midnight.  On the busiest day in 2016, only six locations were
open past 9:00pm and not one early voter cast a ballot after 10:00pm.

In the introduction, we commented on improvements in voter wait times
that survey-based research designs have found in 2016 compared to
2012; indeed, and Figure \ref{fig:nrlocs2016} is consistent with
findings in the 2012 \citep{spae2012} and 2016 \citep{spae2016}
editions of the SPAE.  See Table \ref{tab:floridaspae}.  Early voting
in 2016---at least in our six Florida counties of
interest---represents an improvement over 2012 in terms of reducing
voting station congestion, which is presumably tied not only to the
expanded number of sites but also to the number of days and hours of
allowable early voting.\footnote{We use the verb ``appears'' here
  because of voter-level sorting that may affect the types of
  individuals in Florida who cast their ballots prior to a given
  Election Day.  That is, when the Florida state legislature changed
  the state's early voting period between 2012 and 2016, voters may
  have altered their most preferred voting times.}

% latex table generated in R 3.4.2 by xtable 1.8-2 package
% Sun Dec 10 10:42:55 2017
\begin{table}[ht]
\centering
\caption{Wait times of Florida respondents in the SPAE, 2012 and 2016
  General Elections, broken down by percentages} 
\label{tab:floridaspae}
\begin{tabular}{lrr}
  \hline
 & 2012 & 2016 \\ 
  \hline
  Not at all &  23 &  49 \\ 
  Fewer than 10 minutes &  21 &  35 \\ 
  10 - 30 minutes &  35 &  20 \\ 
  31 minutes - 1 hour &  24 &   4 \\ 
  More than 1 hour &  28 &   1 \\ 
  I don't know &   1 &   1 \\ 
   \hline
   % \multicolumn{3}{l}{\emph{Note: ignores individuals with missing responses to the SPAE wait time question, which was}}\\
   % \multicolumn{3}{l}{\emph{"Approximately, how long did you have to wait in line to vote?"}}
\end{tabular}
\begin{flushleft}
  \emph{Note: ignores individuals with missing responses to the SPAE
    wait time question, "Approximately, how long did you have to wait
    in line to vote?"}
\end{flushleft}
\end{table}

% \input{spae_florida.tex}

\subsubsection*{Late voting in 2012 versus 2016}

Figure \ref{fig:race2012and2016} provides a race/ethnicity-based
perspective on the 2012 versus 2016 comparison for our six counties.
For three key racial/ethnic groups in Florida---black, Hispanic, and
white---the figure reveals the total number of check-ins by time,
aggregated across all days in the 2012 and 2016 early voting periods.

Per Figure \ref{fig:race2012and2016}, not only did more voters
check-in for early voting in 2016 compared to 2012, fewer voters voted
past 7:00pm.  This is the case for black, Hispanic, and white voters.
We already have seen that non-white early voters in Florida had
disproportionately late check-ins in 2012, but Figure
\ref{fig:race2012and2016} shows how this problem did not appear in
2016.  There were slightly more non-whites with late check-ins in this
latter year, but the magnitude of the white versus non-white gap
shrunk in 2016 compared to 2012.

\begin{figure}[!ht]
  \caption{Distribution of voter check-ins, 2012 General Election
    versus 2016 General Election, by time and race/ethnicity}
  \label{fig:race2012and2016}
  \centering
  \centering\includegraphics[scale = 0.8]{histogram_by_hour_by_race_2012_2016.pdf}
\end{figure}

Even though 2016 attracted far more early voters than 2012, the
changes made to the early voting period appear to have reduced the
congestion outside of the polls. The increased number of days and
hours may have helped, as these six counties still served roughly the
same number of early voters per day in 2016 (approximately 120,516) as
they did in 2012 (approximately 117,771).\footnote{It is an open
  question in the literature on election administration as to the
  marginal effect of additional days versus hours per day on early
  voting usage.  See \citet{walkeretal:ncearly} for a study of these
  issues in North Carolina in 2016.}

%  Given reduced prevalence of
% lines in 2016, it is likely that these and other reforms made the
% difference for congestion.

%> 942166/8
%[1] 117770.8
%> 1687217/14
%[1] 120515.5

\subsection*{Effects of waiting to vote}

We now consider the downstream effects on future electoral political
participation of waiting in line to vote.  We focus first on the
consequences of waiting in an early voting line in the 2012 General
Election on turnout in the 2016 General Election; this is our most
important analysis insofar as it considers presidential elections.  We
then consider how waiting in line to vote early in 2012 affected early
voting in 2016.  And finally we consider how waiting in 2012 was
associated with turnout in 2014, the year of a midterm election.

\subsubsection*{Effects of 2012 lines on 2016 turnout}

Our individual-level EViD files include Florida voter identification
numbers, which we link to statewide Florida voter files.  These latter
files specify the elections in which registered Florida voters
participated and, if so, whether they voted absentee, early in-person,
or on Election Day.  Thus, for any early voter in 2012 we can
determine whether the individual voted in the 2016 General Election,
assuming that this individual still lived in Florida as of November
2016 and was registered in the state.\footnote{Registered voters
  moving within Florida maintain their voter identification numbers.
  Our estimates in this section are thus not confounded by 2012 early
  voters moving across county lines within Florida.}

The challenges associated with assessing the effects of waiting to
vote in 2012 are twofold.  First, while we know voter check-in times
from our EViD data, the same cannot be said of voter arrival times,
which are not collected by election officials or anyone else for that
matter.  Indeed, were arrival times regularly tracked by election
administrators across the United States, there would be no need for
the research design employed here, much less national surveys asking
about wait times.  Second, early voters who voted at, say, 9:00am on a
given day of early voting may be systematically different from those
who voted at 6:00pm. Consequently, estimating the effect of voting
after closing time---when we know voters must have waited in line---is
potentially confounded by voter-level selection.  Hence, we need to
ensure that we control for differences across early voters as best as
we can so that selection into time of early voting does not confound
our estimates of the effect of waiting in 2012 on future turnout.

% Hence,
% distinguishing those early voters in 2012 who waited in line from
% those who did not is a challenge and requires an alternative method
% along with some assumptions.  

Our approach is as follows.  We condition on county, race/ethnicity,
party registration, gender, age, neighborhood income, and previous
voting history.  And, we make the following assumption: individuals
who voted at Florida early voting locations which stayed open well
past the 7:00pm cutoff time are more likely to have waited in line
than those who voted in locations that closed before 7:00pm or at this
time exactly.  Moreover, we assume that those who voted just before
the 7:00pm cutoff in early voting locations that stayed open late are
more likely to have waited in line than those who voted earlier in the
day.  We cannot know for sure if these assumptions hold, but there are
reasons to believe that they do.

Take, for example, the two polling locations in Figure
\ref{fig:nrlocs2012}.  The first location---the Karl County Center in
Hillsborough County---closed on time on November 3, 2012.  Therefore,
we assume that the Karl Center was not a congested polling location
and that the voters who voted just before 7:00pm did not have to wait
in much of a line.  If they did have to wait in line to vote, then the
line remarkably stopped just in time for the final early voter to
check-in right before 7:00pm.  While this is technically possible, it
would be remarkably coincidental.

On the other hand, the second location in Figure
\ref{fig:nrlocs2012}---the West Kendall Library in Miami-Dade
County---remained open until around 1:00am on November 3.  We know
with certainty that this last voter waited in line to vote.  In fact,
because all voters had to arrive before 7:00pm in order to check-in,
we know with certainty that he or she waited at least six hours to
vote.  Moreover, we can be fairly confident that the voters who voted
just before 7:00pm here also waited in line.  Voting that continues
past 7:00pm is indicative of a polling place that was operating
essentially at capacity at 7:00pm.  Hence, those voters who voted just
before 7:00pm are similarly likely to have been caught up in a long
voting line.

With the above assumptions as background, our strategy for
distinguishing voters who waited in line from voters who did not wait
in line is first to identify those who voted in an an early voting
polling place that closed well past 7:00pm from those who voted in a
polling place that closed on time.  In other words, we compare voters
who cast their ballots in places like the Karl County Center to voters
who voted in places like the West Kendall Library.  We want to compare
those individuals who voted closest to the 7:00pm cutoff since these
are the individuals who we are most confident were affected by
potential voting location congestion.
 
Therefore, we create a variable, called \emph{Over}, that identifies
all early voters who voted at a polling location on a day where the
last voter checked-in past 7:30pm.  We use this latter cutoff as
opposed to 7:00pm so that our results are conservative.  Voters who
cast ballots after 7:30pm did so at locations where we know that the
voting line at the end of the day was at least 30 minutes long.
Conditioning on county, race/ethnicity, partisanship, gender, age, log
median household income at the zip-code level, and previous voting
record (whether the voter voted in the 2008 General Election), we use
a logistic regression to estimate the effect of voting at a polling
location that is congested in 2012---one that goes ``over''
7:30pm---on the probability of voting in 2016:
\begin{equation*}
  \begin{aligned}
    \textup{Pr}\left(\mathrm{Voted16}_{i} = yes\right) =
    \textup{logit}^{-1}&(\beta + \eta_{County} + \alpha_{Over} + \gamma_{Hour} +
    \sigma_{Over \times Hour} + \upsilon_{Gender}  + \\& \rho_{Race/ethnicity} +
      \tau_{Age Group} + \psi_{Party} + \theta_{Income} + \pi_{Voted\emph{08}})
  \end{aligned}  
\end{equation*}
%
In the above model, $i$ denotes our collection of 2012 early voters
who appear in the history of Florida voting in 2008; $County$
indicates a voter's county; $Over$ indicates if the voter had to wait
in line in 2012; $Hour$ indicates the hour of the day in 2012 at which
a voter checked-in, between 7:00am and 7:00pm; $Gender$ indicates
whether the voter identifies as male or female; $Race/Ethnicity$
indicates if the voter self-identifies as white, black, Hispanic, or
Asian; $Age Group$ classifies voters according to their 2016 age and
bins them into 10-year age groups (20-29, 30-39, 40-49, 50-59, 60-69,
and 70+); $Party$ indicates party registration, either Democratic,
Republican, No party affiliation, or none of the above; $Income$
indicates log median household income in a voter's zip
code;\footnote{U.S.\ Census Bureau, 2010-2014 American Community
  Survey 5-Year Estimates} and, $Voted08$ and $Voted16$ are indicators
for participation in the 2008 and 2016 General Elections.

The covariates in our regression are based on findings in the
literature on turnout in American elections.  This literature is
extensive, and among other things it considers how age
\citep{strateetal:age,hightonwolfinger:lifecycle}, gender
\citep{schlozman:genderdifferentvoice}, party
\citep{martinezgill:partisanturnout,grofmanetal:turnout}, and race
\citep{verbaetal:raceparticipation,fraga:raceturnout} are associated
with individual-level turnout decisions.  We estimate our logistic
regression using 758,266 individuals, and coefficient estimates appear
in Table \ref{tab:reg} in the appendix.\footnote{Our regression is
  estimated only using individuals who voted before 7:00pm.  In
  addition, we restrict attention to individuals who are registered
  either Democrat, Republican, or Independent, or have recorded no
  party affiliation.  Finally, we drop registered voters who have no
  recorded gender, race, or birthdate.}

Per the leftmost column (1) of Table \ref{tab:reg}, having voted in
2008 in Florida is positively associated with subsequent voting in
2016, and being older is associated with greater turnout,
\emph{ceteris paribus}.  The latter result is consistent with findings
on turnout and age \citep{costaetal:walkingthewalk}.  Our age
estimates in Table \ref{tab:reg} are presumably underestimated; the
regression model that generates them excludes the youngest voters in
2016, who could not legally register to vote in Florida in 2008, and
younger registered voters are, \emph{ceteris paribus}, less likely to
turn out to vote \citep{shinosmith:registrationtiming}.  Table
\ref{tab:reg} also shows that, \emph{ceteris paribus}, registered men
vote less frequently than women and that registered whites vote often
than non-whites.  This gender finding is consistent with
\citet{leighleynagler:whovotesnow} although the race/ethnicity result
is not.

Our objective is not to explore the relationship between demographic
covariates and voter turnout in Florida in 2016.  Rather, we want to
know the marginal effect of waiting in line on future turnout.  To
this end, Figure \ref{fig:prvoting2016} displays graphically the
probability of voting in 2016 of a black male between the ages of
50-60 who is a registered Democrat, voted in 2008, and lives in a
zip-code with the average median household income.

Figure \ref{fig:prvoting2016} estimates are conditioned on each hour
of the early voting day in 2012. Its black points represent estimated
probabilities of voting in 2016 conditional on voting at a polling
location that closed \emph{before} 7:30pm; based on our discussion
above, these estimates reflect individuals who likely did not wait in
line.  In contrast, the grey points in Figure \ref{fig:prvoting2016}
are the estimated probabilities of voting in 2016 conditional on
casting an early ballot at a polling location that closed \emph{after}
7:30pm.  These estimates reflect individuals who likely did wait in
line to vote.  Hence, the difference between the figure's two sets of
estimates should reflect the effect of waiting in line.  Moreover, the
effect should be most isolated as one compares voters casting ballots
closer to 7:00pm.

\begin{figure}[!ht]
\caption{Probability of voting in 2016, given 2012 check-in time}
  \label{fig:prvoting2016}
  \centering
    \centering\includegraphics[scale = 0.8]{probability_of_voting_in_2016_over_under.pdf}
\end{figure}

One can see that, beginning at around 2:00pm, there is a small albeit
negative difference between individuals who voted in congested polling
places compared to those who cast their ballots in non-congested
places.  This suggests that voting lines may have started in the early
afternoon, and the reason there is no difference between congested and
non-congested polling places earlier in the day may be because long
lines had yet to form then. As congested polling places become
congested in the afternoon, we begin to see a small but statistically
significant effect on future political participation, an effect which
suggests that voting lines slightly dampen future electoral
participation.  The effect is small, amounting to no more than roughly
a percentage point difference.

Voting probability plots for other demographic groups are similar to
the plot in Figure \ref{fig:prvoting2016}, and rather than showing
many such plots we generalize to our complete sample the marginal
effect of waiting in line in 2012 on voting in 2016.  Across our
sample, Figure \ref{fig:margfx2016} displays estimates, along with
95\% confidence intervals, of the change in probability of voting in
2016 based on time of early vote in 2012.  We generated the results in
the Figure \ref{fig:margfx2016} with a simulation.  Our simulation
(500 repetitions) drew from a multivariate normal distribution where
the mean vector consists of values of the coefficient vector
corresponding to the leftmost column (1) in Table \ref{tab:reg}.  The
covariance matrix for the multivariate normal is the estimated
covariance matrix from our logistic regression.  The numbers in Figure
\ref{fig:margfx2016} can be thought of as average treatment effects
where the treatment on individuals in our sample is being forced to
wait in a voting line in 2012.

%\input{table_out_ate.tex}

\begin{figure}[!ht]
\caption{The effect of waiting in line to vote in 2012 on voting in 2016}
  \label{fig:margfx2016}
  \centering
    \centering\includegraphics[scale = 0.8]{margfx_2016_vote.pdf}
\end{figure}


The point estimates in Figure \ref{fig:margfx2016} are all negative,
indicating that waiting in line in 2012 is associated with decreased
turnout probabilities in 2016.  This is a key substantive result.  The
magnitude of the estimates range from very small (e.g., a 0.03
percentage point difference in turnout probabilities associated with
waiting to vote at 7:00am) to slightly over one percentage point later
in the day.

Overall, our findings on the effects of lines---they depress future
turnout, albeit only barely---are similar to the conclusions in
\citet{pettigrew:longlinesminorityprecincts} wherein each hour an
in-person voter spends waiting in line to vote is associated with a
one percentage point drop in the likelihood of future turnout.  To the
best of our knowledge,
\citeauthor{pettigrew:longlinesminorityprecincts} is the first scholar
to try to estimate the effect of waiting in line to vote on a later
propensity to vote, and our results are qualitatively similar to his.

Our regression results are based on individuals who were registered to
vote in Florida in 2008, 2012, and 2016.  This means that our analysis
excludes the youngest registrants in Florida, who were not legally
permitted to be registered in Florida as of the 2008 General Election;
individuals who moved to Florida from out of state between the years
of 2008 and 2016; registered Floridians who died, were convicted of
committing a felony, or adjudicated mentally incompetent between 2008
and 2016; and, individuals who moved out of the state between 2008 and
2016.

In light of these issues, a potential concern regarding our estimates
of the effect of waiting in line in 2012 on turnout in 2016 is that,
by selecting against younger voters, we are biasing our estimates of
these effects. To this end, there is a literature on waiting in line
in hospitals, and studies of the willingness of individuals to leave
emergency rooms without being seen by medical professionals find that
being young is a risk factor for this behavior
\citep{sunetal:lwbs,clareycooke:emergencyroomleave,shaikh:howlongwaiter}.
In other words, young individuals seems unduly sensitive to wait
times.  If this is true, by excluding relatively younger voters
in our regression model, our estimates on the effect of
waiting on future election turnout are conservative.

Our regression analysis also selects against recent movers into or out
of Florida and individuals who became felons, and thus lost the right
to vote, between 2008 and 2016.  We do not know of any theoretical or
empirical reasons to think that these types of individuals are
systematically different in terms of sensitivity to voting lines.
However, conditioning our results on individuals who remained
registered to vote in Florida across a span of eight years presumably
means that we are conditioning our analysis on politically engaged
individuals.  This will again make our results on the effect of voting
lines on future turnout conservative.

% While there may be idiosyncrasies in the rates at which the above
% types of individuals vote---e.g., young voters tend to have
% disproportionately low turnout rates
% \citep{shinosmith:registrationtiming}---whether these types of
% individuals are unusually sensitive to lines is unclear.  


% %% XXX Below might be wrong...
% Consider our exclusion of the youngest voters, and as context note
% that younger registered voters are \emph{ceteris paribus} less likely
% to turn out \cite{shinosmith:registrationtiming} than older
% counterparts.  If younger voters are more impatient then older
% voters, we will underestimate the effect of long lines on future
% turnout, to the extent younger voters were caught in lines in 2008 or
% 2012.  With respect to the other categories of excluded voters, it is
% difficult to imagine that they are have any systematic biases
% although, in principle, sensitivity to lines could be correlated with
% the willingness to live in Florida and thus to be listed in multiple
% voter files.

\subsubsection*{Effects of 2012 lines on 2016 early voting}

Now restricting attention to 2016 voters only, we consider a logistic
regression that analyzes whether these individuals voted early in
2016---or, in contrast, voted in another fashion.  \mbox{Results} are
in the second (2) column of Table \ref{tab:reg}, and Figure
\ref{fig:atevotingearly2016} contains a plot of 2016 early voting
probabilities structured similarly as our earlier turnout probability
plot.

\begin{figure}[!ht]
  \caption{The effect of waiting in line to vote in 2012 on voting early in 2016}
  \centering\includegraphics[width = \linewidth]{margfx_2016_vote_early.pdf}
  \caption{Voting early in 2016}
  \label{fig:atevotingearly2016}
\end{figure}

Intuitively, and to the extent that time spent in line increases the
cost of voting, we would expect that waiting in an early voting line
in 2012 is associated with a decreased likelihood of voting early in
2016.  Indeed, Figure \ref{fig:atevotingearly2016} shows that the 2016
voters whom we identify as having waited in an early voting line in
2012 were less like to vote early in 2016, \emph{ceteris paribus}.
This result shows how some voters responded to high costs of early
voting by choosing an alternative way to vote in a later election, and
this highlights a downstream consequence of voting lines.  Still,
Figure \ref{fig:atevotingearly2016}'s percentage point gaps (black dot
to grey dot) are around one to two points, which is small albeit not
negligible.  Early voting was designed as a convenience, and Figure
\ref{fig:atevotingearly2016} suggests that the value of this
convenience can be lost in the shadow of a previous bad experience.

There is a U-shape pattern to the points in Figure
\ref{fig:atevotingearly2016}, and we can only speculate as to what may
be responsible for this.  We know from the number of post-7:00pm
closing times that, in 2012, early voting locations in our six
counties of interest were congested in the evening.  We suspect that
morning early voting hours also suffered from congestion albeit from a
different source, perhaps when individuals stopped by to vote, say,
before going to work.  If lines discourage future early voting, this
pattern would yield a U-shape as in Figure
\ref{fig:atevotingearly2016}.  It is also interesting to note that
early voters in 2012 who voted at locations that closed before 7:30pm
also have a U-shape in their early voting probabilities.  We suspect
this is because these locations probably suffered from morning and
evening lines, albeit not very long ones.\footnote{This discussion
  suggests that employed individuals may be disproportionately
  sensitive to waiting in line to vote, and this would be consistent
  with the relationship between affluence and tolerance for waiting in
  a supermarket line \citep[e.g.,][]{bennett:waitinginlinechars}.
  There is no employment data in Florida statewide voter files, but a
  relationship between work opportunities and one's willingness to
  wait in line is worth considering in future research.}

A full analysis of the determinants of what leads individuals to cast
votes early is beyond our scope.  However, what is most important
about Figure \ref{fig:atevotingearly2016} is that it highlights how
voting experiences can affect forms of future political participation.
Wanting to voting early, but not doing so on account of fear of lines,
imposes a cost on voters, and future research on election
administration needs to pay attention to the way that electoral
experiences, both positive and negative, can accumulate over time and
potentially affect future choices.

\subsubsection*{Effects of 2012 lines on 2014 turnout}

Our final set of results on the effects of lines considers turnout in
the 2014 General Election as a function of line waiting in 2012.  Our
consideration of 2014 reflects the possibility that the effect of
waiting in line attenuates in time (because bad memories of line
waiting become less salient, for example).  If this is true, we would
expect to see greater consequences of waiting in line in 2012 on 2014
turnout compared to 2016 turnout. 

Nonetheless, one drawback of analyzing 2014 turnout based on 2012
voting experiences is the fact that the 2014 statewide election in
Florida was a midterm election as opposed to a presidential election.
As midterms are generally less attractive to voters as the stakes are
ostensibly lower in the former, a negative voting experience in 2012
might be sufficient to discourage a voter from turning out to vote in
2014 when such an experience might not have been enough to discourage
that same voter from voting in the presidential election in 2016.
Hence, having to wait in line to vote in 2012 may cause a more
profound reduction in turnout in 2014 compared to 2016.  On the other
hand, midterm electorates may be more energized and politically
engaged than presidential electorates, the latter of which attract all
sorts of voters on account of the perceived high stakes.  As a result,
a negative experience waiting in line in 2012 might not be enough to
deter an already motivated midterm voter.  If this is the case, we
might expect to see smaller effects of 2012 lines on 2014 turnout
compared to 2016 turnout.

% Per the above, there are limitations inherent in our consideration of
% 2014 turnout as a function of 2012 line experiences.  In an
% observational study like ours, these limitations are unavoidable, and
% we note them in the interest of transparency.

%in particular
%if their Election Day polling place is relocated \cite{amos_etal2017}.


\begin{figure}[!ht]
  \caption{The effect of waiting in line to vote in 2012 on turnout in 2014}
  \centering\includegraphics[width = \linewidth]{margfx_2014_vote.pdf}
  %\caption{Turnout in 2014}
  \label{fig:prvoting2014}
\end{figure}

Figure \ref{fig:prvoting2014} presents our estimates of the effect of
having waited in line to vote in 2012 on the probability of turning
out to vote in the 2014 midterm election.  What is most noticeable
about this figure is that the downstream effect of having waited in
line in 2012 is much stronger in 2014 than it is in 2016.  Individuals
who voted in congested polling locations in 2012 were between 2.5 and
4.5 percentage points less likely to vote in 2014 compared to those
who voted in polling locations that were not so congested in 2012.  In
2016, those same individuals were between zero and 1.25 percentage
points less likely to vote.  This suggests that negative voting
experiences in 2012 were more profound in altering midterm turnout
than they were in altering turnout in a subsequent presidential
election.

In both Figure \ref{fig:prvoting2014} and the earlier Figure
\ref{fig:atevotingearly2016}, effects of waiting to vote on 2014
turnout and on 2016 turnout are conditional on the hour in which a
voter voted in 2012.  In both figures, early voters who cast their
ballots in the afternoon in 2012 were more affected by congested
polling locations than voters who cast their ballots before noon.
Although a so-called afternoon effect is more pronounced in 2016, we
do see that, in 2014, the magnitude of the effect of line waiting on
turnout increases between noon and 3:00pm and then gradually returns
to before-noon levels for voters who voted just before the 7:00pm
cutoff.  While it is unclear why the effect of lines is strongest
regarding 2014 turnout just before 3:00pm, it is clear that, in 2014,
there is a slight conditional effect of time from 2012 and that this
effect is outweighed by a much stronger effect of having voted in a
congested precinct.\footnote{An interesting question for future
  research is the extent to which bad experiences via voting lines
  compound over time or cancel themselves out.  We do not at present
  have data that would allow us to analyze the 2016 turnout decisions
  of, say, individuals who experienced a long line in 2012 but no line
  in 2014.}

\subsection*{A comment on our key assumption}

Having now presented results on the effect of voting lines on future
turnout, it is important to review the key assumption we have made in
our research design, namely that late-closing (after 7:30pm) early
voting precincts in Florida in 2012 were different, i.e., congested,
compared to non-late-closing precincts. We made this assumption
because EViD check-in times, the key to our research design, are not a
panacea in the quest to understand voting lines and their
consequences. Our check-in times are useful, but they do not specify
when voters arrived to cast their ballots. Using check-ins in a line
study such as ours thus requires assumptions, but of course a
dependence on assumptions does not make this study unique.

To wit, the existing literature on voting lines relies heavily on
voter self-reports of time spent in line.  Despite research on memory
distortion, this literature assumes that voters can accurately recall
and report in a recent election how long they waited prior to
voting. In contrast, nothing in the results reported on here requires
voters to have accurate recollections of any aspects of their voting
experiences.  We are not arguing that scholars must, or even should,
avoid voter self-reports of time spent in lines. Rather, we are
arguing that research with a diversity of assumptions will strengthen
the literature on election administration.

What if we are wrong that early voting locations in Florida that in
2012 closed after 7:30pm were not congested? If so, then we have
confused congested early voting locations with non-congested locations
and presumably vice versa. This will make it difficult for us to
estimate the effect of an hour of waiting on future turnout. Our
assumption about late-closing early voting locations is thus
conservative, and we note that we have found significant, albeit
small, effects on future turnout of voters having cast their ballots
in voting locations that we identified as closing late.

\section*{Conclusion}

The extent to which all voters in an election are treated equally
depends in part on whether the cost of voting in the election is
distributed uniformly across voters. Administrative aspects of
elections that affect the cost of voting are thus part of the overall
calculus of whether an election was characterized by equal treatment
or the lack thereof.  A notable administrative aspect of elections is
the presence of voting lines, and lengthy voting lines in recent
elections in the United States have been of concern to scholars,
election officials, policy makers, and voters.  Burdensome wait times
not only impose opportunity costs on those forced to wait in line, but
they can also discourage future electoral participation.

Research on voting lines has historically been challenged by the lack
of data on who waits to vote.  Other than noting the times at which
polls officially closed and recording information about which voters
cast their ballots after voting lines were capped, there is little
data collected by election officials that might distinguish those who
waited in line to vote from those who did not (not to mention
individuals who waited in a voting line for a limited period and then
departed a queue without casting a ballot).  Notwithstanding a limited
number of exceptions, local election officials do not maintain
official records on line evolution or how long individual voters
waited \citep{herron:confidence}.

% Previous research using election
% records to study lines have thus been limited to examining variables
% like aggregate precinct results
% \citep[e.g.,][]{herronsmith:closingtimes,
%   pettigrew:longlinesminorityprecincts}.  Data limitations have
% prevented these projects from identifying individuals who waited in
% line from those who did not \emph{within} the same precinct.

To overcome the lack of data that identifies who waits in line to
vote, scholars have relied on surveys that ask respondents to recall
their experiences at the polls and report the length of time that they
waited prior to voting \citep{stewart:waitingtovote2012,
  pettigrew:racegapwaittimes}.  These survey-based analyses have
illuminated important variation in wait times across geographies and
racial/ethnic groups, but associated with them is the caveat that
survey responses are subject to biases associated with self-reporting.

We contribute to the literature on voting lines by analyzing early
voter check-in times in Florida's 2012 and 2016 General Elections.
Using electronic voting check-in time stamps from Alachua, Broward,
Hillsborough, Miami-Dade, Orange, and Palm Beach counties, we identify
the day and time at which over two million voters checked-in to vote.
These check-in times not only provide insight into voting patterns
across time but, given allow us to characterize individual voters who
likely waited in line to vote.

Our analysis reveals that Florida's early voting period in 2012 had a
significant number of voters who voted past the official 7:00pm
closing time and, therefore, must have waited in line to vote.  These
voters were disproportionately black, Hispanic, and Democratic---a
finding that coincides with survey results on voting lines.  In 2016,
we find that congestion observed in 2012 had all but vanished.  Voters
rarely voted late into the night in 2016, and most early voting
locations in our Florida counties of interest closed promptly.  Even
though the counties we study in 2016 served as many individual per day
as they had in 2012, they appear to have been more prepared to handle
the high rates of early voting.
  
We also estimated the effect that waiting in line in 2012 had on future
electoral participation, i.e., on voting in 2014 and in 2016.  We
found a very slight negative effect of 2012 line waiting on turnout in
2016, an effect amounting to no more than a one percentage point
decrease in the propensity to vote. The result is similar to that
identified by \citet{pettigrew:racegapwaittimes}.  A one percentage
point effect is small in magnitude, which is normatively pleasing, but
any negative effect or lines on future turnout is normatively
troubling.  In 2014, a midterm election year, we find larger effects
of 2012 line waiting on turnout.  Thus, not only are some voters
penalized by waiting in line, but a small number of these individuals
appear to be dissuaded from voting in the future, thus lessening their
voices.

% We offered evidence as well that long early voting lines dissuade
% voters from casting early ballots.  To the extent that early voting is
% a form of convenience voting, this result shows how the value of this
% convenience can be compromised by long lines, something that most
% observers would classify as administrative failures.


% As
% such, our our analysis of check-in times should be viewed as a
% complement to studies using survey and other observational methods to
% explain who waits in line and for how long.


Overall, our evidence shows that the 2012 General Election in Florida
did not treat all voters equally with respect to time spent waiting in
line.  Minority voters in 2012 voters faced disproportionately long
lines---thus, disproportionately high costs of voting---and then, in
some cases, a small downstream effect in 2014 and 2016 of a decreased
propensity to vote.  Our results illustrate how administrative aspects
of elections, in particular, resource allocation decisions that affect
whether an election will have lines at the polls or not, should not only be understood
as administrative in nature, but also as indicative of the underlying fairness of the electoral process and 
the extent to which all voters are treated equally under the law.

% Except for the few states that require voters to cast ballots by mail
% (or use drop off boxes), waiting to vote at the polls is likely to
% continue to be an inevitable part of the voting experience for voters
% in certain jurisdictions. Long lines may result in some voters not
% only having to pay a time tax, but they may also serve as a deterrent
% on future voting for some citizens. In Florida, the expansion of the
% number of days, hours, and locations of early voting between the 2012
% and 2016 General Elections likley helped to alleviate some of the
% dealys experienced by those voters who chose to voter during the early
% voting period, especially those casting ballots late in the day.
% Further research needs to be conducted on how such delays at the polls
% may affect future turnout decisions.



%\section*{Appendix}
%\input{../plots/table_out.tex} 

\clearpage
\newpage

\bibliographystyle{apsr}
\bibliography{evid-cites}

\newpage
\appendix
\section*{Appendix}

\input{../plots/table_out.tex}

\end{document}


 
%% Florida: makes us conservative.  MCH cite this in conclusion.  XXX

%State Electoral Institutions and Voter Turnout In Presidential Elections, 1920–2000
%Melanie J. Springer, First Published May 30, 2012 Research Article  
%https://doi.org/10.1177/1532440012442909
