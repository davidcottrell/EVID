\documentclass[12pt,titlepage]{article}

% \usepackages21{fancyhdr}

\usepackage[printwatermark]{xwatermark}

\usepackage{grffile}
\usepackage{xcolor}
\usepackage{lipsum}
\usepackage{times}
\usepackage{soul}
\usepackage{epsfig}
\usepackage{rotating}
%\usepackage[hyphens]{url}
%\usepackage[breaklinks=true]{hyperref}
\usepackage{hyperref}
\usepackage{etoolbox}
\appto\UrlBreaks{\do\-}

\usepackage{latexsym}
\usepackage{graphicx}
\usepackage{amsfonts}
\usepackage{amsmath, amsthm, amssymb}
\usepackage{setspace}
\usepackage{natbib}
\usepackage{longtable}
\usepackage{keyval}
\usepackage{caption,subcaption}
\usepackage{arydshln}


\providecommand{\keywords}[1]{\textbf{\normalsize{Keywords: }} \normalsize{#1}}

%% set 1-inch margins
\usepackage{fullpage}

%% APSR submission: no commas in citations between name and year
%% See http://merkel.zoneo.net/Latex/natbib.php
\bibpunct{(}{)}{;}{author-year}{}{;}

% the opening bracket symbol, default = (
% the closing bracket symbol, default = )
% the punctuation between multiple citations, default = ;
% the letter `n' for numerical style, or `s' for numerical superscript style, any other letter for author-year, default = author-year;
% the punctuation that comes between the author names and the year
% the punctuation that comes between years or numbers when common author lists are suppressed (default = ,);

\usepackage{footmisc}
\renewcommand{\footnotelayout}{\doublespacing} % set spacing in footnotes
\newlength{\myfootnotesep}
\setlength{\myfootnotesep}{\baselineskip}
\addtolength{\myfootnotesep}{-\footnotesep}
\setlength{\footnotesep}{\myfootnotesep} % set spacing between footnotes

% make footnote font size same as regular font size in text
\renewcommand{\footnotesize}{\normalsize} 

%% make possessivecite macro since this does not exist in natbib
\newcommand{\possessivecite}[1]{\citeauthor{#1}'s (\citeyear{#1})}

%% Use this for a "DRAFT" watermark
% \newwatermark[allpages,color=pink!30,angle=45,scale=5,xpos=-25,ypos=40]{DRAFT}

%% List all locations for graphics here
\graphicspath{ {../Plots/} }
\begin{document}
\sloppy
\thispagestyle{empty}

%% APSR submission requires double-spaced footnotes
%%\newcommand{\footnote}[1]{\footnote{\doublespacing #1}} %% <-- note \doublespacing here.

\renewcommand{\topfraction}{.85}
\renewcommand{\bottomfraction}{.7}
\renewcommand{\textfraction}{.15}
\renewcommand{\floatpagefraction}{.66}
\renewcommand{\dbltopfraction}{.66}
\renewcommand{\dblfloatpagefraction}{.66}

% \urldef\myurlncsl1\url{foo%.com}
% \begin{document}
% text\footnote{WWW: \myurl}


\title{Voting lines, equal treatment,\\and early voting check-in times
  in Florida\thanks{Earlier versions of this manuscript were presented
    at the 2017 Annual Meeting of the Midwest Political Science
    Association, and Northwestern University, and at the Free
    University of Berlin.  The authors thank Evan Morgan for research
    assistance, Michael Hanmer for comments on an earlier draft, and
    staff members from various Florida Supervisor of Elections offices
    for data on early voting check-in times.}\author{David
    Cottrell\thanks{Department of Government, Dartmouth College.  6108
      Silsby Hall, Hanover, NH 03755
      (\texttt{david.cottrell@dartmouth.edu})} \and Michael C.\
    Herron\thanks{Professor of Government, Dartmouth College.  6108
      Silsby Hall, Hanover, NH 03755-3547
      (\texttt{michael.c.herron@dartmouth.edu}).} \and Daniel A.\
    Smith\thanks{Professor of Political Science, University of
      Florida, 234 Anderson Hall, Gainesville, FL 32611-7325
      (\texttt{dasmith@ufl.edu}).}}\vspace{1cm}\\\keywords{voting
    lines, equal treatment, election administration, turnout, early
    voting}}

%\title{Voting lines, equal treatment,\\and early voting check-in times
% in Florida}


\maketitle \doublespacing 



% \topskip0pt
% \vspace*{\fill}
% \begin{center}
%   \Large{\textbf{Word count: 8,790}}
% \end{center}
%  \vspace*{\fill}

\begin{abstract}
  \noindent 
  % The extent to which voters in a given election must wait in line
  % prior to casting their ballots \mbox{affects} the cost of voting.
  Lines at the polls can lead to unequal treatment of voters if some
  voters are compelled to wait longer than others, thus causing the
  cost of voting to vary systematically across an electorate. In
  addition, long lines may influence future electoral participation.
  We leverage voter check-in times from Florida---involving 942,166
  early in-person voters from the 2012 General Election and 1,687,217
  from 2016---and highlight disproportionately long wait times
  incurred by minority voters. We find, however, fewer problems in
  2016 compared to 2012. Florida early in-person voters who waited
  excessively in 2012 had a slightly lower probability---approximately
  one percent---of turning out to vote in 2016, \emph{ceteris
    paribus}.  Our results draw attention to the ongoing importance of
  the administrative features of elections that influence the cost of
  voting, and ultimately, whether voters in an election are treated
  equally.
\end{abstract}


% > evid %>% group_by(year) %>% count
% # A tibble: 2 x 2 # Groups: year [2] year n <int> <int> 1 2012
% 942166 2 2016 1687217

\newpage
\section*{Introduction}

Modern democracies are characterized by regular and free elections,
the legitimacy of which can suffer in the face of perceptions of
unfairness and disparate treatment of groups of voters
\citep{norris2014electoral}.  An election can fail to treat all voters
equally if, for example, there are meaningful differences in vote
tabulating technologies across the electorate
\citep[e.g.,][]{kimballkropf:tech}; if some groups of voters must
contend with registration hurdles that others do not
\cite[e.g.,][]{ansolhersh:registration}; if some voters face strict
photo identification requirements and others no such requirements
\citep[e.g.,][]{benteleetal:newjimcrow}; and, if some voters have to
travel much longer distances than others to cast their ballots
\citep[e.g.,][]{dyckgimpel:distance}.  More generally, when the
\emph{cost of voting} in a given election varies significantly by
voter group---where group membership could be based on race/ethnicity
or party affiliation, for example---the principle of equal treatment
of voters is at risk.

%   Elections affected by fraud certainly fall in this
% category, as \citet{cottrelletal:fraud2016} argue, and elections
% conducted in the midst of acrimonious debates over the nature of the
% franchise risk being thought of as unfair \citet{hasen:votingwars}.
% Fairness can also be in the eye of the beholder:
% \citet{sancesstewart:confidence} show how confidence in a given
% election outcome can be a function of whether an individual's
% preferred candidate won.

Here we consider how voting lines, an aspect of election
administration, are related to equal treatment.  The lines that
develop outside a polling station reflect administrative decisions
about resource allocation as well as the extent of voter turnout
\citep{herronsmith:hanoverstudy}.  If lines adversely affect some
voters more than others, then the cost of voting will vary across
voters.  Thus, if we seek to know whether a given election featured
equal treatment of voters, we must assess the extent to which the
voter groups who participated in it were subjected to long lines.

Voting lines are potential issues wherever voters cast their ballots
\emph{in-person}, as opposed to voting remotely via mail.  The vast
majority---over 75 percent in 2016 and approximately 66 percent in
2012---of voters in the United States who cast ballots in presidential
elections do so in-person, either on an Election Day itself or during
a corresponding early voting period whose duration depends on relevant
local laws.\footnote{For the composition of the 2012 and 2016
  electorates, see \citet{eac:2012} and \citet{eac:2016},
  respectively.}

%  In-person voting experiences have a number of
% features that are distinct from the experiences of those who vote by
% mail, and our interest here is the act of waiting in line.

% Casting an in-person ballot requires traveling to
% a voting precinct; possibly spending time in a line prior to voting;
% authenticating oneself in some fashion to election officials;
% physically registering candidate and ballot measure preferences; and,
% submitting a ballot for tabulation.  

%% 2016 EAC is from p. 10
%% really think we need to stick with polling station terminology, not precincts, as early voting is not precinct-based.

Our analysis considers two features of voting lines, by which we mean
lines that form in front of polling stations as opposed to lines in
which voters, already in the act of casting ballots, are forced to
stand. First, waiting in line to vote constitutes a time tax
\citep{mukherjee:timetax}. This tax can be negligible---for example, a
voter waits a scant number of seconds prior to initiating her voting
process---or imposing---some of the Florida voters we describe shortly
waited several hours to vote in 2012. Associated with a voting line
time tax are the distributional questions associated with all forms of
taxation: is the time tax fair and, in particular, is the burden of
this tax spread uniformly across voters or concentrated on selected
groups?

Second, waiting in line is an experience that may have ``downstream''
electoral effects, as characterized by
\citet{pettigrew:longlinesminorityprecincts}. When an individual is
compelled to pay a relatively high tax in order to participate in a
social or political activity, a natural response might be to avoid the
activity, or substitute for it, in the future.  Given the relatively
low rate of voter participation in American presidential
elections---approximately 58 percent in 2012 and 60 percent in
2016---any aspect of the voting experience that might have a
depressive effect on turnout should be considered a matter worthy of
study.\footnote{The national turnout rate for a general election
  depends on the way that the number of eligible voters in the country
  is calculated.  The figures cited here are from the United States
  Election Project, available at
  \url{http://www.electproject.org/home} (last accessed December 11,
  2017).\label{fn:uselectionproject}}

% To make matters
% concrete, if a line in front of a particular restaurant is long, some
% individuals may substitute an alternative eatery for the restaurant in
% question, one without an imposing line, or even forego
% eating.\footnote{Alternatively, a line in front of a restaurant might
%   be a quality signal, in which case the presence of a line might be
%   an incentive to wait in it.  The same could be said for voting, and
%   here we ignore the possibility that lines form due to potential
%   voters using the presence of lines as a means to gather information
%   on whether voting is valuable.}

%  While it is not
% \emph{a priori} clear that voting can be substituted for in the way
% that selecting a restaurant might be, we should not dismiss outright
% the possibility that the effect of waiting in line to vote, which as
% we have already argued raises the cost of voting, might decrease the
% likelihood of voting in the future.  


These two features of voting lines---time tax and potential downstream
participatory consequences---are conceptually distinct. The time tax
for voting could be uniformly spread across voters, which might be
normatively pleasing in the sense of not violating equal treatment of
voters, and yet there could be a significant effect of waiting in line
on future electoral participation. Or, the time tax could be
concentrated on certain voters group, which most would argue violate
equal treatment regardless of whether there are downstream
consequences of waiting to vote. With this in mind, our empirical
analysis of voting lines during early in-person voting in Florida is
divided into two sections. First, we assess which voters are more
likely to wait in line. Second, we consider whether individuals who
wait excessively to vote are less likely to vote in a subsequent
election.

% One of the difficulties in studying voting lines in the United States
% is data availability: it is not easy to determine precisely how long
% voters waited in line at the polls.  It is accordingly difficult to
% study if there are consequences for subsequent voting behavior of
% waiting in line.  Some researchers have dealt with this hurdle by
% using surveys---either Election Day exit polls or post-election
% questionnaires---which integrate responses to waiting time questions
% with other survey items, namely socioeconomic and partisan queries
% \citep{stewart:waitingtovote2012}.  Post-election surveys on voting
% experiences can foster access to wide swaths of voters, and this is
% valuable.  A limitation of surveys regarding electoral experiences,
% though, is that they rely on self-reports of wait times, which may not
% be accurate, and in addition it is known that self-reporting of
% turnout in general can suffer from over-reporting
% \citep{karpbrockington:overreport,bellietal:overreport}.

%In light of potential concern with survey-based evidence on voting
%lines,

In contrast to much of the literature on voting lines, which relies on
surveys, we turn here to a relatively untapped source of electoral
information, namely, voter check-in times collected by Florida
counties that recorded check-ins from early voting sites in the 2012
and 2016 General Elections.  The hundreds of thousands of check-in
times which form the basis of our results are not subject to potential
biases arising in self-reports of time spent in line.  Moreover, we
can associate a voter's check-in time with her race/ethnicity, party
registration, and electoral participation, pieces of information
publicly available in Florida's statewide voter file.  Our use of
early voting check-in times complements the literature's reliance on
survey results.
 

% Still, in the interests
% of being transparent about limitations in our research design, a
% voter's check-in time is not directly connected to the amount of time
% she spent waiting in line.  We describe below how we link voter
% check-in times with corresponding waiting times, but the connection is
% not perfect.

Regarding the time tax associated with waiting to vote, we find
disproportionate concentration of this tax on minority voters.  The
situation was worse in Florida in 2012 than in 2016, and in general
our collection of check-in times highlights fewer troubling issues in
the latter year.  This is consistent with national, survey-based
evidence on voting lines in the 2012 and 2016 General
Elections.\footnote{For a national 2012 versus 2016 comparison, see
  \url{http://electionupdates.caltech.edu/2016/12/14/this-just-in-lines-at-the-polls-shorter-in-2016-than-in-2012}
  (last accessed March 30, 2017), as well as
  \url{http://www.richmond.com/opinion/their-opinion/guest-columnists/fortier-and-palmer-column-who-waits-the-longest-to-vote/article_81efabad-aa23-577b-83d1-0e524cf844ef.html}
  (last accessed June 30, 2017).}  Whether the recent decrease in
waiting times reflects progress in Florida election administration,
substitution effects and voter sorting, or idiosyncrasies from either
the 2012 or 2016 General Election is not clear.  Two election-years of
data are unlikely to be sufficient to gauge broad progress in a state
as large and heterogeneous as Florida.

Regarding the effect of waiting to vote on future electoral
participation, conditional on our strategy for identifying those
Florida early voters who suffered long waits in 2012, we find small
yet negative consequences regarding turnout in 2016. The consequences
are not negligible insofar as long lines do appear to be associated
with lower participation levels; they are, however, small, around one
percent. This is consistent with
\citet{pettigrew:longlinesminorityprecincts}.  We show as well that
early voters in 2012 were less likely to vote early in 2016,
\emph{ceteris paribus}; in particular, early voters who voted early or
late in a day of early voting in 2012 were particularly less likely to
vote early in 2016.

% XXX 2014 as well?  Check.

The next section situates our study in the literature on voting lines
and election administration.  We then turn to early voting check-in
times in Florida and explain what these times mean and how they can be
interpreted.  Our results, which leverage check-in times, are divided
into two sections: first we describe the distribution of the voting
time tax across our set of Florida voters, and second we analyze the
downstream effects of the tax.  We conclude with observations about
congestion at the polls and equal treatment of voters.

\section*{Voting lines and equal treatment in American elections}

Voting is a fundamental right. As the United States Supreme Court
ruled in \emph{Reynolds v.\ Sims} (1964), ``[T]he right of suffrage
can be denied by a debasement or dilution of the weight of a citizen's
vote just as effectively as by wholly prohibiting the free exercise of
the franchise.''\footnote{See
  \url{https://supreme.justia.com/cases/federal/us/377/533/case.html}
  (last accessed December 12, 2017) for the text of the decision.}  In
assessing the costs that may debase or dilute voting, the high court
has considered the fairness of an assortment of rules adopted by
states, including rules that relate to redistricting, literacy tests,
proof of citizenship, voter identification laws, and the purging of
registered voters. In various cases, the Court has been asked to
balance the interests of the state with the equal treatment of
voters. Although there is ample evidence that long voting lines in
Florida in the 2012 General Election disproportionately affected
racial and ethnic minorities \citep{herron_smith2014}, as of yet the
Supreme Court has not been asked to rule on whether Section 2 of the
Voting Rights Act applies to excessively long wait times at the polls,
constituting ``a denial or abridgement of the right of any citizen of
the United States to vote on account of race or color'' (52 U.S.C.\ \S
10301(a)).

Long voting lines have received considerable attention in the last
several election cycles in the United States.  ``To me,'' Cynthia
Perez complained in March 2016 after seeing a long line wending its
way around an early voting center in Maricopa County, Arizona, with
those in line reporting standing in line for more than three hours
before checking in to vote, ``this is not what democracy is
about.''\footnote{This quote is drawn from ``Angry Arizona Voters
  Demand: Why Such Long Lines at Polling Sites?'' \emph{The New York
    Times}, March 24, 2016, available at
  \url{https://www.nytimes.com/2016/03/25/us/angry-arizona-voters-demand-why-such-long-lines-at-polling-sites.html}
  (last accessed December 15, 2017)} Ahead of her state's 2016
presidential primary, election administrators in Maricopa County (over
four million residents and one of the largest counties in the United
States) had cut the number of Election Day polling places by roughly
70 percent, increasing the strain on early voting
centers.\footnote{According to the 2012-2016 American Community
  Survey, 5-Year Estimates, the population of Maricopa County is
  4,088,549.}  Voters in a handful of states also experienced long
lines in the 2016 General Election, but their experiences were largely
mild compared to what transpired four years prior in
Florida.\footnote{On various states with line issues in 2016, see
  ``Why Long Voting Lines Could Have Long-Term Consequences,''
  \emph{The New York Times}, November 8, 2016, available at
  \url{https://www.nytimes.com/2016/11/09/upshot/why-long-voting-lines-today-could-have-long-term-consequences.html}
  (last accessed December 9, 2017).  On Georgia in particular, see
  ``Early voting lines are so long, people are fainting. That harms
  democracy,'' \emph{The Guardian}, October 19, 2016,
  \url{https://www.theguardian.com/commentisfree/2016/oct/19/early-voting-lines-georgia}
  (last accessed December 9, 2017).}  In 2011, the Florida state
legislature curtailed the state's early voting period, which, as some
observers had expected, resulted in long lines for many electors who
tried to vote in person during the truncated early voting period in
the 2012 General Election \citep{herron_smith2014}.  Following the
election and the seemingly inevitable anger over long waits at the
polls, then-United States President Barack Obama formed a Presidential
Commission in early 2013 to address general election administrative
issues in the United States.  In doing so, he highlighted the plight
of 102-year old Desiline Victor, a Haitian-American woman who on
October 27, 2012, waited in line for nearly four hours at Florida's
North Miami Public Library early voting center.  Not surprisingly,
Obama's Presidential Commission devoted some of its January, 2014,
report to long lines, concluding that, ``[A]s a general rule, no voter
should have to wait more than half an hour in order to have an
opportunity to vote'' \citep[p.\ 13,][]{pcea:2014}.\footnote{The
  executive order creating this commission is available at
  \url{https://obamawhitehouse.archives.gov/the-press-office/2013/03/28/executive-order-establishment-presidential-commission-election-administr}
  (last accessed July 12, 2017).}

\subsection*{Studying voting lines with surveys}

These anecdotes above are striking and raise questions about voting
lines in general---in particular, how long and which voters have to
wait, how much variance there is across jurisdictions in waiting
times, and what consequences lines have, if any, beyond the cost of
standing in line.  With few exceptions
\citep[e.g.,][]{spencermarkovits:renege, herronsmith:hanoverstudy,
  pettigrew:longlinesminorityprecincts}, much of what we know about
wait times for voters is derived from survey data. Most notably,
\citet{stewart:waitingtovote2012} and his coauthors have conducted
post-election, Internet-based surveys in the United States following
the last three general elections.  These surveys query voters about
their experiences at the polls, and the surveys include items on
estimated wait times.  \citeauthor{stewart:waitingtovote2012}'s
\emph{Survey of the Performance of American Elections} (SPAE), which
surveys 200 individuals in each state and Washington, D.C., along with
the Cooperative Congressional Election Study (CCES), have been used to
gauge relative wait times across states the as well as overall wait
times for different sub-populations of voters.  According to
\citeauthor{stewart:waitingtovote2012}, wait times across the United
States vary considerably, but a state's average wait time tends to be
consistent over time.

Surveys relying on self-reported wait times can serve as valuable
barometers for gauging both individual-level and regional
distributions of voting delays.  For example, Floridians have for a
decade consistently reported having to endure some of the longest wait
times in the country. In the 2012 General Election, in-person voters
in Florida reported on average waiting 39 minutes to cast a ballot,
three times the national average \citep{stewart:waitingtovote2012}.
Surveys like the SPAE can help identify when during the voting process
wait times are most likely to occur.  In the 2012 General Election,
\citeauthor{stewart:waitingtovote2012} documented that over
three-fifths of early in-person voters nationwide who reported waiting
in line to vote said their wait was primarily during the check-in
stage.

Surveys are also helpful when trying to tease out the possibility of
differential voting line time taxes across sub-populations of voters.
Drawing on 2008 CCES data, \cite{mukherjee:timetax} finds minority
voters were more likely, relative to White voters, to pay such a tax
when queuing to vote. \citet{kimball:voting}, using the 2012 SPAE,
reports that voters in urban areas faced longer lines than rural
voters, and \citet{pettigrew:racegapwaittimes} estimates that
typically non-white voting locations in the United States were
associated with voter wait times approximately twice as long as those
in white locations.  \cite{herron:confidence} use exit polls in
Miami-Dade in 2014 to gauge the polling place experiences of voters
leaving polling stations.  In his study of the 2012 General Election,
which builds on the 2008 General Election research design of
\citet{alvarez:survey}, \cite{stewart:waitingtovote2012} finds that a
voter's race/ethnicity is an important ``individual-level
demographic'' that explains disparate wait times.  ``African Americans
waited an average of 23 minutes to vote,''
\citeauthor{stewart:waitingtovote2012} found, ``compared to only 12
minutes for Whites; Hispanics reported waiting 19 minutes, on
average.''  \citeauthor{stewart:waitingtovote2012} concludes that
these differences in wait times could be ``due to factors associated
with where minority voters live, rather than with minority voters as
individuals'' (pp.\ 457-458).

There are several limitations with the use of survey results to
estimate voter wait times at the polls. First, surveys by construction
draw on voter self-reports.  There are a variety of reasons that
respondents might not recall accurately how long they waited in line
before voting \citep[e.g.,][]{sackettetal:timeflies}, and social
desirability may confound accurate reporting on turnout, leading to
overreporting \citep{ansolhersh:bigdata}.  Second, surveys often do
not distinguish among the various wait times that in-person voters can
experience when parking, queuing up to check-in, filling out ballots,
or processing completed ballots. Third, not all surveys have access to
tens of thousands of respondents.  Given variability across states and
even local jurisdictions in election administration practices,
national surveys may not be not particularly well-suited for assessing
variation in wait times across types of voters and polling places
within a state.

\subsection*{An alternative to voter line surveys}

In light of limitations with survey data, our approach to voting lines
draws on an alternative source of information: observed check-in times
of voters in Florida who cast their ballots at early in-person polling
sites prior to the 2012 and 2016 General Elections.  In 2016, more
Florida voters---3.88 million---cast early in-person ballots than
voters who mailed in their absentee ballots (2.76 million) or who
voted in person at their local Election Day precincts (2.96 million)
\citep{FDOS:2016vote}. The prevalence of early in-person voting is not
unique to Florida. Across the United States, early voting is
increasingly popular; for analyses of early voting reforms and their
consequences, see \citet{neelyrichardson:earlyvoting},
\citet{gronkebaum:growth}, \citet{gronketoffey:psychological},
\citet{gronke:2012}, and \citet{burdenetal:unanticipated}.

% \citet{gronke:earlyvotingreforms},


% Although there are debates in the
% literature about the extent to which early voting has changed the
% electorate, if at all, here we take this mode of voting as given.

Florida notwithstanding, there is good reason to focus on wait times
during early in-person voting.  Michael P.\ McDonald, on his Election
Project website, estimates that over 23 million voters cast early
in-person ballots in ``advance'' of the November 8, 2016, General
Election, some 17 percent of the 137 million votes
cast.\footnote{These numbers are drawn from the United States Election
  Project; see fn.\ \ref{fn:uselectionproject}.}  If one were to be
concerned that our focus on early voters in Florida limits our scope,
these statistics imply that our results apply to millions of
Americans.

Florida's open records laws make the state an excellent laboratory for
the study of election administration. For example, in their study of
congestion at the polls in Florida in the 2012 General Election,
\cite{herronsmith:closingtimes} use precinct-level data detailing when
the last voter in a precinct checked-in to vote in a study showing
significant differences in the closing times of precincts within
counties, contingent on a precinct's racial and ethnic
composition. Florida's laws also provide a wealth of information not
only about who votes, but when and how voters cast ballots.  Drawing
on publicly available voter demographic and vote history data in
Florida, \cite{shinosmith:registrationtiming} demonstrate that voters
who register to vote immediately prior to a registration deadline are
more likely to turn out in that proximate General Election, but not in
subsequent elections, and \cite{amos_etal2017} find differences in
individual-level turnout across political party and racial and ethnic
categories among registered voters whose precincts were eliminated or
relocated by local election administrators.  Similarly, by linking
multiple statewide voter files and individual-level early voting
files, \cite{herron_smith2014} demonstrate how reductions in the
number of early voting days following the 2008 General Election were
associated with a drop in early voting in 2012, especially among
minority voters who had cast ballots on the final Sunday of early
voting in 2008, a day of early voting that was eliminated by the state
legislature ahead of the 2012 election.

When it comes to direct election monitoring, however, Florida grants
considerable privacy to voters.  With the exception of those serving
as candidate, political party, or ballot issue representatives, state
law prohibits precinct observers from tracking voting processes inside
a polling place when votes are actively being cast. This effectively
precludes scholars who want to understand why lines form in front or
inside Florida precincts from replicating observational studies of
voter activities in the vein of
\possessivecite{spencermarkovits:renege} study of California and
\possessivecite{herronsmith:hanoverstudy} research on New Hampshire.

\section*{Florida early voting check-in times}

Election administration in the Florida---from registering voters, to
determining precinct sizes, to staffing and locating polling places,
and to setting early voting days and hours---is largely controlled at
the county level within a framework established by the Florida state
legislature. Our early voting check-in times are thus gleaned from
Florida counties.\footnote{We made public records requests to
  individual Florida county Supervisors of Elections, and the data use
  in this paper is based on these requests.  Most counties that we
  contacted were not able to provide us with early voter check-in
  times.}

% , and these times cover check-ins
% across a variety of early voting stations during the 2012 and 2016
% General Elections.  

A Florida voter who wishes to cast his or her ballot early may vote at
any early voting polling location in the county in which he or she is
registered.  This is distinct from in-person, Election Day voting,
during which a registered Florida voter may vote only at his or her
assigned precinct.  This flexibility necessitates that a county track
its early voters.  Rather than relying on traditional, paper-based
pollbooks to check-in these individuals, as is the case for Election
Day in many Florida counties, the 67 county Supervisors of Elections
in Florida use electronic pollbooks to check-in voters during the
state's early voting period.  These pollbooks are for the most part
known as Electronic Voter iDentification machines, or EViDs for short.
Although there is variance across Florida counties in electronic
pollbook implementation, for simplicity we refer to all electronic
pollbooks as EViD machines.\footnote{For example, Sarasota County uses
  a different electronic voting system for its early voters (phone
  call with Cathy Fowler, office of the Sarasota Supervisor of
  Elections, on March 31, 2017).}

%% Cathy Fowler:  cfowler@sarasotavotes.com

EViD machines allow county pollworkers to check-in and verify the
registration statuses and identities of Floridians casting early,
in-person ballots. The EViD system is designed to reduce the time it
takes to process early voters, and it fosters synchronization across a
county's early voting centers (important because, as just noted, early
voters in Florida can choose where to cast their ballots) as well as
with the Florida statewide voter database.  For our purposes, EViD
machines record the check-in times for all early voters, and this
provides us with timestamps that specify when a voter began his or her
voting process. EViD timestamps are recorded to the minute, and using
these timestamps we can, for example, identify early voters who had
not checked-in when polls closed at 7:00pm on a given day of early
voting but nonetheless cast ballots.  Per Florida state law, any
elector in line at 7:00pm is allowed to cast a ballot.\footnote{See
  ``The 2017 Florida Statutes,'' Title IX ELECTORS AND ELECTIONS,
  Chapter 100, Section 100.011, available at
  \url{http://www.leg.state.fl.us/Statutes/index.cfm?App\_mode=Display\_Statute\&Search\_String=\&URL=0100-0199/0100/Sections/0100.011.html}
  (last accessed December 15, 2017).}

We have EViD check-in times from six Florida counties, Alachua,
Broward, Hillsborough, Miami-Dade, Orange, and Palm Beach, and Table
\ref{tab:evidcounts} lists corresponding numbers of EViD timestamps.
In the 2012 General Election, there were 2,409,097 total early ballots
cast in Florida, and in 2016, 3,876,753 early ballots.\footnote{2012
  General Election turnout statistics are located at
  \url{http://dos.myflorida.com/media/693340/2012ballotscast.pdf}
  (last accessed December 10, 2017), and comparable 2016 General
  Election statistics at
  \url{http://dos.myflorida.com/media/697842/2016-ge-summaries-ballots-by-type-activity.pdf}
  (last accessed December 10, 2017).}  Our EViD data thus cover
approximately 39 percent and 44 percent of total ballots cast in
Florida in the 2012 and 2016 General Elections,
respectively.\footnote{The numbers in Table \ref{tab:evidcounts} do
  not include 115 EViD check-in times (28 from 2012 and 87 from 2016)
  that are listed as midnight.  We are suspicious that these times are
  not accurate and hence drop them.  Given the total number of EViD
  votes cast, dropping 115 cases is negligible.\label{fn:midnight}}

% > 942166 / 2409097 * 100
% [1] 39.10868
% > 

% > 1687217 / 3876753 * 100
% [1] 43.5214
% > 

\input{county_evid_counts.tex}

EViD timestamps, one per early voter, are not subject to potential
self-reporting biases and hence are presumably more accurate than
surveys of voter times or even within-polling place observations of
when voters checked-in.  Moreover, EViD timestamps as associated with
official Florida voter identification numbers, which we can link to
the nearly 14 million individual voter registration records in
Florida's statewide voter file.  To this end, we merge our EViD
timestamps with statewide voter files that cover the 2012 and 2016
General Elections.\footnote{Per Table \ref{tab:evidcounts}, we
  recorded 942,166 early voters across our six counties in 2012 and
  1,687,217 early voters in 2016.  Of those 2012 early voters, 40,455
  could not be matched to the registration records we have from the
  January 2014 statewide Florida voter extract file.  Of the 2016
  early voters, 969 could not be matched to the registration records
  we have from the January 2016 statewide Florida voter extract file.
  In any analysis that uses demographic or partisan information about
  the voters, we drop those voters who could not be matched to
  registration records.}

% > evid %>% filter (year == 2012) %>% count
% # A tibble: 1 x 1
%        n
%    <int>
% 1 942166
% > evid %>% filter (year == 2016) %>% count
% # A tibble: 1 x 1
%         n
%     <int>
% 1 1687217
% > 



In 2012, all 67 of Florida's counties offered between six and 12 daily
hours of early voting over an eight-day period (Saturday through
Saturday) that ended immediately prior to the General Election,
November 6, 2012.  In 2016, after long-lines in 2012 had apparently
convinced the state legislature to grant counties more flexibility in
offering early voting opportunities, counties were permitted to offer
up to 12 hours of early voting per day, spread over 14 days, ending
the final Sunday before the November 8, 2016, General Election.
Counties were permitted to offer a maximum of 168 total hours of early
voting.\footnote{See Florida Governor Rick Scott's statement, issued
  in the aftermath of the 2012 General Election, available at
  \url{http://www.flgov.com/2013/01/17/governor-rick-scott-statement-on-election-reforms/}
  (last accessed March 31, 2017).}

Figure \ref{fig:tenminhist} displays the distribution of EViD check-in
times on the last Saturday (November 3) of the 2012 General Election
early voting period in two different polling locations in Florida, one
at the Fred B.\ Karl County Center in Hillsborough County and the
other at the West Kendall Regional Library in Miami-Dade County.  The
histograms in this figure describe in ten minute blocks the number of
voters who checked into these two polling stations.  The counts begin
with the first voter who checked-in just as polls opened at 7:00am,
and they end with the last voter who checked-in.

\begin{figure}[!ht]
  \caption{Early voting check-in times on Saturday, November 3, 2012, in two Florida locations}
  \label{fig:tenminhist}
  \centering
  \begin{subfigure}[b]{\linewidth}
    \centering\includegraphics[scale = 0.6]{example00.pdf}
    \caption{Fred B.\ Karl County Center, Hillsborough County}
    \label{fig:karlexample}
  \end{subfigure}%
  \\
  \begin{subfigure}[b]{\linewidth}
    \centering\includegraphics[scale = 0.6]{example01.pdf}
    \caption{West Kendall Regional Library, Miami-Dade County}
    \label{fig:kendallexample}
  \end{subfigure}
\end{figure}

Several features of Figure \ref{fig:tenminhist} are notable.  First,
both panels in the figure contain vertical black lines at 7:00pm.
This is the time beyond which additional voters were not allowed to
join a (possibly existing) voting line.  In the Hillsborough County
location (Figure \ref{fig:karlexample}), no check-ins occurred after
7:00pm, and from this it follows that, at 7:00pm, where was no voting
line.  In contrast, the Miami-Dade location (Figure
\ref{fig:kendallexample}) had many post-7:00pm check-ins, the last one
of which occurred around 1:00am on Sunday, November 4.  We thus know
that the last-voting voter at West Kendall Regional Library waited at
least almost six hours to vote.

Figure \ref{fig:tenminhist} incorporates information on voter
self-reported race/ethnicity, and corresponding details are depicted
via bar shading.  Early voters at the Fred B.\ Karl County Center were
primarily white, although there were periods on November 3 when the
fraction of non-white votes was disproportionately high, e.g., toward
the end of the day.  At the West Kendall Regional Library, though, the
vast majority of early voters were non-white.  The fraction of
non-white voters appears plausibly consistent across November 3.

Lastly, and very roughly speaking, we observe a flatter or
more-uniform distribution of check-in times at Karl County Center than
in West Kendall Library.  We will draw on this fact later, when we try
to determine which early voters in our six counties waited in line,
and here we provide some intuition.  As illustrated in Figure
\ref{fig:tenminhist}, it appears that early voting locations in
Florida that shut down very late had flatter distributions of check-in
times.  We suspect that this is evidence of persistent lines.  In
contrast, the non-uniformity of check-in times that we observe in
Figure \ref{fig:karlexample} is consistent with more of an ebb and
flow of early voters.  We know that there was not a line to vote at
Karl County Center at 7:00pm on Saturday, November 3, this despite the
fact that, per check-in times, the Center was regularly processing
voters.

Figure \ref{fig:tenminhist} highlights the value of EViD check-in
times as well as their limitations; check-ins are not linked to
arrivals.  All data sources have advantages and disadvantages, and we
will draw on the former and attempt to work with the latter as we turn
to results.

\section*{Results}

We present results in two sections.  First, we describe patterns in
check-in times across the 2012 and 2016 General Elections with
particular attention to race/ethnicity and partisanship.  Second we
consider the consequences of extensive early voting wait times.

% based
% on a identifying assumption we use to characterize individuals who
% waited in line before voting.

\subsection*{Who waits?}

Our analysis of who waits to vote turns in large part on check-in
times that occurred \emph{after} 7:00pm on a day of early voting.  Any
voter with such a late check-in must have waited to vote.

\subsubsection*{Early voting in 2012}

Across our six Florida counties of interest, there were 78 early
voting stations in 2012.  These serviced a total of 942,166 voters in
the eight day long, 2012 early voting period.\footnote{This ignores
  the small number of EViD check-in times that look suspicious; see
  fn.\ \ref{fn:midnight}.  In addition, locations where two or fewer
  individuals recorded votes were dropped from our analysis.  The EViD
  data from Orange County and Palm Beach County in 2012 and 2016, from
  Miami-Dade County in 2016, and from Broward County in 2016 was not
  accompanied with voting locations for each early voter.  To
  determine the location at which these voters case their ballots, we
  match their voterids to a masterfile in Florida that records the
  location of every vote.}  % XXX fix masterfile

%> unique(evid12$county)
%[1] "ALA" "BRO" "HIL" "DAD" "ORA" "PAL"

%> length(unique(evid12$location))
%[1] 78

\begin{figure}[!ht]
  \caption{Number of locations where early votes were cast, 2012 General Election}
  \label{fig:nrlocs2012}
  \centering
    \centering\includegraphics[scale = 0.8]{number_of_Locations.pdf}
\end{figure}

Figure \ref{fig:nrlocs2012} describes by day of early voting and by
hourly window the number of locations across our six counties that
actively served voters.  All locations served early voters virtually
the entire day, and this is evident in the flat line, for the most
part pegged a bit under 80 prior to 7:00pm, at the top of the figure;
the sole exception occurred at the earliest time of the day, during
which a few early voting stations did not have any active voters.

After 7:00pm, however, Figure \ref{fig:nrlocs2012} shows that check-in
uniformity across locations quickly changed.  On initial days within
Florida's eight-day long early voting period, there was a steep drop
in active locations starting at 7:00pm; note the vertical black line
at this time.  Even with this drop, however, there were still at least
five open locations at 9:00pm on every day of early voting.  Then, on
the last two days of early voting, November 2 and 3, many early voting
locations were open well beyond 7:00pm.  On the final Saturday of
early voting, over 30 locations were still open at 9:00pm, and a few
locations continued processing voters through midnight.

\begin{figure}[!ht]
\caption{Distribution of check-in times among early voters by hour, 2012 General Election}
  \label{fig:hist2012}
  \centering
    \centering\includegraphics[scale = 0.8]{histogram_by_hour.pdf}
\end{figure}

Figure \ref{fig:hist2012} displays the total number of voters who
voted early by the hour of the day at which they checked-in. Prior to
7:00pm, EViD check-ins were distributed fairly uniformly across the
day with a small peak before noon.  From 7:00am to 7:00pm, our
aggregated six counties consistently served approximately 70,000
Florida early voters per hour. Then, after 7:00pm, at which point new
voters could not join existing voting lines, the number of voters
served per hour dropped drastically---albeit but not entirely.  A
total of 83,250 (8.8\%) early voters in 2012 across the six counties
cast ballots after 7:00pm. These represent individuals who were in
line at the time the polls closed and were allowed to continue to wait
to vote.  We know that all of these voters had to wait in this way
although we do not know precisely how long each waited.  Of this
group, 36,120 voted after 8:00pm.  These voters waited at least one
hour to vote, and 13,567 voters, who checked-in after 9:00pm, must
have waited at least two hours to vote.

% > evid12 %>% filter(time2 > 19) %>% count()
% # A tibble: 1 × 1
%       n
%   <int>
%   83250

% > evid12 %>% filter(time2 > 20) %>% count()
% # A tibble: 1 × 1
%       n
%   <int>
%   36120

% > evid12 %>% filter(time2 > 21) %>% count()
% # A tibble: 1 × 1
%       n
%   <int>
%   13567

\begin{figure}[!ht]
\caption{Racial/Ethnic composition of early voters by hour, 2012 General Election}
  \label{fig:race2012}
  \centering
    \centering\includegraphics[scale = 0.8]{racial_composition.pdf}
\end{figure}

Aggregating across locations, Figure \ref{fig:race2012} describes the
composition of the 2012 early voting pool by race/ethnicity and by
hour of check-in.  For most of the day, whites were the majority
racial group, followed by blacks, Hispanics, and Asians.  This ranking
is similar to, but does not mirror, Florida's registered voter pool.
In the December 2012 Florida voter file, which lists 12,580,602
individuals, approximately 66.4 percent are white, 13.9 percent
Hispanic, 13.6 percent black, and 1.63 percent Asian.  As in years
prior, blacks in Florida in 2012 were disproportionate users of the
state's early voting period \citep{herronsmith:souls}.

% mysql> select count(*) from fLvoterfile_dec_2012_extract;
% +----------+
% | count(*) |
% +----------+
% | 12580602 |
% +----------+
% 1 row in set (4.20 sec)

% mysql> select race, count(*) AS num, round(100 * count(*) / (select count(*) from fLvoterfile_dec_2012_extract),2) AS percent from fLvoterfile_dec_2012_extract group by race;
% +------+---------+---------+
% | race | num     | percent |
% +------+---------+---------+
% |    0 |     220 |    0.00 |
% |    1 |   42165 |    0.34 |
% |    2 |  204542 |    1.63 |
% |    3 | 1711107 |   13.60 |
% |    4 | 1748251 |   13.90 |
% |    5 | 8358730 |   66.44 |
% |    6 |  203878 |    1.62 |
% |    7 |   29860 |    0.24 |
% |    8 |       4 |    0.00 |
% |    9 |  281839 |    2.24 |
% |   10 |       5 |    0.00 |
% |   11 |       1 |    0.00 |
% +------+---------+---------+
% 12 rows in set (5.72 sec)

% mysql>

% RACE CODES
% Race Code	Race Description
% 1	American Indian or Alaskan Native
% 2	Asian Or Pacific Islander
% 3	Black, Not Hispanic
% 4	Hispanic
% 5	White, Not Hispanic
% 6	Other
% 7	Multi-racial
% 9	Unknown

What is striking in Figure \ref{fig:race2012}, however, is the
racial/ethnic composition of the early voting pool immediately before
and then after 7:00pm.  Simply put, the pool becomes rapidly non-white
starting around 7:00pm; by the 8:00pm-9:00pm window, the pool is less
than 25 percent white.


The party registration of the early voting pool varies slightly with
time although not nearly as starkly as its racial/ethnic composition.
This is illustrated in Figure \ref{fig:party2012}, which describes by
hour the partisan breakdown of all early voters in our six counties as
well as the party breakdown of the four aforementioned racial/ethnic
groups.  While Democrats made up 54\% of all those who voted early in
our set of Florida counties, we can see from the black line in the
figure that this percentage varied by hour of the day.  Notably,
Democrats composed a greater share of the voters in both the early
hours of voting and the hours after 7:00pm.  Hence, Democrats were
disproportionately affected by voting lines that forced voters to cast
their ballots after 7:00pm.

\begin{figure}[!ht]
\caption{Partisan composition of early voters by hour and by race, 2012
  General Election}
  \label{fig:party2012}
  \centering
    \centering\includegraphics[scale = 0.8]{partisan_composition_by_race.pdf}
\end{figure}

%  XXX Need to check 54% figure above.  
%table(evid12$party)/nrow(evid12)
%
%        DEM         IDP         NPA         OTH         REP 
%0.541646589 0.002073945 0.163840555 0.057074868 0.235364044 

If we consider the breakdown of party registration by race/ethnicity,
as in Figure \ref{fig:party2012}, we can see how this effect might
largely be due to the aforementioned racial/ethnic patterns of voting.
For example, black early voters in Florida were almost entirely
Democratic, regardless of when they checked-in to vote.  Similarly,
the party registration of Hispanic voters---the least Democratic group
in the six counties---remained relatively constant throughout the day.
However we do see a slight trend for white and Asian early voters, who
became increasingly likely to be registered Democratic as time
progressed.

% > evid12 %>% summarize(pct = mean(party == "DEM"))
% # A tibble: 1 x 1
%        pct
%       <dbl>
% 1 0.5416466

\subsubsection*{Early voting in 2016}

We now turn to early voting in the 2016 General Election.  By the time
this election had occurred, Florida had increased it number of early
voting days from eight to 14, ostensibly in an effort to reduce voting
location congestion. Was this change effective?  Figure
\ref{fig:nrlocs2016} is analogous to the earlier figure that described
active early voting locations. Our six counties had 104 early voting
stations in place for the 2016 General Election.

% > length(unique(evid16$location))
% [1] 102

\begin{figure}[!ht]
  \caption{Number of locations where early votes were cast, 2016 General
    Election}
  \label{fig:nrlocs2016}
  \centering
    \centering\includegraphics[scale = 0.8]{number_of_locations_2016.pdf}
\end{figure}

The most important aspect of Figure \ref{fig:nrlocs2016} is the
pictured dropoff in voter check-ins that occurred after 7:00pm.  There
were indeed check-ins that took place after this time but not nearly
as many as in 2012.  On the busiest day in 2012, more than half of the
polling locations were open past 9:00pm and more than five were open
past midnight.  On the busiest day in 2016, only six locations were
open past 9:00pm and not one early voter cast a ballot after 10:00pm.

In the introduction, we commented on improvements in voter wait times
that others, using surveys, have found in 2016 compared to 2012, and
Figure \ref{fig:nrlocs2016} is consistent with findings in the 2012
\citep{spae2012} and 2016 \citep{spae2016} versions of the SPAE.  See
Table \ref{tab:floridaspae}.  Early voting in 2016---at least in the
six Florida counties---appears to represent an improvement over 2012
in terms of reducing congestion, which is presumably tied not only to
the expanded number of sites but also to the number of days and hours
of allowable early voting.\footnote{We use the verb ``appears'' here
  because of voting-level sorting that may affect the types of
  individuals in Florida who cast their ballots prior to a given
  Election Day.  When the Florida state legislature changed the
  state's early voting period between 2012 and 2016, voters may have
  altered their most preferred voting times.}

\input{spae_florida.tex}

\subsubsection*{Late voting in 2012 versus 2016}

Figure \ref{fig:race2012and2016} provides a race/ethnicity-based
perspective on the 2012 versus 2016 comparison for our six counties.
For three key racical/ethnic groups in Florida---black, Hispanic, and
white---the figure reveals the total number of check-ins by time,
aggregated across all days in the 2012 and 2016 early voting periods.

Figure \ref{fig:race2012and2016} shows clearly that, not only did more
voters check-in for early voting in 2016 compared to 2012, but fewer
voters voted past 7:00pm.  This is the case for black, Hispanic, and
white voters. Moreover, non-white early voters in Florida had
disproportionately late check-ins in 2012.  We already have seen
evidence of this, but Figure \ref{fig:race2012and2016} shows as well
how this problem did not appear in 2016.  There were slightly more
non-whites with late check-ins in 2016, but the magnitude of the white
versus non-white gap shrunk between the two years.

\begin{figure}[!ht]
  \caption{Distribution of voter check-ins, 2012 General Election versus 2016 General Election}
  \label{fig:race2012and2016}
  \centering
  \centering\includegraphics[scale = 0.8]{histogram_by_hour_by_race_2012_2016.pdf}
\end{figure}

Even though 2016 attracted far more early voters than 2012, the
changes made to the early voting period appears to have reduced the
congestion outside of the polls. The increased number of days and
hours may have helped, as these six counties still served roughly the
same number of early voters per day in 2016 (approximately 120,516) as
they did in 2012 (approximately 117,771). 

%  Given reduced prevalence of
% lines in 2016, it is likely that these and other reforms made the
% difference for congestion.

%> 942166/8
%[1] 117770.8
%> 1687217/14
%[1] 120515.5


%% Make a plot using all early voters in all sites, but this is
%% post-midwest!  XXX  (MCH: I am not sure what this comment means)

\subsection*{Effects of waiting to vote}

We now consider the effects on future electoral political
participation of waiting in line to vote.  In particular, we focus
here on the consequences of waiting in an early voting line in the
2012 General Election on turnout in the 2016 General Election.

We have already noted that the individual-level EViD files we have
include Florida voter identification numbers, and this enables us to
link these files with statewide Florida voter files.  These latter
files specify the elections in which voters participated and, if so,
whether they voted absentee, early in-person, or on Election Day.
Thus, for any early voter in 2012 we can determine whether the
individual voted in the 2016 General Election, assuming that this
individual still lived in Florida as of November, 2016.\footnote{When
  a registered voter moves within Florida, the voter maintains the
  same voter identification number.  Our estimates in this section are
  thus not confounded by the possibility of 2012 early voters moving
  across county lines within Florida.}

The difficulty in our exercise here is twofold.  First, while we know
voter check-in times from our EViD data, we do not know associated
voter wait times \emph{per se}.  To determine wait times precisely we
would have to know early voter arrival times, which are not collected
by election officials or anyone else for that matter.  Second, early
voters who voted at, say, 9:00am on a given day of early voting may be
systematically different from those who voted at 6:00pm. Consequently,
estimating the effect of voting after closing time---when we know
voters have waited in line---is potentially confounded by voter-level
selection.  Hence, we need to ensure that we control for differences
across early voters as best as we can so that selection into time of
early voting does not confound our estimates of the effect of waiting
on future turnout.

% Hence,
% distinguishing those early voters in 2012 who waited in line from
% those who did not is a challenge and requires an alternative method
% along with some assumptions.  

Our approach is as follows.  We condition on race/ethnicity,
partisanship (as before, measured by party registration), gender, age,
and previous voting history.  And, we make the following assumption:
individuals who voted at Florida early voting locations which stayed
open well past the 7:00pm cutoff time are more likely to have waited
in line than those who voted in locations that closed before 7:00pm or
at this time exactly.  Moreover, we assume that those who voted just
before the 7:00pm cutoff in early voting locations that stayed open
late are more likely to have waited in line than those who voted
earlier in the day.  We cannot know for sure if these assumptions
hold, but there are reasons to believe that they do.

Take, for example, the two polling locations in Figure
\ref{fig:nrlocs2012}.  The first location---the Karl County Center in
Hillsborough County---closed on time on November 3, 2012.  Therefore,
we assume that the Karl Center was not a congested polling location
and that the voters who voted just before 7:00pm did not have to wait
in much of a line.  If they did have to wait in line to vote, then the
line remarkably stopped just in time for the final early voter to
check-in right before 7:00pm.  While this is technically possible, it
would be remarkably coincidental.

On the other hand, the second location in Figure
\ref{fig:nrlocs2012}---the West Kendall Library in Miami-Dade
County---remained open until around 1:00am on November 3.  We know
with certainty that this last voter waited in line to vote.  In fact,
because all voters had to arrive before 7:00pm in order to check-in,
we know with certainty that he or she waited at least six hours to
vote.  Moreover, we can be fairly confident that the voters who voted
just before 7:00pm here also waited in line.  Voting that continues
past 7:00pm is indicative of a polling place that was operating
essentially at capacity at 7:00pm.  Hence, those voters who voted just
before 7:00pm are similarly likely to have been caught up in a long
voting line.

With the above assumptions as background, our strategy for
distinguishing voters who waited in line from voters who did not wait
in line is first to identify those who voted in an an early voting
polling place that closed well past 7:00pm from those who voted in a
polling place that closed on time.  In other words, we attempt to
compare voters who cast their ballots in places like the Karl County
Center to voters who voted in places like the West Kendall Library.
More specifically, we want to compare those individuals who voted
closest to the 7:00pm cutoff, since those are the individuals who we
are most confident were affected by the congestion that caused polling
locations to close late.
 
Therefore, we create a variable that identifies all early voters who
voted at a polling location on a day where the last voter checked-in
past 7:30pm.  These are voters who cast ballots at locations where we
know the line at the end of the day was at least 30 minutes long.  We
call this variable \emph{Over}.  Conditioning on race/ethnicity,
partisanship, gender, age, log median income at the zip-code level,  and previous voting record (whether the
voter voted in the 2008 General Election), we use a logistic
regression to estimate the effect of voting at a polling location that
is congested in 2012---one that goes ``over'' 7:30pm---on the
probability of voting in 2016:
\begin{equation*}
  \begin{aligned}
    \textup{Pr}\left(\mathrm{Voted16}_{i} = yes\right) =  \textup{logit}^{-1}&(\beta + \alpha_{Over} + \gamma_{Hour} +
    \sigma_{Over \times Hour} + \upsilon_{Gender}  + \\& \rho_{Race/Ethnicity} +
      \tau_{Age Group} + \psi_{Party} + \pi_{Voted\emph{08}})
  \end{aligned}
\end{equation*}
%
In the above model, $i$ denotes our collection of 2012 early voters
who appear in the history of Florida voting in 2008; $Hour$ indicates
the hour of the day in 2012 at which each voter checked-in, between
7:00am and 7:00pm; $Gender$ indicates whether the voter identifies as
male or female; $Race/Ethnicity$ indicates if the voter
self-identifies as white, black, Hispanic, or Asian; $AgeGroup$
classifies voters according to their 2016 age and bins them into
10-year age groups (20-29, 30-39, 40-49, 50-59, 60-69, and 70+); and,
$Voted08$ and $Voted16$ are indicators for participation in the 2008
and 2016 General Elections.  All told we estimate our logistic
regression based on 758,266 individuals, and coefficient estimates
appear in Table \ref{tab:reg} in the appendix.\footnote{Our regression
  is estimated only using individuals who voted before 7:00pm.  In
  addition, we restrict attention to individuals who are registered
  either Democrat, Republican, or Independent, or have recorded no
  party affiliation.  Finally, we drop registered voters who have no
  recorded gender, race, or birthdate.}

Per the left column of Table \ref{tab:reg}, having voted in 2008 in
Florida is positively associated with subsequent voting in 2016, and
being older is associated with greater turnout, \emph{ceteris
  paribus}.  Our age estimates in Table \ref{tab:reg} are presumably
underestimated; the regression model excludes the youngest voters in
2016, who could not legally register to vote in Florida in 2008, and
younger registered voters are, \emph{ceteris paribus}, less likely to
turnout \citep{shinosmith:registrationtiming}.  Table \ref{tab:reg}
also shows that, \emph{ceteris paribus}, men vote less often than
women and that whites vote often than non-whites.  This gender finding
is consistent with \citet{leighleynagler:whovotesnow} although the
race result is not.

Our objective is not to explore the relationship between demographic
covariates and turnout in Florida in 2016.  Rather, we want to know
about the marginal effects of waiting in line, and to this end Figure
\ref{fig:prvoting2016} displays graphically the probability of voting
in 2016 of a black male between the ages of 50-60 who is a registered
Democrat, voted in 2008, and is living in a zip-code with the median
income.

Figure \ref{fig:prvoting2016} estimates are conditioned on each hour
of the early voting day in 2012. Its black points are the estimated
probabilities of voting in 2016 conditional on voting at a polling
location that closed \emph{before} 7:30pm; based on our discussion
above, these estimates reflect individuals who likely did not wait in
line.  In contrast, the grey points in Figure \ref{fig:prvoting2016}
are the estimated probabilities of voting in 2016 conditional on
casting an early ballot at a polling location that closed \emph{after}
7:30pm.  These estimates reflect individuals who likely did wait in
line to vote.  Hence, the difference between the figure's two sets of
estimates should reflect the effect of waiting in line.  Moreover, the
effect should be most isolated as one compares voters casting ballots
closer to 7:00pm.

\begin{figure}[!ht]
\caption{Probability of voting in 2016, given 2012 check-in time}
  \label{fig:prvoting2016}
  \centering
    \centering\includegraphics[scale = 0.8]{probability_of_voting_in_2016_over_under.pdf}
\end{figure}

One can see that, beginning at around 2:00pm, there is a small albeit
negative difference between individuals who voted in congested polling
places compared to those in non-congested polling places.  This might
suggest that the lines started in the early afternoon and why a
difference is induced so early.  Hence, the reason there is no
difference between congested and non-congested polling places earlier
in the day is plausibly because long lines have yet to form. As
congested polling places become congested in the afternoon, we begin
to see a small but significant effect on future participation, an
effect which suggests that voting lines slightly dampen future
electoral participation.  However, the effect is small, amounting to
no more than roughly a percentage point difference.

Voting probability plots for other demographic groups are similar, and
rather than showing these we generalize effect sizes to our complete
sample.  Figure \ref{fig:margfx2016} displays estimates, along with 95\%
confidence intervals, of the change in probability of voting in 2016
based on time of early vote in 2012.\footnote{We generated the results
  in Figure \ref{fig:margfx2016} with a simulation.  Our simulation (500
  repetitions) drew from a multivariate normal distribution where the
  mean vector consists of values of the coefficient vector
  corresponding to the left-hand column in Table \ref{tab:reg}.  The
  covariance matrix for the multivariate normal is the estimated
  covariance matrix from our logistic regression.}  The numbers in
Figure \ref{fig:margfx2016} can be thought of as average treatment effects
where the treatment on individuals in our sample is being forced to
wait in a line in 2012.

%\input{table_out_ate.tex}

\begin{figure}[!ht]
\caption{The effect of waiting in line to vote in 2012 on voting in 2016}
  \label{fig:margfx2016}
  \centering
    \centering\includegraphics[scale = 0.8]{margfx_2016_vote.pdf}
\end{figure}


The estimates in Figure \ref{fig:margfx2016} are all negative, indicating that
waiting in line in 2012 led to decreased turnout probabilities in
2016.  The magnitude of the estimates range from very small (e.g., a
0.03 percentage point difference in turnout probabilities associated
with waiting to vote at 7:00am) to slightly over one percentage point
later in the day.

Overall, our findings on the effects of lines---they depress future
turnout, albeit barely---are similar to the conclusions in
\citet{pettigrew:longlinesminorityprecincts}.  In particular,
\citeauthor{pettigrew:longlinesminorityprecincts} argues that each
hour an in-person voters spends waiting in line to vote is associated
with a one percentage point drop in future turnout.  To the best of
our knowledge, \citeauthor{pettigrew:longlinesminorityprecincts} is
the first scholar to try to estimate the effect of waiting in line to
vote on the later propensity to vote, and our results are
qualitatively similar to his.

In the interests of transparency, we want to draw attention to the
sample of registered Florida voters whose turnout decisions in 2016
were analyzed with a logistic regression.  We estimated our regression
using individuals who were registered to vote in Florida in 2008,
2012, and 2016.  This means that our analysis excludes the youngest
registrants in Florida who were not legally permitted to be registered
in Florida as of the 2008 General Election; individuals who moved to
Florida from out of state between the years of 2008 and 2016;
registered Floridians who died, were convicted of committing a felony,
or adjudicated mentally incompetent between 2008 and 2016; and,
individuals who moved out of the state between 2008 and 2016.

A potential concern is that, by selecting against younger voters, we
are biasing our estimates of the effect of waiting in line on future
turnout. To this end, there is a literature on waiting in line in
hospitals, and studies of the willingness of individuals to leave
emergency rooms without being seen find that being young is a risk
factor for this behavior
\citep{sunetal:lwbs,clareycooke:emergencyroomleave,shaikh:howlongwaiter}.
In other words, young individuals seems unduly sensitive to wait
times. If this is true, then by excluding relatively younger voters in
our regression model, we are making our results about the effect of
waiting on future turnout conservative.

As noted, our regression analysis also selects against, say, recent
movers and individuals who became felons between 2008 and 2016.  We do
not know of any theoretical or empirical reasons to think that these
types of individuals are systematically different in terms of
sensitivity to voting lines.

% While there may be idiosyncrasies in the rates at which the above
% types of individuals vote---e.g., young voters tend to have
% disproportionately low turnout rates
% \citep{shinosmith:registrationtiming}---whether these types of
% individuals are unusually sensitive to lines is unclear.  


% %% XXX Below might be wrong...
% Consider our exclusion of the youngest voters, and as context note
% that younger registered voters are \emph{ceteris paribus} less likely
% to turn out \cite{shinosmith:registrationtiming} than older
% counterparts.  If younger voters are more impatient then older
% voters, we will underestimate the effect of long lines on future
% turnout, to the extent younger voters were caught in lines in 2008 or
% 2012.  With respect to the other categories of excluded voters, it is
% difficult to imagine that they are have any systematic biases
% although, in principle, sensitivity to lines could be correlated with
% the willingness to live in Florida and thus to be listed in multiple
% voter files.

Now restricting attention to 2016 voters only, we consider a second
logistic regression that analyzes whether these individuals voted
early in 2016---or, in contrast, voted in another fashion.
Corresponding regression results are in the right column of Table
\ref{tab:reg}, and Figure \ref{fig:atevotingearly2016} contains a plot
of early voting probabilities that is structured similarly as our
earlier probability plot.

\begin{figure}[!ht]
\caption{The effect of waiting in line to vote in 2012 on voting early in 2016 and voting 2014}
\begin{subfigure}{.5\textwidth}
    \centering\includegraphics[width = \linewidth]{margfx_2016_vote_early.pdf}
    \caption{Voting early in 2016}
     \label{fig:atevotingearly2016}
 \end{subfigure}
 \begin{subfigure}{.5\textwidth}
 \centering\includegraphics[width = \linewidth]{margfx_2014_vote.pdf}
 \caption{Voting in 2014}
  \label{fig:prvoting2014}
\end{subfigure}
\end{figure}

Figure \ref{fig:atevotingearly2016} shows that the 2016 voters whom we
identify as having waited in an early voting line in 2012 were less
like to vote early in 2016, \emph{ceteris paribus}.  This is intuitive
(voters responded to high costs of early voting but choosing an
alternative way to vote) and highlights a downstream consequence of
voting lines.  The percentage point gaps (black dot to grey dot) are
around one to two points, which is small albeit not negligible.  Early
voting, after all, was designed as a convenience, and Figure
\ref{fig:atevotingearly2016} suggests that the value of this
convenience can be lost in the shadow of a previous bad experience.

There is a U-shape to the points in Figure
\ref{fig:atevotingearly2016}, and we can only speculate as to what is
responsible for this.  We know from the number of post-7:00pm closing
times that, in 2012, early voting locations were congested in the
evening.  We suspect that morning early voting hours also suffered
from congestion albeit from a different source, when individuals
stopped by to vote, say, before going to work.  If lines discourage
future early voting, this pattern would yield a U-shape as in Figure
\ref{fig:atevotingearly2016}.  It is also interesting to note that
early voters in 2012 who voted at locations that closed before 7:30pm
also have a U-shape in their early voting probabilities.  We suspect
this is because these locations probably suffered from morning and
evening lines, albeit not very long ones.\footnote{This discussion
  suggests that employed individuals may be disproportionately
  sensitive to waiting in line to vote, which would be consistent with
  the relationship between affluence and tolerance for waiting in a
  supermarket line \citep[e.g.,][]{bennett:waitinginlinechars}.  There
  is no employment data in the Florida statewide voter files, but a
  relationship between work opportunities and one's willingness to
  wait in line is worth considering in future research.}

A full analysis of Figure \ref{fig:atevotingearly2016} is beyond our
scope.  But, what is most important about the figure is that it 
highlights how voting experiences can affect forms of future political
participation.  Wanting to voting early, but not doing so on account
of fear of lines, imposes a cost on voters, and future research on
election administration needs to pay attention to the way that
electoral experiences, both positive and negative, can accumulate over
time and potentially affect future choices. 

%in particular
%if their Election Day polling place is relocated \cite{amos_etal2017}.

\section*{Conclusion}

The extent to which all voters in an election are treated equally
depends in part on whether the cost of voting in the election is
distributed uniformly across voters. Administrative aspects of
elections that affect the cost of voting are thus part of the overall
calculus of whether an election was characterized by equal treatment
or the lack thereof.  A notable administrative aspect of elections is
the presence of voting lines, and lengthy voting lines in recent
United States General Elections have been of concern, both to scholars
and election officials.  Burdensome wait times not only impose
opportunity costs on those forced to wait in line, but they can also
discourage future electoral participation.

Research on voting lines is challenged by the lack of observational
data on who waits to vote.  Other than noting the times at which polls
officially closed and recording information about which voters cast
their ballots after voting lines were capped, there is little data
collected by election officials that might distinguish between those
who waited in line to vote from those who did not.  Notwithstanding a
limited number of exceptions, local election officials do not maintain
official records on line evolution or how long individual voters
waited \citep{herron:confidence}.

% Previous research using election
% records to study lines have thus been limited to examining variables
% like aggregate precinct results
% \citep[e.g.,][]{herronsmith:closingtimes,
%   pettigrew:longlinesminorityprecincts}.  Data limitations have
% prevented these projects from identifying individuals who waited in
% line from those who did not \emph{within} the same precinct.

To overcome the lack of data that identifies who waits in line to
vote, scholars have relied on surveys that ask respondents to recall
their experiences at the polls and report the length of time that they
waited to vote \citep{stewart:waitingtovote2012,
  pettigrew:racegapwaittimes}.  These survey-based analyses have
illuminated important variation in wait times across geographies and
racial/ethnic groups, but associated with them is the caveat that
survey responses are subject to biases associated with self-reporting.

With these points in mind, we contribute to the literature on voting
lines by analyzing early voter check-in times in Florida's 2012 and
2016 General Elections. Using electronic voting check-in time stamps
from Alachua, Broward, Hillsborough, Miami-Dade, Orange, and Palm
Beach Counties, we are able to identify the exact day and time at
which many thousands of voters checked-in to vote.  These check-in
times not only provide novel insight into voting patterns across time
but, given some natural assumptions, also allow us to characterize
individual voters who likely waited in line to vote. Of course,
despite their advantages, EViD check-in times are not a panacea in the
quest to understand voting lines and their consequences, as such times
do not specify how long an individual may have waited to vote.  

% As
% such, our our analysis of check-in times should be viewed as a
% complement to studies using survey and other observational methods to
% explain who waits in line and for how long.

Our analysis reveals that Florida's early voting period in 2012 had a
significant number of voters who voted past the official 7:00pm
closing time and, therefore, must have waited in line to vote.  These
voters were disproportionately black, Hispanic, and Democratic---a
finding that coincides with survey results on voting lines.  In 2016,
we find that congestion observed in 2012 had all but vanished.  Voters
rarely voted late into the night in 2016, and most early voting
locations in our Florida counties of interest closed promptly.  Even
though the counties we study in 2016 served as many individual per day
as they had in 2012, they appear to have been more prepared to handle
the high rates of early voting.
  
We estimated the effect that waiting in line has on future electoral
participation.  Using temporal variation in voting across polling
locations, we found a very slight negative effect on turnout,
amounting to no more than a one percentage point decrease in the
propensity to vote. The result is similar to that identified by
\citet{pettigrew:racegapwaittimes}.  The effect is small in magnitude,
which is normatively pleasing, but any negative effect on future
turnout is normatively troubling.  Not only are some voters penalized
by waiting in line, but a small number of these individuals appear to
be dissuaded from voting in the future, thus lessening their voices.
We offered some evidence as well that long early voting lines dissuade
voters from casting early ballots.  To the extent that early voting is
a form of ``convenience'' voting, this result shows how the value of
this convenience can be compromised by long lines, something that most
observers would classify as administrative failures.

Overall, our evidence shows that the 2012 General Election in Florida
did not treat all voters equally with respect to time spent waiting in
line.  Minority voters in 2012 voters faced disproportionately long
lines---thus, disproportionately high costs of voting---and then, in
some cases, a small downstream effect of a decreased propensity to
vote.  Our results illustrate how administrative aspects of elections,
like the resource allocation decisions that influence whether an
election will have lines or not, should be understood not just as
administrative in nature but also as relevant to underlying fairness
and the extent to which voters are treated in a like manner.

% Except for the few states that require voters to cast ballots by mail
% (or use drop off boxes), waiting to vote at the polls is likely to
% continue to be an inevitable part of the voting experience for voters
% in certain jurisdictions. Long lines may result in some voters not
% only having to pay a time tax, but they may also serve as a deterrent
% on future voting for some citizens. In Florida, the expansion of the
% number of days, hours, and locations of early voting between the 2012
% and 2016 General Elections likley helped to alleviate some of the
% dealys experienced by those voters who chose to voter during the early
% voting period, especially those casting ballots late in the day.
% Further research needs to be conducted on how such delays at the polls
% may affect future turnout decisions.



%\section*{Appendix}
%\input{../plots/table_out.tex} 

\clearpage
\newpage

\bibliographystyle{apsr}
\bibliography{evid-cites}

\newpage
\appendix
\section*{Appendix}

\input{../plots/table_out.tex}

\end{document}


 
