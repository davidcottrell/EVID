\documentclass[12pt,titlepage]{article}

% \usepackages21{fancyhdr}

\usepackage[printwatermark]{xwatermark}

\usepackage{grffile}
\usepackage{xcolor}
\usepackage{lipsum}
\usepackage{times}
\usepackage{soul}
\usepackage{epsfig}
\usepackage{rotating}
%\usepackage[hyphens]{url}
%\usepackage[breaklinks=true]{hyperref}
\usepackage{hyperref}
\usepackage{etoolbox}
\appto\UrlBreaks{\do\-}

\usepackage{latexsym}
\usepackage{graphicx}
\usepackage{amsfonts}
\usepackage{amsmath, amsthm, amssymb}
\usepackage{setspace}
\usepackage{natbib}
\usepackage{longtable}
\usepackage{keyval}
\usepackage{caption,subcaption}
\usepackage{arydshln}

%% set 1-inch margins
\usepackage{fullpage}

%% APSR submission: no commas in citations between name and year
%% See http://merkel.zoneo.net/Latex/natbib.php
\bibpunct{(}{)}{;}{author-year}{}{;}

% the opening bracket symbol, default = (
% the closing bracket symbol, default = )
% the punctuation between multiple citations, default = ;
% the letter `n' for numerical style, or `s' for numerical superscript style, any other letter for author-year, default = author-year;
% the punctuation that comes between the author names and the year
% the punctuation that comes between years or numbers when common author lists are suppressed (default = ,);

\usepackage{footmisc}
\renewcommand{\footnotelayout}{\doublespacing} % set spacing in footnotes
\newlength{\myfootnotesep}
\setlength{\myfootnotesep}{\baselineskip}
\addtolength{\myfootnotesep}{-\footnotesep}
\setlength{\footnotesep}{\myfootnotesep} % set spacing between footnotes

% make footnote font size same as regular font size in text
\renewcommand{\footnotesize}{\normalsize} 

%% make possessivecite macro since this does not exist in natbib
\newcommand{\possessivecite}[1]{\citeauthor{#1}'s (\citeyear{#1})}

%% Use this for a "DRAFT" watermark
% \newwatermark[allpages,color=pink!30,angle=45,scale=5,xpos=-25,ypos=40]{DRAFT}

%% List all locations for graphics here
\graphicspath{ {../Plots/} }
\begin{document}
\sloppy
\thispagestyle{empty}

%% APSR submission requires double-spaced footnotes
%%\newcommand{\footnote}[1]{\footnote{\doublespacing #1}} %% <-- note \doublespacing here.

\renewcommand{\topfraction}{.85}
\renewcommand{\bottomfraction}{.7}
\renewcommand{\textfraction}{.15}
\renewcommand{\floatpagefraction}{.66}
\renewcommand{\dbltopfraction}{.66}
\renewcommand{\dblfloatpagefraction}{.66}

% \urldef\myurlncsl1\url{foo%.com}
% \begin{document}
% text\footnote{WWW: \myurl}


\title{Voting lines and early voting check-in times in
  Florida\thanks{An earlier version of this manuscript was presented
    at the 2017 Annual Meeting of the Midwest Political Science
    Association.  The authors thank Michael Hanmer for comments and
    staff members from many Florida Supervisor of Elections offices
    for data on early voting check-in times.}\author{David
    Cottrell\thanks{Postdoctoral Research Fellow, Program in
      Quantitative Social Science, Dartmouth College.  6108 Silsby
      Hall, Hanover, NH 03755
      (\texttt{david.cottrell@dartmouth.edu}).} \and Michael C.\
    Herron\thanks{Visiting Scholar, Hertie School of Governance,
      Berlin, Germany, and Professor of Government, Dartmouth College.
      6108 Silsby Hall, Hanover, NH 03755-3547
      (\texttt{michael.c.herron@dartmouth.edu}).} \and Daniel A.\
    Smith\thanks{Professor of Political Science, University of
      Florida, 234 Anderson Hall, Gainesville, FL 32611-7325
      (\texttt{dasmith@ufl.edu}).}}\vspace{1cm}}

\maketitle \doublespacing 

% \topskip0pt
% \vspace*{\fill}
% \begin{center}
%   \Large{\textbf{Word count: 8,790}}
% \end{center}
%  \vspace*{\fill}

\begin{abstract}
  \noindent 
  Lines can constitute meaningful, albeit unfortunate, aspects of
  individuals' voting experiences. The causes and consequences of
  lines are the subject of ongoing research in the field of election
  administration, and we use early voter check-in times from Florida
  in the General Elections of 2012 and 2016 to study the times at
  which voters cast their ballots and to estimate the effect of
  waiting in line on future turnout.  Our check-in times---involving
  879,709 early in-person voters from 2012 and 1,492,350 from 2016---highlight
  the disproportionately problematic experiences faced by minority
  voters, although we find many fewer such problems in 2016 compared
  to 2012.  Moreover, we estimate that Florida early in-person voters who waited
  excessively in 2012 had a very slightly lower probability of voting
  in 2016, \emph{ceteris paribus}.  Our results draw attention to the
  continued importance of voting lines and the potential effect they
  have on future political activity.
\end{abstract}

% source for numbers in abstract:
%
% > nrow(evid12)
% [1] 879709
% > nrow(evid16)
% [1] 1492350
% > 

\newpage
\section*{Introduction}

The majority---over 75\% in 2016---of voters in the United States who
cast ballots in presidential elections do so in-person, either on
Election Day itself or during early voting periods whose durations
depend on relevant state or local laws.\footnote{For details on the
  2016 electorate, see \citet{eac:2016}} This style of voting---as
opposed to voting via mail---requires traveling to a voting location,
which is often called a precinct, possibly waiting in a line, possibly
authenticating oneself with a permitted form of identification,
physically registering candidate and ballot measure preferences, and
submitting a ballot for tabulation.  An individual's overall voting
experience thus has a number of facets, and our interest here is the
second of the steps noted above, namely, the act of waiting in line
before being processed to vote.

%% 2016 EAC is from p. 10

We draw attention to two important features of voting lines, by which
we mean lines that form \emph{in front of} precincts as opposed to
lines in which voters, already in the act of casting ballots, are
forced to stand.  First, waiting in line to vote constitutes a time
tax \citep{mukherjee:timetax}.  This tax can be negligible---for
example, a voter waits a scant ten seconds prior to initiating her
voting process---or imposing---some of the Florida voters we describe
shortly waited over four hours to vote in 2012. Associated with a
voting line time tax is the type of distributional question that is
associated with all forms of taxation: is the voting time tax fair
and, in particular, is the burden of this tax spread uniformly across
voters or concentrated on certain types of individuals?

Second, waiting in line is an experience that in principle can have
``downstream'' effects, as described by
\citet{pettigrew:longlinesminorityprecincts}. When an individual is
forced to pay a high tax in order to participate in a social or
political activity, a natural response might be to avoid the activity,
or substitute for it, in the future. To make matters concrete, if a
line in front of a particular restaurant is long, some individuals may
substitute an alternative eatery for the restaurant in question, one
without an imposing line.  On the other hand, a line in front of a
restaurant might be a quality signal, in which case the presence of a
line might be an incentive to eat at said restaurant.  While it is not
\emph{a priori} clear that voting is substitutable in the way that
selecting a restaurant might be, we should not dismiss outright the
possibility that the effect of waiting in line to vote, which raises
the cost to vote, might decrease the likelihood of voting in the
future.  Given the relatively low rate of voter participation in
American presidential elections \citep{IDEA:turnout}, any aspect of
the voting experience that might have a depressive effect on turnout
should be considered a matter worthy of study.

%% Need to add cite comparing American pres turnout with voting in other industrial democracies
%%[DAS: added: International Institute for Democracy and Electoral Assistance http://www.idea.int/data-tools 

These two features of voting lines---time tax and potential downstream
consequences---are conceptually distinct. The time tax for voting
could be uniformly spread across voters, which might be normatively
pleasing, and yet there could be a significant effect of waiting in
line on future electoral participation, which might be less so. Or,
the time tax could be concentrated on certain types of voters, which
most would argue is not fair regardless of whether there are
downstream consequences of waiting to vote. With this in mind, our
forthcoming analysis of voting lines during early in-person voting
periods in Florida is divided into two sections. The first addresses
questions associated with the notion of waiting in line as a time tax,
and the second considers the consequences of this tax.

One of the difficulties in studying voting lines in the United States
is the lack of data on who waits to vote and for how long. It is not easy to precisely determine
how long voters wait in line at the polls; it is even more
difficult to study if there are consequences of waiting in line on
subsequent voting behavior.  Researchers have dealt with this issue 
by using surveys---either Election Day exit polls or post-election
questionnaires---which can foster access to wide swaths of voters and
integrate responses to waiting time questions with other survey items,
namely socioeconomic and partisan queries
\citep{stewart:waitingtovote2012}.  A limitation of surveys regarding
electoral experiences, though, is that self-reports of voting wait
times may not be accurate.  Moreover, reports of turnout in general
may suffer from social desirability biases
\citep{karpbrockington:overreport}, \citep{bellietal:overreport}.

In light of potential issues with survey-based evidence on voting
lines, we turn here to a relatively untapped source of data, namely,
voter check-in times. Specifically, we draw on data collected by
several Florida counties that recorded check-in times from early
voting sites in the 2012 and 2016 General Elections.  As will be clear
shortly, the thousands of check-in times which form the basis of our
research are not subject to biases involving voluntary self-reporting.
This is advantageous, as is our ability to link a voter's check-in
time with her race/ethnicity, party registration, and electoral
participation history, all of which are part of every Florida voter's
registration file.  However, in the interests of being open about
limitations in our research design, a voter's check-in time is not
directly connected to the amont of time she spent waiting in line.  We
describe shortly how we associate voter check-in times with waiting
times, but the connection is not perfect.  Our use of early voting
check-in times thus complements the literature's reliance on survey
data, and we believe that researchers engaged in election
administration projects benefit when multiple sources of data are used
in line-based research.

Briefly, our results are as follows.  Regarding the time tax
associated with waiting to vote, we find disproportionate
concentration of this tax on minority voters.  The situation was worse
in Florida in 2012 than in 2016, and in general our data highlight
fewer troubling issues in the latter.  This is consistent with
national, survey-based evidence on voting lines in the 2012 and 2016
General Elections.\footnote{For a national 2012 versus 2016
  comparison, see
  \url{http://electionupdates.caltech.edu/2016/12/14/this-just-in-lines-at-the-polls-shorter-in-2016-than-in-2012}
  (last accessed March 30, 2017), as well as
  \url{http://www.richmond.com/opinion/their-opinion/guest-columnists/fortier-and-palmer-column-who-waits-the-longest-to-vote/article_81efabad-aa23-577b-83d1-0e524cf844ef.html}
  (last accessed June 30, 2017).}  Whether the recent decrease in
waiting times reflects progress in Florida election administration,
substitution effects and voter sorting, or idiosyncrasies from either
the 2012 or 2016 General Election is not clear.  Two election-years of
data are unlikely to be sufficient to gauge broad progress in a state
as large and heterogeneous as Florida.  Regarding the effect of
waiting to vote on future electoral participation, conditional on our
strategy for identifying those Florida early voters who suffered long
waits in 2012, we find small yet negative consequences regarding
turnout in 2016. The consequences are not completely negligible
insofar as long lines do appear to be associated with lower
participation levels, but they are small.  Our results have both
positive and normative implications, and we discuss these at length in
our conclusion.

The next section situates our study in the literature on voting lines
and election administration.  We then turn to early voting check-in
times in Florida and explain precisely what these times mean and how
they can be interpreted.  Our results are divided into two sections,
first on the distribution of the voting time tax and the second on the
effects of the tax.  The last section of the paper discusses our findings.

\section*{Voting lines in American elections}

``To me,'' Cynthia Perez complained after seeing a long line wending its way around an early voting center in Maricopa County, Arizona, with those in line reporting standing in line for more than three hours before checking in to vote, ``this is not what democracy is about'' \citep{santos:AZ2016}. Ahead of the state's March 2016 presidential primary, election administrators in Maricopa County had cut Election Day polling places by roughly 70 percent, which put greater strain on early voting centers to process voters. Voters in a handful of states also experienced long lines in the 2016 General Election, but it was nothing compared to what transpired four years earlier in Florida.  In 2011, the Florida state legislature curtailed the state's early voting period, which, not surprisingly, resulted  in long lines for many electors who tried to vote in person during the truncted early voting period in the 2012 General Election \citep{cite{herron_smith2014}. Following the election, then-president Barack Obama formed a Presidential Commission in early 2013 to address general election administrative issues in the United States. In doing so, he highlighted the plight of 102-year old Desiline Victor, a Haitian-American woman, who on October 27, 2012, waited in line for nearly four hours at the North Miami Public Library early voting center.  Not surprisingly, Obama's Presidential Commission devoted some of its January, 2014, report to long lines, concluding that, ``[A]s a general rule, no voter should have to wait
more than half an hour in order to have an opportunity to vote'' \citep[p.\ 13,][]{pcea:2014}.\footnote{The executive order creating 
this commission is available at
  \url{https://obamawhitehouse.archives.gov/the-press-office/2013/03/28/executive-order-establishment-presidential-commission-election-administr}
  (last accessed July 12, 2017).}

For their part, scholars have tried to better understand and explain why long lines at the polls form. Though still nacent, the literature on voting lines in American elections is growing. Focusing primarily on recent elections, 



%% XXX To be written: why should we care about lines?  Review
%% Pettigrew, balking, reneging.  Affect participation?  Decrease
%% legitimacy?    Make people angry?


With few exceptions, what we know to date about congestion at the
polls and ancillary wait times for voters is derived from survey
data. Most notably, \citet{stewart:waitingtovote2012} and his
coauthors have conducted post-election, Internet-based surveys in the
United States following the last three general elections.  These
surveys query voters about their experiences at the polls, including
estimated wait times.  \citeauthor{stewart:waitingtovote2012}'s
\emph{Survey of the Performance of American Elections} (SPAE), which
surveys 200 individuals in each state and Washington, D.C., along with
the Cooperative Congressional Election Study (CCES), have been used to
gauge relative wait times across states as well as overall wait times
for different sub-populations of voters.  According to
\citeauthor{stewart:waitingtovote2012}, wait times across the states
and D.C.\ tend to vary considerably across states, but a state's
average wait times tend to be consistent over time.

Surveys relying on self-reported wait times can serve as valuable
barometers for gauging both individual-level and regional
distributions of voting delays.  For a decade, Floridians have
consistently reported having to endure some of the longest wait times
in the country. In the 2012 General Election, for example, in-person
voters reported on average waiting 39 minutes to cast a ballot, three
times the national average \citep{stewart:waitingtovote2012}. In addition, 
surveys can help identify when during the voting process wait times are most likely to occur.
For example, in the 2012 General Election, \citeauthor{stewart:waitingtovote2012}'s SPAE 
documented that over three-fifths of early in-person voters nationwide who reported 
waiting in line to vote said their wait was primarily during the check-in stage.

Surveys are also helpful when trying to tease out the possibility of
differential voting line time taxes across sub-populations of voters.
Drawing on 2008 CCES data, \cite{mukherjee:timetax} finds minority
voters were more likely, relative to White voters, to pay such a tax
when queuing to vote. \citet{kimball:voting}, using the 2012 SPAE,
reports that voters in urban areas faced longer lines than rural
voters, and \citet{pettigrew:racegapwaittimes} estimates that
typically non-white voting locations in the United States were
associated with voter wait times approximately twice as long as those
in white locations.  \cite{herron:confidence} use exit polls in
Miami-Dade in 2014 to gauge the polling place experiences of voters
leaving polling stations.  In his study of the 2012 General Election,
which built on the 2008 General Election research design of
\citet{alvarez:survey}, \cite{stewart:waitingtovote2012} finds that a
voter's race is an important ``individual-level demographic'' that
explains disparate wait times.  ``African Americans waited an average
of 23 minutes to vote,'' \citeauthor{stewart:waitingtovote2012} found,
``compared to only 12 minutes for Whites; Hispanics reported waiting
19 minutes, on average.''  \citeauthor{stewart:waitingtovote2012}
concluded that these differences in wait times could be ``due to
factors associated with where minority voters live, rather than with
minority voters as individuals'' (pp.\ 457-458).

There are several limitations with survey data when used to estimate
voter wait times at the polls. As we mentioned at the outset, surveys
draw on voter self-reports; respondents might not recall accurately
how long they waited in line before voting, and social desirability
may confound accurate reporting on what has become a controversial,
and at times political, issue.  Second, surveys often do not
distinguish among the various wait times that voters may experience
when parking, queuing up to check-in, filling out ballots, or
processing completed ballots. Third, not all
surveys have access to tens of thousands of respondents; given
variability across states and even local jurisdictions in election administration practices,
national surveys may not be not particularly well-suited for assessing
variation in wait times across polling places within a state, or even within smaller designation, such as a ZIP Code.

In light of the limitations with survey data, our approach draws on an
alternative source of data---the check-in times of voters in
Florida who cast their ballots at early in-person polling sites prior to the 2012 and 2016 General Elections. 
In 2016, more Florida voters---3.88 million---cast early in-person ballots than voters who mailed in their 
absentee ballots (2.76 million) or who voted in person at their local Election Day precincts (2.96 million)
\citep{FDOS:2016vote}. The prevalence of early in-person voting
is not unique to Florida. Across the United States, early voting is increasingly popular; for
analyses of early voting reforms and their consequences, see 
\citet{neelyrichardson:earlyvoting}, \citet{gronke:2012}, 
\citet{gronke:earlyvotingreforms}, \citet{gronketoffey:psychological},
\citet{gronkebaum:growth}, and \citet{burdenetal:unanticipated}.
Although there are debates in the literature about the extent to which early
voting has changed the electorate, if at all, here we take this
mode of voting as given. Beyond Florida, there is good reason to focus on wait times 
during early in-person voting.  Michael P. McDonald, on his Election Project website,
estimates that over 23 million voters cast early in-person ballots in
``advance'' of the November 8, 2016, General Election, some 17 percent
of the 137 million votes cast.\footnote{The 2016 estimates are drawn from
 \url{http://www.electproject.org/home/voter-turnout/voter-turnout-data}
 (last accessed March 31, 2017).}  If one were to be concerned that
our focus on early voters in Florida limits our scope, these
statistics imply that our results nonetheless apply to millions of
Americans.

More generally, Florida's open records laws make the state an excellent laboratory for
the study of election administration, including wait times. For example, in their study of congestion at the
polls in Florida in the 2012 General Election, \cite{herronsmith:closingtimes} obtained precinct-level
data detailing when the last voter in a precinct checked in to vote,
when the final voter in a precinct cast a ballot, or when the last
voting optical scan machine in a precinct was shut down. They find significant 
differences in the closing times of precincts within counties, contingent on a 
precicnt's racial and ethnic composition.  In addition, Florida's open records laws provide 
a wealth of information not only about who votes, but when and how voters cast ballots. 
For instance, drawing on publicly available voter demographic and vote history data in Florida, \cite{amos_etal2017} 
find differences in individual-level turnout across political party and racial and ethnic categories 
among registered voters whose precincts were eliminated or relocated by local election administrators. 
Similarly, by linking multiple statewide voter files and individual-level early voting files, \cite{herron_smith2014} 
are able to demonstrate how reductions in the number of early voting days following the 2008 General Election 
led to a drop in early voting in 2012, especially among minority voters who had cast ballots on the final 
Sunday of early voting in 2008, a day of early voting that was eliminated by the state legislature ahead of the 2012 election. 
When it comes to direct election monitoring, however, Florida grants considerably greater privacy to voters. 
With the exception of those serving as candidate, political party, or ballot
issue representatives, state law prohibits precinct observers
from tracking voting processes inside a polling place when votes are
actively being cast. This effectively precludes scholars who want to
understand why lines form in front or inside Florida precincts
from replicating observational studies of voter activities in the vein
of \possessivecite{spencermarkovits:renege}'s study of California and
\possessivecite{herronsmith:hanoverstudy}'s research on New Hampshire.


\section*{Florida early voting check-in times}

Election administration in the Sunshine State---from registering
voters, to determining precinct sizes, to staffing and locating
polling places, and to setting early voting days and hours---is
largely controlled at the county level within a framework established
by the Florida state legislature. The primary data source in our
analysis is thus gleaned from Florida counties, and it consists of
early voting check-in times across early voting polling stations
during the 2012 and 2016 General Elections.  We now explain what these
times represent.

In 2012, all 67 of Florida's counties offered between six and 12 daily
hours of early voting over an eight-day period (Saturday through
Saturday) that ended immediately prior to the General Election,
November 6, 2012.  In 2016, after long-lines in 2012 had apparently
convinced the state legislature to grant counties more flexibility in
offering early voting opportunities, counties were permitted to offer
up to 12 hours of early voting per day, spread over 14 days, ending
the final Sunday before the November 8, 2016, General Election.
Counties were permitted to offer a maximum of 168 total hours of early
voting.\footnote{See Florida Governor Rick Scott's statement, issued
  in the aftermath of the 2012 General Election, available at
  \url{http://www.flgov.com/2013/01/17/governor-rick-scott-statement-on-election-reforms/}
  (last accessed March 31, 2017).}

A Florida voter who wishes to cast his or her ballot early may vote at
any early voting polling location in the county in which he or she is registered.  This is
distinct from in-person, Election Day voting, during which a
registered Florida vote must vote at his or her assigned precinct.
Rather than relying on traditional, paper-based pollbooks to check-in
voters in local precincts, as is the case on Election Day in some
Florida counties, the 67 county Supervisors of Elections in Florida
use electronic pollbooks to check-in voters during the state's early
voting period.  These pollbooks are for the most part known as
Electronic Voter iDentification machines, or EViDs for short.
Although there is variance across Florida counties in electronic
pollbook implementation, for simplicity we refer to all electronic
pollbooks as EViD machines.\footnote{For example, Sarasota County uses
  a different electronic voting system for its early voters (phone
  call with Cathy Fowler, office of the Sarasota Supervisor of
  Elections, on March 31, 2017).}

%% Cathy Fowler:  cfowler@sarasotavotes.com

EViD machines allow county pollworkers to check-in and verify the
registration statuses and identities of voters casting early,
in-person ballots. The EViD system is designed to reduce the time it
takes to process early voters, and it fosters synchronization across a
county's early voting centers as well as with the Florida statewide
voter database.  This can thwart potential double-voting.  More
importantly for our purposes, EViD machines clock the check-in times
for all early voters, providing us with effective timestamps that
specify when a voter began his or her voting process. EViD timestamps
are recorded to the second, and this allows us, for example, to
pinpoint early voters who had not checked-in when polls closed at
7:00pm on a given day of early voting but nonetheless cast ballots.
Per Florida state law, any elector in line at 7:00pm is allowed to cast
a ballot \citet{FLStatutes:7pm}.

EViD timestamps are not subject to potential self-reporting biases and
hence are presumably more accurate than surveys of voter times or even
within-polling place observations of when voters checked-in.
Moreover, EViD timestamps include Florida voter identification
numbers, which we can directly link to the nearly 14 million
individual voter registration records in Florida's statewide voter
file.  To this end, we merge our EViD with statewide voter files that
cove the 2012 and 2016 General Elections.  

%%  David: footnote about
%%  merge.  Did we drop any early
%%  voters?  How many early voters do
%%  note appear in the voter file?

Despite their advantages, EViD check-in times are not a panacea in the
quest to understand voting lines and their consequences.  We
ultimately care about voter wait times, and in this sense our use of
EViD check-in times has limitations.  An individual check-in time
indicates when a given voter finished waiting in a line, if there were
a line to begin with.  However, such a time does not specify how long
an individual waited to vote, if at all. As such, our analysis of check-in times serves as a complement to
studies using survey and other observational methods to explain who waits in line and for how long.

Figure \ref{fig:tenminhist} displays the distribution of EViD check-in
times on the last Saturday (November 3) of the 2012 General Election
early voting period for two different polling locations in Florida,
one at the Fred B.\ Karl County Center in Hillsborough County and the
other at the West Kendall Regional Library in Miami Dade County.  The
histograms in this figure describe in ten minute blocks the number of
voters who checked into these two polling stations.  The counts begin
with the first voter who checked-in just as polls opened at 7:00am,
and they end with the last voter who checked-in (regardless of whether
this individual checked-in before or after the official 7:00pm
deadline for poll closing).

\begin{figure}[!ht]
  \caption{Early voting check-in times on Saturday, November 3, 2012, in two Florida locations}
  \label{fig:tenminhist}
  \centering
  \begin{subfigure}[b]{\linewidth}
    \centering\includegraphics[scale = 0.6]{example00.pdf}
    \caption{Fred B.\ Karl County Center, Hillsborough County}
    \label{fig:karlexample}
  \end{subfigure}%
  \\
  \begin{subfigure}[b]{\linewidth}
    \centering\includegraphics[scale = 0.6]{example01.pdf}
    \caption{West Kendall Regional Library, Miami Dade County}
    \label{fig:kendallexample}
  \end{subfigure}
\end{figure}

Several features of Figure \ref{fig:tenminhist} are notable.  First,
both panels in the figure contain horizontal red lines at 7:00pm.
This is the time beyond which additional voters were not allowed to
join a (possibly existing) voting line.  In the Hillsborough County
location (Figure \ref{fig:karlexample}), no check-ins occurred after
7:00pm, and from this it follows that, at 7:00pm, where was no voting
line.  In contrast, the Miami-Dade location (Figure
\ref{fig:kendallexample}) had many post-7:00pm check-ins, the last one
of which occurred around 1:00am on Sunday, November 4.  This time is
noted with a dashed red line.  We thus know that the last-voting voter
at West Kendall Regional Library waited at least almost six hours to
vote.

Figure \ref{fig:tenminhist} also includes information on voter
self-reported race, and racial details are depicted via bar colors.
Early voters at the Fred B.\ Karl County Center were primarily white,
although there were periods on November 3 when the fraction of
non-white votes was disproportionately high, e.g., toward the end of
the date.  At the West Kendall Regional Library, though, the vast
majority of early voters were non-white.  The fraction of non-white
voters appears roughly consistent across November 3.

Lastly, and very roughly speaking, we observe a flatter or
more-uniform distribution of check-in times at Fred B.\ Karl County
Center than in West Kendall Regional Library.  We will draw on this
fact later, when we try to determine which voters waited in line, and
here we provide some intuition.  Overall, and as illustrated in Figure
\ref{fig:tenminhist}, It appears that early voting locations in
Florida that shut down very late had flatter distributions of check-in
times.  We suspect that this is evidence of persistent lines.  In
contrast, the non-uniformity of check-in times that we observe in
Figure \ref{fig:karlexample} is consistent with more of an ebb and
flow of early voters.  We know that there was not a line to vote at
Fred B.\ Karl County Center at 7:00pm on Saturday, November 3, this
despite the fact that, per check-in times, the Center was regularly
processing voters.

Figure \ref{fig:tenminhist} highlights the value of EViD check-in
times (timestamps, association with race, and so forth) as well as
their limitations (check-ins are not linked to arrivals).  All data
sources have advantages and disadvantages, and we will draw on the
former and attempt to work with the latter as we turn to results.

\section*{Results}

We present results in two sections.  First, we describe patterns in
check-in times across the 2012 and 2016 General Elections with
particular attention to race and partisanship.  Second we consider the
consequences of extensive early voting wait times based on a
identifying assumption we use to characterize individuals who waited
before voting.

\subsection*{Who waits?}

The election participation history data included in Florida statewide
voter files do not contain EViD check-in times.  For these times we
have to rely on public records requests made to individual county
Supervisors of Elections.

\subsubsection*{Early voting in 2012}

For the 2012 General Election, we have been able to obtain 2012 EViD
check-in data from six Florida counties: Alachua, Broward,
Hillsborough, Miami Dade, Orange, and Palm Beach.  Altogether these
counties had 82 total early voting locations---akin to precincts,
except that voters in a given county may cast early ballots in the
county---that serviced a total of 879,709 voters in the eight day
long, 2012 early voting
period.  % David to-do: insert footnote here about inferring locations

%> unique(evid12$county)
%[1] "ALA" "BRO" "HIL" "DAD" "ORA" "PAL"

%> length(unique(evid12$location))
%[1] 82

\begin{figure}[!ht]
  \caption{Number of locations where early votes were cast, 2012
    General Election}
  \label{fig:nrlocs2012}
  \centering
    \centering\includegraphics[scale = 0.8]{number_of_Locations.pdf}
\end{figure}

Figure \ref{fig:nrlocs2012} describes by day of early voting and by
hourly window the number of locations across our six counties that
actively served voters.  All locations served early voters virtually
the entire day, and this is evident in the flat line, for the most
part pegged a bit under 80 prior to 7:00pm, at the top of the figure;
the sole exception occurred at the earliest time of the day, during
which a few early voting stations did not have any active voters.

After 7:00pm, however, Figure \ref{fig:nrlocs2012} shows that check-in
uniformity across locations quickly changed.  On early days within
Florida's eight-day long early voting period, there was a steep drop
in active voting locations starting at 7:00pm; note the vertical red
line at this time.  Even with this drop, however, there were still at
least five open locations at 9:00pm on every day of early voting in
the early voting period.  Then, on the last two days of early voting,
November 2 and 3, many early voting locations were open well beyond
7:00pm.  On the final Saturday of early voting, over 30 locations were
still open at 9:00pm, and several locations were continuing to process
voters at midnight.

\begin{figure}[!ht]
\caption{Distribution of check-in times among early voters by hour, 2012
  General Election}
  \label{fig:hist2012}
  \centering
    \centering\includegraphics[scale = 0.8]{histogram_by_hour.pdf}
\end{figure}

Figure \ref{fig:hist2012} displays the total number of voters who
voted early by the hour of the day at which they checked-in. Prior to
7:00pm, voters check-ins were distributed fairly uniformly across the
day with a small peak before noon.  From 7:00am to 7:00pm, our
aggregated four counties consistently served approximately 50,000
voters per hour. Then, after 7:00pm, at which point new voters could
not join voting lines, the number of voters served per hour dropped
drastically but not entirely.  A total of 54,329 (7.8\%) early voters
in 2012 cast ballots after 7:00pm. These are individuals who were in
line at the time the polls closed and were allowed to continue to wait
to vote.  We know that all of these voters had to wait in this way
although we do not know precisely how long each waited.  Of this
group, 24,294 voted after 8:00pm.  These voters waited at least one
hours to vote, and 9,684 voters, who checked-in after 9:00pm, must
have waited at least two hours to vote.

% > vte %>% filter(time > 19) %>% count()
% # A tibble: 1 × 1
%       n
%   <int>
%   54329

% > vte %>% filter(time > 20) %>% count()
% # A tibble: 1 × 1
%       n
%   <int>
%   24294

% > vte %>% filter(time > 21) %>% count()
% # A tibble: 1 × 1
%       n
%   <int>
%   24294

\begin{figure}[!ht]
\caption{Racial composition of early voters by hour, 2012 General Election}
  \label{fig:race2012}
  \centering
    \centering\includegraphics[scale = 0.8]{racial_composition.pdf}
\end{figure}

Aggregating across locations, Figure \ref{fig:race2012} describes the
composition of the 2012 early voting pool by race and by hour of
check-in.  For most of the day, whites were the majority racial group,
followed by blacks, Hispanics, and Asians.  This ranking is similar
to, but does not mirror, Florida's registered voter pool.  In the
Florida voter file as of December, 2012, which contains 12,580,602
individuals, approximately 66.4 percent are white, 13.9 percent
Hispanic, 13.6 percent black, and 1.63 percent Asian.  As in years
prior, blacks in Florida in 2012 were disproportionate users of the
state's early voting period \citep{herronsmith:souls}.

% mysql> select count(*) from fLvoterfile_dec_2012_extract;
% +----------+
% | count(*) |
% +----------+
% | 12580602 |
% +----------+
% 1 row in set (4.20 sec)

% mysql> select race, count(*) AS num, round(100 * count(*) / (select count(*) from fLvoterfile_dec_2012_extract),2) AS percent from fLvoterfile_dec_2012_extract group by race;
% +------+---------+---------+
% | race | num     | percent |
% +------+---------+---------+
% |    0 |     220 |    0.00 |
% |    1 |   42165 |    0.34 |
% |    2 |  204542 |    1.63 |
% |    3 | 1711107 |   13.60 |
% |    4 | 1748251 |   13.90 |
% |    5 | 8358730 |   66.44 |
% |    6 |  203878 |    1.62 |
% |    7 |   29860 |    0.24 |
% |    8 |       4 |    0.00 |
% |    9 |  281839 |    2.24 |
% |   10 |       5 |    0.00 |
% |   11 |       1 |    0.00 |
% +------+---------+---------+
% 12 rows in set (5.72 sec)

% mysql>

% RACE CODES
% Race Code	Race Description
% 1	American Indian or Alaskan Native
% 2	Asian Or Pacific Islander
% 3	Black, Not Hispanic
% 4	Hispanic
% 5	White, Not Hispanic
% 6	Other
% 7	Multi-racial
% 9	Unknown

What is striking in Figure \ref{fig:race2012}, however, is the racial
composition of the early voting pool immediately before and then after
7:00pm.  Simply put, the pool becomes rapidly non-white starting
around 7:00pm and by the 8:00pm-9:00pm window is less than 15 percent
white.

\begin{figure}[!ht]
\caption{Partisan composition of early voters by hour and by race, 2012
  General Election}
  \label{fig:party2012}
  \centering
    \centering\includegraphics[scale = 0.8]{partisan_composition_by_race.pdf}
\end{figure}

The partisanship of the early voting pool varies slightly with time
although not nearly as starkly as its racial composition.  Evolution
in partisanship is illustrated in Figure \ref{fig:party2012}, which
describes by early voting hour the partisan breakdown of all early
voters in our four counties as well as the partisan breakdown of the
four aforementioned racial groups.  Partisanship here is measured by
party registration and is plotted as the percent of early voters in
each hour of check-in that are registered Democrat.  While Democrats
made up 56\% of all those who voted early in our set of Florida
counties, we can see from the black line in the figure that this
percentage varied by hour of the day.  Notably, Democrats composed a
greater share of the voters in both the early hours of voting and the
hours after 7:00pm.  Hence, Democrats were disproportionately affected
by voting lines that forced voters to cast their ballots after 7:00pm.

%  XXX Need to check 56% figure above.

If we consider the breakdown of partisanship by race, as in in Figure
\ref{fig:party2012}, we can see how this effect might largely be due
to the aforementioned racial patterns of voting.  For example, black
early voters in Florida were almost entirely Democratic, regardless of
when they checked-in to vote.  Similarly, the partisanship of Hispanic
voters---the least Democratic group---remained relatively constant
throughout the day.  However we do see a slight trend for White and
Asian early voters, who became increasingly Democratic as time
progressed.

%> vte %>% summarize(pct = mean(party == "DEM"))
%# A tibble: 1 × 1
%       pct
%     <dbl>
%1 0.564601

\subsubsection*{Early voting in 2016}

We now turn to early voting in 2016.  By the General Election in 2016,
Florida had increased the number of early voting days from eight to
14, ostensibly in an effort to reduce congestion. Was this change
effective?  Figure \ref{fig:nrlocs2016} is analogous to the earlier
figure that described active early voting locations.  Here we have
data from the same counties as in 2012.  These six counties had 106
early voting stations in place for the 2016 General Election.

% > length(unique(evid16$location))
% [1] 106

\begin{figure}[!ht]
  \caption{Number of locations where early votes were cast, 2016 General
    Election}
  \label{fig:nrlocs2016}
  \centering
    \centering\includegraphics[scale = 0.8]{number_of_locations_2016.pdf}
\end{figure}

Compared to its 2012 version, the most important aspect of Figure
\ref{fig:nrlocs2016} is the pictured dropoff in voter check-ins that
occurred after 7:00pm.  There were indeed check-ins after this time
but not nearly as many as in 2012.  On the busiest day in 2012, more
than half of the polling locations were open past 9:00pm and five were
open past midnight.  On the busiest day in 2016, though, only two
locations were open past 8:00pm and only one early voter cast a ballot
past 9:00pm.  In the introduction we noted the improvements in voter
wait times that others have found in 2016 compared to 2012, and Figure
\ref{fig:nrlocs2016} is consistent with national, survey-based
findings on voter waiting.  Early voting in 2016 appears to represent
a major improvement over 2012 in terms of reducing
congestion.\footnote{We use the verb ``appears'' here because of
  voting-level sorting that may affect with individuals cast their
  ballots.  When the Florida legislature changed the state's early
  voting period between 2012 and 2016, voters may have altered their
  most preferred voting times.}

\subsubsection*{Late voting in 2012 versus 2016}

Figure \ref{fig:race2012and2016} provides a race-based perspective on
the 2012 versus 2016 comparison for our six counties.  For three key
races groups in Florida, black, Hispanic, and white, the figure shows
the total number of check-ins by time, aggregated across all days in
the 2012 and 2016 early voting periods.

\begin{figure}[!ht]
  \caption{Distribution of voter check-ins, 2012 General Election versus 2016 General Election}
  \label{fig:race2012and2016}
  \centering
  \centering\includegraphics[scale = 0.8]{histogram_by_hour_by_race_2012_2016.pdf}
\end{figure}

Figure \ref{fig:race2012and2016} shows clearly that, not only did more
voters check-in for early voting in 2016 compared to 2012, but fewer
voters voted past 7:00pm.  This is the case for black, Hispanic, and
white voters. Moreover, non-white early voters in Florida had
disproportionately late check-ins in 2012.  We already have seen
evidence of this, but Figure \ref{fig:race2012and2016} shows as well
how this problem did not appear in 2016.  There were slightly more
non-whites with late check-ins in 2016, but the magnitude of the white
versus non-white gap shrunk between the two years.

Even though 2016 attracted far more early voters than 2012, the
changes made to the early voting period appears to have reduced the
congestion. Although the increased number of days may have helped,
these three counties still served roughly the same number of early
voters per day in 2016 (approximately 54,933) as they did in 2012
(approximately 55,236).  Given reduced prevalence of lines in 2016, it
is likely that other reforms made the difference for congestion.

%> 441895/8
%[1] 55236.88
%> 769063/14
%[1] 54933.07


%% Make a plot using all early voters in all sites, but this is
%% post-midwest!  XXX

\subsection*{Effects of waiting to vote}

We now consider the effects on future political participation of
waiting in line to vote, in particular waiting in an early voting line
in the 2012 General Election.  The future participation we have in
mind is voting in the 2016 General Election.

We have already noted that the individual-level EViD files we have
include Florida voter identification numbers, and this enables us to
link these files with statewide Florida vote files.  These latter
files specify the elections in which voters participated and, if so,
how, e.g., voted absentee, voted early, and so forth.  Thus, for any
early voter in 2012 we can determine whether the individual voted in
the 2016 General Election, assuming that this individual still lived
in Florida as of November, 2016.\footnote{When a registered voter
  moves within Florida, the voter maintains the same voter
  identification number.  Our estimates in this section are thus not
  confounded by the possibility of 2012 early voters moving across
  county lines within Florida, for example.}  

The difficulty in our exercise here is twofold.  First, while we know
voter check-in times from our EViD data, we do not know associated
voter wait times \emph{per se}.  To determine wait times precisely we
would have to know early voter arrival times, which are not collected
by election officials.  Hence, distinguishing those early voters in
2012 who waited in line from those who did not is a challenge and
requires an alternative method along with some assumptions.  Second,
early voters who voted at, say, 9:00am on a given day of early voting
may be systematically different from those who voted at
6:00pm. Consequently, estimating the effect of voting after closing
time---when we know voters have waited in line---is potentially
confounded by voter-level selection effects.  Hence, we need to ensure
that we control for differences across early voters as best as we can
so that voter selection into time of early voting does not confound
our estimates of the effect of waiting on future participation.

Our approach is as follows.  We condition on race, partisanship (as
before, measured by party registration), gender, age, and previous
voting history.  And, we make the following assumption: those who
voted at early voting locations which stayed open well past the 7:00pm
cutoff time are more likely to have waited in line than those who
voted in voting locations that closed before 7:00pm or at this time
exactly.  Moreover, we assume that those who voted just before the
7:00pm cutoff in early voting locations that stayed open late are more
likely to have waited in line than those who voted earlier in the day.
We cannot know for sure if these assumptions hold, but there is reason
to believe that it does.

Take, for example, the two polling locations in Figure
\ref{fig:nrlocs2012}.  The first location---the Fred B.\ Karl County
Center in Hillsborough County---closed on time on November 3,
2012. That is to say, voters in this center voted up to, but not
beyond, 7:00pm.  Therefore, we assume that the Karl Center was not a
congested polling location and that the voters who voted just before
7:00pm did not have to wait in much of a line.  If they did have to
wait in line to vote, then the line remarkably stopped just in time
for the final early voter to check-in right before 7:00pm.  While this
is technically possible, it woud be remarkably coincidental.

On the other hand, the second location in Figure
\ref{fig:nrlocs2012}---the West Kendall Regional Library in Miami Dade
County---remained open until the last early voter checked-in around
1:00am on November 3.  We know with certainty that this last voter
waited in line to vote.  In fact, because all voters had to arrive
before 7:00pm in order to check-in, we know with certainty that he or
she waited at least six hours to vote.  Moreover, we can be fairly
confident that the voters who voted just before 7:00pm here also
waited in line.  Voting that continues past 7:00pm is indicative of a
polling place that was operating essentially at capacity at 7:00pm.
Hence, those voters who voted just before 7:00pm are similarly likely
to have been caught up in a long voting line.

With the above assumptions as background, our strategy for
distinguishing voters who waited in line from voters who did not wait
in line is first to distinguish those who voted in a polling place
that closed well past 7:00pm from those who voted in a polling place
that closed on time.  In other words, we attempt to compare voters who
cast their ballots in places like the Fred B.\ Karl County Center to
voters who voted in places like the West Kendall Regional Library.
More specifically, we want to compare those individuals who voted
closest to the 7:00pm cutoff, since those are the individuals who we
are most confident were affected by the congestion that caused polling
locations to close late.
 
Therefore, we identify all early voters who voted at a polling
location on a day where the last voter checked-in past 7:30pm.  These
are voters who cast ballots at locations where we know the line at the
end of the day was at least a half-hour long.  In what follows we call
this variable \emph{Over}.  Conditioning on race, partisanship,
gender, age, and previous voting record (whether the voter voted in
the 2008 election) we estimate the effect of voting at a polling
location that is congested in 2012---one that goes ``over''
7:30pm---on the probability of voting in 2016.  We employ a logistic
regression as follows:

\begin{equation*}
  \begin{aligned}
    \textup{Pr}\left(\mathrm{Voted16}_{i} = yes\right) =  \textup{logit}^{-1}&(\beta + \alpha_{Over} + \gamma_{Hour} +
    \sigma_{Over \times Hour} + \upsilon_{Gender}  + \\& \rho_{Race} +
      \tau_{Age Group} + \psi_{Party} + \pi_{Voted\emph{08}})
  \end{aligned}
\end{equation*}

In the above model, $i$ denotes our collection of 2012 early voters
who appear in the history of Florida voting in 2008; $Hour$ indicates
the hour of the day at which each voter checked-in, between 7:00am and
7:00pm; $Gender$ indicates whether the voter identifies as male or
female; $Race$ indicates if the voter self-identifies as white, black,
Hispanic, or Asian; $AgeGroup$ classifies voters according to their
2016 age and bins them into 10-year age groups (20-29, 30-39, 40-49,
50-59, 60-69, and 70+); and, $Voted08$ and $Voted16$ are indicators
for participation in the 2008 and 2016 General Elections.  All told we
estimate our logistic regression based on 709,956 individuals, and
coefficient estimates appear in Table \ref{tab:reg} in the appendix.

Regression results are displayed graphically in Figure
\ref{fig:prvoting2016}. In this figure, we have estimated the
probability of voting in 2016 for a male Floridian who is 50 years
old, black, registered Democratic, and who also voted in 2008.  The
estimates are conditioned on each hour of the early voting day in
2012. The black points in the figure are the estimated probabilities
of voting in 2016 conditional on voting at a polling location that
closed before 7:30pm.  Based on our discussion above, these estimates
reflect individuals who likely did not wait in line.  The grey points
are the estimated probabilities of voting in 2016 conditional on
casting an early ballot at a polling location that closed \emph{after}
7:30pm.  These estimates reflect individuals who likely did wait in
line to vote.  Hence, the difference between these two sets of
estimates should reflect the effect of waiting in line.  Moreover, the
effect should be most isolated as one compares voters casting ballots
closer to 7:00pm.

\begin{figure}[!ht]
\caption{Probability of voting in 2016, given 2012 check-in time}
  \label{fig:prvoting2016}
  \centering
    \centering\includegraphics[scale = 0.8]{probability_of_voting_in_2016_over_under.pdf}
\end{figure}

One can see that, beginning at around 2:00pm, there is a small albeit
negative difference between those who voted in congested polling
places compared to those in non-congested polling places.  This might
suggest that the lines started in the early afternoon and this is why
a difference is induced so early.  Hence, the reason there is no
difference between congested and non-congested polling places earlier
in the day might be because lines have yet to form. As congested
polling places become congested in the afternoon, we begin to see the
small but significant effect on future participation.  This result
suggests that voting lines have a negative effect on future
participation.  However, the effect is very small, amounting to no
more than a percentage point difference in turnout.  Our results are
similar to those in \citet{pettigrew:longlinesminorityprecincts}.

\section*{Conclusion}

The presence of lengthy voting lines in General Elections in the
United States has been of recent concern to scholars of election
administration. Burdensome wait times not only impose opportunity
costs, but they might also discourage future electoral participation.
Hence, researchers have been working to learn more about the
prevalence of long lines, who waits in them, and how the burden of
waiting might affect future turnout.

Research on voting lines, however, is challenged by the lack of data
on lines and wait times.  Other than recording the times at which
polls officially closed and recording the voters who voted late, there
is little additional data recorded by election officials that might
distinguish those who waited in line to vote from those who did not.
Notwithstanding a small number of exceptions, nor are there official
records on how long voters waited.  Previous research using election
records to study lines have been limited to looking at aggregate
precinct results \citep{herronsmith:closingtimes,
  pettigrew:longlinesminorityprecincts}.  Data limitations have
prevented these projects from identifying individuals who waited in
line from those who did not \emph{within} the same precinct.

To overcome the lack of records, scholars have relied on surveys that
ask respondents to recall their experience at the polls and report the
length of time that they waited to vote
\citep{stewart:waitingtovote2012, pettigrew:racegapwaittimes}.  While
these survey-based analyses have illuminated important variation in
wait times across geographies and across racial groups, there is
always the caveat that survey responses are subject to biases
associated with self-reporting and are limited in sample size.

With these points in mind, we have attempted to add to the existing
literature on voting lines by analyzing a new data set on early voter
check-in times in Florida's 2012 and 2016 general elections. Using
electronic voting check-in data from Alachua, Broward, Hillsborough,
Miami Dade, Orange, and Palm Beach Counties, we were able to identify
the exact day and time at which each voter checked-in to vote.  This
data not only provided novel insight into voting patterns across time
but also allowed us to identify individual voters who likely waited in
line to vote.

We have shown that Florida's early voting period in 2012 had a
significant number of voters that voted past the official 7:00pm
closing time and, therefore, had to wait in line to vote.  Moreover,
the voters were disproportionately black, Hispanic, and Democrat---a
finding that coincides with survey results in prior studies.  In 2016,
we found that congestion observed in 2012 had all but vanished.
Voters rarely voted late into the night in 2016, and most early voting
locations closed on time.  Even though the counties we studied served
as many individual per day as they had in 2012, their polling
locations appear to have been more prepared to handle the high rates
of voting.
  
As a second result of our analysis, we estimated the effect that
waiting in line has on future participation.  Using temporal variation
in voting across polling locations, we found a very slight negative
effect on participation, amounting to no more than a one percentage
point decrease in the propensity to vote. The result is similar to
that found by \citet{pettigrew:racegapwaittimes}.

%\section*{Appendix}
%\input{../plots/table_out.tex} 

\clearpage
\newpage

\bibliographystyle{apsr}
\bibliography{evid-cites}


\newpage
\appendix
\section*{Appendix}

\input{../plots/table_out.tex} 

\end{document}


 
